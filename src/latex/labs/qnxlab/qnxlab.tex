 \documentclass[11pt,twoside,a4paper]{article}

\usepackage[polish]{babel}
\usepackage{polski}
\usepackage[utf8]{inputenc}
\usepackage[OT4]{fontenc}
\usepackage{tgtermes}	%Times New Roman
\usepackage[left=2.5cm,top=2.5cm,right=2.5cm,bottom=2.5cm,
headsep=0.5cm,headheight=1.0cm,marginpar=2cm,reversemp]{geometry}
\usepackage{fancyhdr}
\usepackage[table]{xcolor}
\usepackage{graphicx}
\usepackage{amsmath}
\usepackage{amsthm,thmtools}
\usepackage[nottoc]{tocbibind}
\usepackage{ragged2e}
\usepackage{bbding}
\usepackage{makeidx}
\usepackage{titlesec}
\usepackage{tcolorbox}
\usepackage{url}
\usepackage{color}
\usepackage{setspace}
\usepackage[font=small,format=plain,labelfont=bf,up,textfont=it,up]{caption}
\usepackage{BeamerColor}
\usepackage{listings}
\usepackage{pdfpages}
\usepackage{hyperref}
\usepackage{tikz}

\usetikzlibrary{arrows,positioning}

% Define tikz styles
\tikzset{
  % Text box, 6em width, centered, rounded corners
  TBox6emCentered/.style={
    rectangle,
    rounded corners,
    draw=black, very thick,
    text width=6.5em,
    minimum height=2em,
    text centered}
}

\definecolor{mycolor1}{RGB}{0,0,128}
\definecolor{lightgray}{gray}{0.9}
\definecolor{lightlightgray}{gray}{0.95}
\definecolor{lightyellow}{RGB}{255,255,224}
\definecolor{lemonchiffon}{RGB}{255,250,205}



\newcommand{\mat}[1]{\boldsymbol{\mathrm{#1}}}

\hypersetup{
	pdftitle={Metody Programowania Robotów},
	pdfauthor={Pawe\l{} Malczyk, Pawe\l{} Tomulik},
	pdfkeywords={systemy operacyjne czasu rzeczywistego, QNX, robot},
	pdfsubject={Zestaw instrukcji laboratoryjnych QNX}
	pageanchor=true,
	breaklinks=true,
	plainpages=false,
	linktocpage=true
}
% zmiana nazw domyślnych
\addto\captionspolish
{
	\renewcommand{\tablename}{Tabela}
	\renewcommand{\listtablename}{Spis tabel}
	\renewcommand{\bibname}{Piśmiennictwo}
	\renewcommand{\chaptername}{Laboratorium}
}


%\newtheoremstyle{mystyle}% name of the style to be used
%  {12pt}% measure of space to leave above the theorem. E.g.: 3pt
%  {12pt}% measure of space to leave below the theorem. E.g.: 3pt
%  {\itshape}% name of font to use in the body of the theorem
%  {6pt}% measure of space to indent
%  {\bfseries}% name of head font
%  {\newline}% punctuation between head and body
%  {.5em}% space after theorem head; " " = normal interword space
%  {}% Manually specify head
%\theoremstyle{mystyle}
%\newtheorem{example}{\color{black}Przykład}[section]


\declaretheoremstyle[
spaceabove=6pt, spacebelow=6pt,
headfont=\normalfont\bfseries,
notefont=\mdseries,
notebraces={[}{]},
%bodyfont=\itshape,
postheadspace=1em,
%qed=\qedsymbol
]{mystyle}
\declaretheorem[name=Przykład,numberwithin=subsection,style=mystyle]{example}
%\declaretheorem[name=Definition]{definition}



%\definecolor{lightyellow}{RGB}{255,255,224}
%\definecolor{lemonchiffon}{RGB}{255,250,205}
\definecolor{syntax}{RGB}{127,0,85}
\definecolor{comments}{RGB}{63,127,95}
\definecolor{strings}{RGB}{42,0,255}

\lstdefinestyle{MyCStyle} {
    language=C, % choose the language of the code
    alsolanguage=C++,
    basicstyle=\linespread{0.9}\fontfamily{lmtt}\selectfont\small\color{black},
    keywordstyle={\bfseries\color{syntax}}, % style for keywords
    emph={int,char,double,float,unsigned,printf,getchar,putchar,
sprintf,scanf,fopen,fscanf,fprintf,fclose,pthread_self,pthread_create,sleep,exit,pthread_t,
pthread_exit,pthread_cancel,pthread_join,pthread_attr_init,pthread_attr_setdetachstate,pthread_attr_destroy,
pthread_attr_getdetachstate,pthread_attr_setdetachstate,pthread_attr_getinheritsched,pthread_attr_setinheritsched,
pthread_attr_getschedpolicy,pthread_attr_setschedpolicy,pthread_attr_getschedparam,pthread_attr_setschedparam,
pthread_attr_getscope,pthread_attr_setscope,pthread_attr_getstacksize,pthread_attr_getstackaddr,pthread_attr_setstacksize,
pthread_attr_setstackaddr,pthread_attr_t,srand,time,rand,pthread_mutex_init,pthread_mutex_t,pthread_mutex_destroy,
pthread_mutex_lock,pthread_mutex_timedlock,time_t,pthread_mutex_trylock,pthread_mutex_unlock,
pthread_cond_init,pthread_cond_destroy,pthread_cond_wait,pthread_cond_timedwait,pthread_cond_signal,pthread_cond_broadcast,
pthread_barrier_init, pthread_barrier_t,pthread_barrierattr_t,pthread_barrier_wait,pthread_barrier_destroy,
pthread_cond_t,pthread_mutexattr_t,pthread_condattr_t,ChannelCreate,ChannelCreate_r,
ChannelDestroy,ChannelDestroy_r,ChannelAttach,ChannelDetach,MsgReceive,MsgReply,MsgSend,strerror,ConnectAttach,ConnectAttach_r,
pid_t,uint32_t,ConnectDetach,ConnectDetach_r,MsgSend_r,MsgReceive_r,name_attach,name_detach,name_open,name_close,
open,MsgReply_r,atoi,strcpy,_uint16,_int8,_uint8,_int32,MsgSendPulse,MsgReceivePulse,clock_gettime,perror,
clock_getres,clock_settime,ctime,nanosleep,delay,select,alarm,nanospin,timer_create,timer_settime,timer_gettime,timer_delete,
getpid},
    emphstyle={\bfseries\color{syntax}},
    stringstyle=\color{strings},
    commentstyle={\fontfamily{lmtt}\selectfont\color{comments}},
    numbers=left, % where to put the line-numbers
    numberstyle=\tiny, % the size of the fonts that are used for the line-numbers
    backgroundcolor=\color{lemonchiffon},
%    backgroundcolor=\color{lightgray},
    showspaces=false, % show spaces adding particular underscores
    showstringspaces=false, % underline spaces within strings
    showtabs=false, % show tabs within strings adding particular underscores
    frame=single, % adds a frame around the code
    tabsize=2, % sets default tabsize to 2 spaces
    rulesepcolor=\color{gray},
    rulecolor=\color{black},
    captionpos=t, % sets the caption-position to bottom
    breaklines=true, % sets automatic line breaking
    breakatwhitespace=false,
    xleftmargin=20pt,
    xrightmargin=20pt,
    aboveskip=12pt,
    belowskip=12pt,
    escapeinside={(*@}{@*)},
%   frameround=tttt,
   framexleftmargin=5mm,
   frame=shadowbox,
   rulesepcolor=\color{lightgray},
   extendedchars=\true,
   inputencoding=utf8,
}

\lstdefinestyle{MyBashStyle} {
    language=bash, % choose the language of the code
    basicstyle=\linespread{0.9}\fontfamily{lmtt}\selectfont\small\color{black},
    keywordstyle={\color{black}}, % style for keywords
    emph={},
    emphstyle={\color{black}},
    stringstyle=\color{black},
    commentstyle={\fontfamily{lmtt}\selectfont\color{black}},
    numbers=left, % where to put the line-numbers
    numberstyle=\tiny, % the size of the fonts that are used for the line-numbers
    backgroundcolor=\color{lemonchiffon},
%    backgroundcolor=\color{lightgray},
    showspaces=false, % show spaces adding particular underscores
    showstringspaces=false, % underline spaces within strings
    showtabs=false, % show tabs within strings adding particular underscores
    frame=single, % adds a frame around the code
    tabsize=2, % sets default tabsize to 2 spaces
    rulesepcolor=\color{gray},
    rulecolor=\color{black},
    captionpos=t, % sets the caption-position to bottom
    breaklines=true, % sets automatic line breaking
    breakatwhitespace=false,
    xleftmargin=20pt,
    xrightmargin=20pt,
    aboveskip=12pt,
    belowskip=12pt,
    escapeinside={(*@}{@*)},
%   frameround=tttt,
   framexleftmargin=5mm,
   frame=shadowbox,
   rulesepcolor=\color{lightgray},
   extendedchars=\true,
   inputencoding=utf8,
}



\renewcommand*\lstlistingname{Kod źródłowy}
\renewcommand{\qedsymbol}{\rule{1ex}{1ex}}


%\lstset{language=Fortran, numbers=left, numberstyle=\tiny, stepnumber=1, numbersep=0pt, tabsize=5,keywordstyle=\color{black}\bfseries,aboveskip=10pt,belowskip=36pt,
%xleftmargin=20pt,xrightmargin=20pt}
%\lstset{emph={Procedura,Oblicz},emphstyle={\color{black}\bfseries},
%label=list1}
%\renewcommand*\lstlistingname{Tekst programu}
%
%\begin{lstlisting}[frame=lines,caption={Pseudokod algorytmu dziel i~zdobywaj z~dyrektywami kompilatora OpenMP}]
%\end{lstlisting}


\renewcommand{\labelitemi}{$\bullet$}
%\renewcommand{\labelitemii}{$\cdot$}
%\renewcommand{\labelitemiii}{$\diamond$}
%\renewcommand{\labelitemiv}{$\ast$}

% Wielkosc wciecia akapitowego
\parindent=0.5cm
% Odstepy miedzy akapitami
\parskip=6pt

% Uklad strony
% \pagestyle{empty}
\fancyhead{}
\pagestyle{fancy}
% Naglowki na stronie parzystej (E) i nieparzystej (O) , Right Left Center
%\fancyhead[CO]{\small \nouppercase{\textbf\rightmark}}
%\fancyhead[CE]{\small \nouppercase{\textbf\leftmark}}

%\lhead{\textbf{\small{\leftmark}}}
%\rhead{\small{Pawel Malczyk}}
%\lfoot{\textbf{\tiny{Konspekt do Podstaw Automatyki i Sterowania II}}}
%\rfoot{}

\fancyhead[CE]{\small \nouppercase{\textbf\leftmark}}
\fancyhead[CO]{\small \nouppercase{\textbf{Metody Programowania Robotów (QNX RTOS)}}}

\fancyfoot[RE,RO]{\footnotesize Paweł Malczyk, Paweł Tomulik}
\fancyfoot[LE,LO]{\footnotesize Rozpowszechnianie bez zgody autorów zabronione}
%\fancyfoot[RE,RO]{\footnotesize Rozpowszechnianie bez zgody autorów zabronione}

\linespread{1.0}
\title{\vspace{4.25cm}\Huge{\textbf{Metody Programowania Robotów}} \\ \vskip10pt\LARGE{\textit{Zestaw instrukcji laboratoryjnych do programowania aplikacji w~systemie operacyjnym czasu rzeczywistego QNX Neutrino}}}
\author{\Large{\textbf{Paweł Malczyk, Paweł Tomulik}}\vspace{0.5cm} \\ Zakład Teorii Maszyn i Robotów \\
Instytut Techniki Lotniczej i Mechaniki Stosowanej \\
Wydział Mechaniczny Energetyki i Lotnictwa \\
Politechnika Warszawska \\
{\href{mailto:pmalczyk@meil.pw.edu.pl}{pmalczyk@meil.pw.edu.pl}}, \href{mailto:ptomulik@meil.pw.edu.pl}{ptomulik@meil.pw.edu.pl} \\ Strona laboratorium: \\ \textbf{\footnotesize\href{http://ztmir.meil.pw.edu.pl/web/Dydaktyka/Prowadzone-przedmioty/Metody-programowania-robotow/Materialy}{http://ztmir.meil.pw.edu.pl/web/Dydaktyka/Prowadzone-przedmioty/Metody-programowania-robotow/Materialy}}}
\date{październik 2015 r.}

\linespread{1.3}

%  \titleformat{<command>}[<shape>]{<format>}{<label>}{<sep>}{<before-code>}[<after-code>]


\titleformat
{\section}
[display]
{\color{black}\normalfont\Large\bfseries\centering}
{\color{black}{\rule{\textwidth}{1pt}\\ \LARGE Laboratorium~\thesection}}
{-1em}
{
    %\rule{\textwidth}{1pt}
    \vspace{1ex}
    \centering
}
[
\vspace{-2.5ex}%
\rule{\textwidth}{0.3pt}
] % after-code

\titleformat{\subsection}
{\color{black}\normalfont\Large\bfseries}
{\color{black}\thesubsection}{1em}{}

\newenvironment{myitemize}
{ \begin{itemize}
    \setlength{\itemsep}{0pt}
    \setlength{\parskip}{0pt}
    \setlength{\parsep}{0pt}     }
{ \end{itemize}                  }

\newenvironment{myenumerate}
{ \begin{enumerate}
    \setlength{\itemsep}{0pt}
    \setlength{\parskip}{0pt}
    \setlength{\parsep}{0pt}     }
{ \end{enumerate}                  }


\begin{document}
\maketitle
\cleardoublepage

\section{Podstawy obsługi systemu operacyjnego QNX RTOS}

\subsection{Powłoka}

Powłoka:

\begin{myitemize}
\item Interpreter poleceń użytkownika
\item Pośredniczy między użytkownikiem a systemem
\item Środowisko pracy użytkownika w systemie
\item Przetwarza pojedyncze polecenia lub skrypty
\end{myitemize}


\begin{example}[Przykład prostej komendy] \label{ex:prostakomenda}

Polecenia można wydawać powłoce z wiersza poleceń (terminala, konsoli) wpisując ich nazwy.

\begin{lstlisting}[style=MyBashStyle]
# ls
\end{lstlisting}

Komenda wyświetla zawartość bieżącego katalogu.
\end{example}

\begin{example}[Przykład komendy z~argumentami]\label{ex:prostakomenda2}
%\addcontentsline{toc}{subsubsection}{Przykład \ref{ex:prostakomenda2}}

Komenda wyświetla zawartość bieżącego katalogu wraz ze szczegółowymi informacjami nt. obiektów. Argumenty (\lstinline[style=MyBashStyle]{-l}) zmieniają zachowanie komendy prostej.

\begin{lstlisting}[style=MyBashStyle]
# ls -l
\end{lstlisting}


Formalna składnia komendy z argumentami:

\begin{lstlisting}[style=MyBashStyle]
# command argument1 argument2 argument3 ... argumentN
\end{lstlisting}
\end{example}

\begin{example}[Przykład złożonej komendy]\label{ex:prostakomenda3}

Komendy proste i~komendy proste z~argumentami możemy łączyć. Separatorem poleceń jest średnik (\lstinline[style=MyBashStyle]{;}).


\begin{lstlisting}[style=MyBashStyle]
# date ; ls
\end{lstlisting}

Formalna składnia złożonej komendy:

\begin{lstlisting}[style=MyBashStyle]
# command1; command2; command3; ... argumentN
\end{lstlisting}
\end{example}

%\begin{example}[Logowanie do systemu z wiersza poleceń]\label{ex:prostakomenda4}
%
%
%Aby zalogować się do systemu z wiersza poleceń używamy polecenia \lstinline[style=MyBashStyle]{login}.
%
%\begin{lstlisting}[style=MyBashStyle]
%# login
%# login: root
%\end{lstlisting}
%
%Polecenie uzyskuje od użytkownika jego nazwę oraz hasło, które następnie weryfikuje z danymi zawartymi w pliku \lstinline[style=MyBashStyle]{/etc/passwd}, który możemy podejrzeć poleceniem \lstinline[style=MyBashStyle]{cat}.
%\end{example}
%
%
%\begin{example}[Plik definiujący użytkowników systemu]\label{ex:prostakomenda5}
%
%
%Po zalogowaniu przez polecenie \lstinline[style=MyBashStyle]{login} wywoływana jest powłoka (\lstinline[style=MyBashStyle]{/bin/sh}). Następnie w fazie inicjalizacji, ustawiane są parametry pracy powłoki. Na ogół jest to proces dwuetapowy, w~którym interpretowane są pliki \lstinline[style=MyBashStyle]{/etc/passwd} oraz \lstinline[style=MyBashStyle]{.profile} z domowego katalogu. Obejrzeć zawrtość pliku \lstinline[style=MyBashStyle]{/etc/passwd}.
%
%\begin{lstlisting}[style=MyBashStyle]
%# cat /etc/passwd
%\end{lstlisting}
%\end{example}

\begin{example}[Wejście i wyjście z powłoki]\label{ex:prostakomenda6}


Powłoka wykonuje polecenia użytkownika. Kiedy powłoka wyświetla znak zachęty, który w domyślnym shell-u (Bourne shell) w systemie QNX jest znak \lstinline[style=MyBashStyle]{#} dla użytkownika \lstinline[style=MyBashStyle]{root}, to czeka na polecenia użytkownika. Możemy uruchomić powłokę w takim trybie. Sytuację tę ilustruje poniższy przykład.


\begin{lstlisting}[style=MyBashStyle]
# /bin/sh
#
# exit
\end{lstlisting}

Aby wyjść z powłoki należy użyć polecenia \lstinline[style=MyBashStyle]{# exit}.
\end{example}

\begin{example}[Obsługa edytora tekstu vi]\label{ex:vi}


Edytor \lstinline[style=MyBashStyle]{vi} jest zaawansowanym edytorem tekstowym często występującym w~systemach Unix. Aby uruchomić edytor należy wpisać komendę:

\begin{lstlisting}[style=MyBashStyle]
# vi
\end{lstlisting}

 Edytor \lstinline[style=MyBashStyle]{vi} jest edytorem modalnym. Oznacza to, że może znajdować się w~dwóch stanach: \textbf{trybie edycji} lub \textbf{trybie poleceń}.

\begin{myitemize}
\item Przejście do trybu edycji poprzez wydanie polecenia \lstinline[style=MyBashStyle]{i} (insert) lub \lstinline[style=MyBashStyle]{a} (append).
\item Przejcie z~trybu edycji do trybu poleceń odbywa się poprzez naciśnięcie klawicza \lstinline[style=MyBashStyle]{Esc}.
\end{myitemize}

Polecenia edytora \lstinline[style=MyBashStyle]{vi} składają się z~kilku grup. Przedstawiono je zbiorczo w~postaci krótkiej instrukcji na stronie~\pageref{viRef}.

W~trybie edycji, tj. po wciśnięciu klawisza \lstinline[style=MyBashStyle]{i} wpiszmy do pliku następujące linie tekstu postaci, np.:


\begin{lstlisting}[style=MyBashStyle]
Jak dobrze wstac
Skoro swit
Jutrzenki blask
Duszkiem pic
Tytul:
Radosc o poranku
Nim w gorze tam
Skowronek zacznie tryl
Jak dobrze wczesnie wstac
Dla tych chwil
\end{lstlisting}

W~następnej kolejności przejść do trybu poleceń, poprzez wciśnięcie klawisza \lstinline[style=MyBashStyle]{Esc} oraz zapisać plik wydając polecenie:

\begin{lstlisting}[style=MyBashStyle]
:w plikPoranny
\end{lstlisting}

Kolejno, zamknąć zapisany plik komendą:

\begin{lstlisting}[style=MyBashStyle]
:q
\end{lstlisting}

Otworzyć ponownie plik:

\begin{lstlisting}[style=MyBashStyle]
vi plikPoranny
\end{lstlisting}

Wykonać następujące eksperymenty:

\begin{myitemize}
\item Wykasować tekst \lstinline[style=MyBashStyle]{Tytul:} litera po literze poleceniem \lstinline[style=MyBashStyle]{x}.
\item Klawiszami \lstinline[style=MyBashStyle]{h}, \lstinline[style=MyBashStyle]{j}, \lstinline[style=MyBashStyle]{k}, \lstinline[style=MyBashStyle]{l} przejść do linii z tekstem \lstinline[style=MyBashStyle]{Radosc o poranku}.
\item Wykasować bieżącą linię za pomocą komendy \lstinline[style=MyBashStyle]{dd}. Wykasować także pustą linię, powstałą po tekście \lstinline[style=MyBashStyle]{Tytul:}.
\item Przejść do pierwszej linii pliku (\lstinline[style=MyBashStyle]{h}, \lstinline[style=MyBashStyle]{j}, \lstinline[style=MyBashStyle]{k}, \lstinline[style=MyBashStyle]{l}) i~wkleić usunięty tekst kombinacją klawiszy \lstinline[style=MyBashStyle]{Shift + P}, tak, aby cały tekst wyglądał następująco:
\end{myitemize}

\begin{lstlisting}[style=MyBashStyle]
Radosc o poranku
Jak dobrze wstac
Skoro swit
Jutrzenki blask
Duszkiem pic
Nim w gorze tam
Skowronek zacznie tryl
Jak dobrze wczesnie wstac
Dla tych chwil
\end{lstlisting}

Przeprowadzić kolejne eksperymenty:

\begin{myitemize}
\item W~trybie poleceń, znaleźć wszystkie wystąpienia słowa \lstinline[style=MyBashStyle]{dobrze} wpisując:
\end{myitemize}

\begin{lstlisting}[style=MyBashStyle]
/dobrze
\end{lstlisting}
\begin{myitemize}
\item Przeszukać tekst w~przód poprzez naciśnięcie klawisza \lstinline[style=MyBashStyle]{N}. Przeszukiwanie w~tył nastąpi poprzez wciśnięcie klawisza \lstinline[style=MyBashStyle]{Shift + N}.
\item Przejść do pierwszej linii poleceniem \lstinline[style=MyBashStyle]{:1} oraz zamienić tekst \lstinline[style=MyBashStyle]
{Radosc} na \lstinline[style=MyBashStyle]{Tytul: Radosc} poprzez sekwencję:
\end{myitemize}

\begin{lstlisting}[style=MyBashStyle]
:s/Radosc/Tytul: Radosc
\end{lstlisting}

\begin{myitemize}
\item Zamienić wszystkie wystąpienia słowa \lstinline[style=MyBashStyle]{dobrze} na \lstinline[style=MyBashStyle]{DOBRZE} w zakresie od bieżącej linii \lstinline[style=MyBashStyle]{.} do ostatniej linii \lstinline[style=MyBashStyle]{$}:
\end{myitemize}

\begin{lstlisting}[style=MyBashStyle]
:.,$s/dobrze/DOBRZE
\end{lstlisting}

Zapisać i~zamknąć plik poleceniem:

\begin{lstlisting}[style=MyBashStyle]{h}
:wq!
\end{lstlisting}
\end{example}

\clearpage
\phantomsection
{\label{viRef}
\includepdf[scale=0.95,pages={1},pagecommand={\thispagestyle{fancy}},pagecommand={\thispagestyle{fancy}{\label{subsec:vi}}},lastpage=1,angle=90]{img/viRef.pdf}
}



\begin{example}[Utworzenie i~uruchomienie skryptu] \label{ex:prostakomenda7}


Powłoka, jako interpreter, wykonuje pewien program.  Kolejne komendy programu mogą być wpisywane na bieżąco w~terminalu lub cały program może być dostarczony do powłoki w~postaci skryptu. Należy utworzyć plik o~nazwie \lstinline[style=MyBashStyle]{skrypt} edytorem tekstu \lstinline[style=MyBashStyle]{vi} o~treści:

\begin{lstlisting}[style=MyBashStyle]
date ; ls
\end{lstlisting}

W~następnej kolejności uruchomić skrypt powłoki wydając polecenie:

\begin{lstlisting}[style=MyBashStyle]
# /bin/sh skrypt
\end{lstlisting}

Przykład ilustruje skrypt powłoki. Na ogół skrypty składają się z plików, w których są zapisane komendy, interpretowane przez powłokę. Skrypt można uruchomić wpisując w wiersz poleceń jego nazwę. Jednak bezpośrednie wpisanie jego nazwy kończy się niepowodzeniem:

\begin{lstlisting}[style=MyBashStyle]
# ./skrypt
sh: ./skrypt: cannot execute - Permission denied
\end{lstlisting}

W tej sytuacji należy zapewnić, aby skrypt miał odpowiednie atrybuty oraz upewnić się, że uruchamiany jest właściwy interpreter poleceń. Aby zmienić atrybuty pliku należy użyć następującego polecenia:

\begin{lstlisting}[style=MyBashStyle]
# chmod a+x ./skrypt
\end{lstlisting}

Uruchomić \lstinline[style=MyBashStyle]{skrypt} w~linii poleceń.
\end{example}


\begin{example}[Prosty skrypt powłoki]\label{ex:prostakomenda8}


Należy także uzupełnić plik \lstinline[style=MyBashStyle]{skrypt}, tak, żeby miał postać:

\begin{lstlisting}[style=MyBashStyle]
#!/bin/sh
# wypisz date i wyswietl zawartosc katalogu
date ; ls
\end{lstlisting}

Znak \lstinline[style=MyBashStyle]{#} stanowi znak komentarza w~skrypcie. Wiersze zaczynające się od tego znaku są ignorowane przez interpreter poleceń i~traktowane jako komentarz, oprócz pierwszej linii z~komendą \lstinline[style=MyBashStyle]{#!/bin/sh}. Pierwsza linia kodu przekazuje informacje o~rodzaju powłoki, która powinna wykonać skrypt.
\end{example}


\begin{example}[Prosty skrypt powłoki]\label{ex:prostakomenda9}

Dokumentacja systemu jest dostępna w formie elektronicznej na stronach \href{www.qnx.com}{www.qnx.com}. Skrócony opis interesującego nas polecenia systemowego uzyskujemy poprzez wpisanie w okno terminala polecenia \lstinline[style=MyBashStyle]{use polecenie}, np.:

\begin{lstlisting}[style=MyBashStyle]
# use date
\end{lstlisting}
\end{example}

\subsection{System plików}

W systemie QNX Neutrino prawie wszystkie zasoby są plikami. Dane, urządzenia, bloki pamięci, a nawet pewne usługi są reprezentowane przez abstrakcję plików. Mechanizm plików pozwala na jednolity dostęp do zasobów zarówno lokalnych, jak i~zdalnych, za pomocą poleceń i programów usługowych wydawanych z okienka poleceń. Typowe drzewo plików w~systemie QNX przedstawiono na rysunku~\ref{fig:drzewo}.

\begin{figure}[!h]
\centering
\includegraphics[width=0.75\textwidth]{img/systemplikow}
\caption{Drzewo plików w systemie QNX}
\label{fig:drzewo}
\end{figure}

\begin{myitemize}
\item Katalog główny \lstinline[style=MyBashStyle]{/} jest miejscem montowania twardego dysku lub pamięci flash.
\item \lstinline[style=MyBashStyle]{/bin} - zawiera podstawowe komendy systemowe (np. \lstinline[style=MyBashStyle]{ls}, \lstinline[style=MyBashStyle]{chmod}).
\item \lstinline[style=MyBashStyle]{/boot} - zawiera pliki i katalogi związane z obrazami systemu operacyjnego.
\item \lstinline[style=MyBashStyle]{/dev} - katalog przynależy do menadżera zasobów, w którym przechowywane są informacje o urządzeniach dostępnych w systemie.
\item \lstinline[style=MyBashStyle]{/etc} - zawiera pliki i programy używane do administracji i konfiguracji systemu.
\item \lstinline[style=MyBashStyle]{/fs} - w katalogu są montowane dodatkowe systemy plików.
\item \lstinline[style=MyBashStyle]{/home} - katalogi domowe użytkowników.
\item \lstinline[style=MyBashStyle]{/lib} - zawiera współdzielone biblioteki używane przez inne programy, procesy.
\item \lstinline[style=MyBashStyle]{/proc} - katalog, w którym są zawarte informacje o procesach i przestrzeni nazw.
\item \lstinline[style=MyBashStyle]{/root} - katalog domowy dla użytkownika root.
\item \lstinline[style=MyBashStyle]{/sbin} - zawiera niezbędne pliki wykonywalne (np. sterowniki, programy inicjujące, konfiguracyjne, menadżery).
\item \lstinline[style=MyBashStyle]{/tmp} - zawiera pliki tymczasowe.
\item \lstinline[style=MyBashStyle]{/usr} - zawiera współdzielone dane, tylko do odczytu.
\item \lstinline[style=MyBashStyle]{/var} - zawiera zmienne dane (np. cache, logi).
\end{myitemize}

Bardziej obszerny opis struktury i~hierarchii katalogów w~systemach klasy Unix, do których należy QNX, można znaleźć pod adresem \href{https://en.wikipedia.org/wiki/Unix\_filesystem}{https://en.wikipedia.org/wiki/Unix\_filesystem}. W~systemie QNX występują różne typy plików. Ich zestawienie przedstawiono w~tabeli \ref{tab:typyplikow}.


\begin{table}[h!]
\centering
\caption{Typy plików w~systemie QNX}
\setlength{\arrayrulewidth}{1pt}
\setlength{\tabcolsep}{6pt}
\renewcommand{\arraystretch}{1.2}
\begin{tabular}{ |p{0.15\textwidth}|p{0.3\textwidth}|p{0.4\textwidth}| }
\hline \rowcolor{gray}
\textbf{Oznaczenie} & \textbf{Typ pliku} & \textbf{Opis} \\ \hline
\mbox{\lstinline[style=MyBashStyle]{d}} & Katalog (ang. directory) & Plik zawierający inne pliki i katalogi. \\ \hline
\mbox{\lstinline[style=MyBashStyle]{l}} & Dowiązanie symboliczne (ang. symbolic link) & Dodatkowa nazwa pliku, który jest umieszczony w innym miejscu. \\ \hline
\mbox{\lstinline[style=MyBashStyle]{n}} & Plik specjalny & Np. blok pamięci współdzielonej. \\ \hline
\mbox{\lstinline[style=MyBashStyle]{c}} & Specjalny plik znakowy (ang. charakter device) & Urządzenie z dostępem znakowym (konsola, porty szeregowe, równoległe). \\ \hline
\mbox{\lstinline[style=MyBashStyle]{p}} & Plik specjalny FIFO & Bufor cykliczny w pamięci operacyjnej. \\ \hline
\mbox{\lstinline[style=MyBashStyle]{b}} & Specjalny plik blokowy (ang. block device) & Urządzenie z dostępem blokowym (dysk, partycja dyskowa). \\ \hline
\mbox{\lstinline[style=MyBashStyle]{s}} & Gniazdko (ang. socket)	 & Plik komunikacji sieciowej. \\ \hline
\end{tabular}
\label{tab:typyplikow}
\end{table}


Pliki są zorganizowane w katalogi. Katalogi mają strukturę drzewa z korzeniem \lstinline[style=MyBashStyle]{/} (\lstinline[style=MyBashStyle]{root}). Aby wskazać na konkretny obiekt w systemie plików, należy podać jego ścieżkę. Rozróżniamy ścieżki absolutne i~względne. Ścieżka absolutna zaczyna się od znaku \lstinline[style=MyBashStyle]{/} , np. \lstinline[style=MyBashStyle]{/etc/passwd}. Ścieżka relatywna zaczyna się od znaku innego niż \lstinline[style=MyBashStyle]{/} , np. \lstinline[style=MyBashStyle]{etc/passwd} .


\begin{example}[Wylistowanie zawartości katalogu]\label{ex:wylistowanie}

Pokazać zawartość bieżącego katalogu:

\begin{lstlisting}[style=MyBashStyle]
# ls -l
total 419571
drwxr-xr-x   2 root      root           3072 Feb 23  2014 bin
drwxr-xr-x   4 root      root           1024 Feb 23  2014 boot
dr-xr-xr-x   2 root      root              0 Oct 22 15:07 dev
drwxr-xr-x  10 root      root           3072 Feb 23  2014 etc
drwxr-xr-x   2 root      root           1024 Feb 23  2014 home
drwxr-xr-x   4 root      root           5120 Feb 23  2014 lib
drwxr-xr-x   3 root      root           1024 Feb 23  2014 libexec
dr-xr-xr-x   2 root      root      214798336 Oct 22 15:07 proc
drwxr-xr-x   2 root      root           1024 Sep 30 21:43 root
drwxr-xr-x   2 root      root           3072 Feb 23  2014 sbin
-rw-rw-rw-   1 root      root             11 Oct 22 12:14 skrypt
drwxr-xr-x   2 root      root           1024 Oct 22 12:30 tmp
drwxr-xr-x   7 root      root           1024 Feb 23  2014 usr
drwxr-xr-x   4 root      root           1024 Feb 23  2014 var
\end{lstlisting}

Pokazać zawartość innego katalogu:

\begin{lstlisting}[style=MyBashStyle]
# ls -la /usr/
\end{lstlisting}

Argument \lstinline[style=MyBashStyle]{-l} służy do listowania w formacie tzw. długim, natomiast argument \lstinline[style=MyBashStyle]{-a} do listowania ukrytych plików. Sprawdzić inne parametry polecenia ls następująco:

\begin{lstlisting}[style=MyBashStyle]
# use ls
\end{lstlisting}
\end{example}

\begin{example}[Obejrzenie zawartości pliku]\label{ex:obejrzenie}

Oprócz listowania zawartości katalogów, istotna jest możliwość oglądania zawartości plików (np. skryptowych):

\begin{lstlisting}[style=MyBashStyle]
# cat /skrypt
# cat -n /skrypt			# -n numerowanie linii
# cat -n /etc/passwd
\end{lstlisting}
\end{example}


\begin{example}[Wyświetlanie liczby wierszy, słów i~bajtów zawartych w pliku]\label{ex:wys}

Aby uzyskać informację o całkowitej liczbie linii, słów i~znaków, zawartych w pliku można użyć polecenie \lstinline[style=MyBashStyle]{wc} (ang. word count):

\begin{lstlisting}[style=MyBashStyle]
# wc /skrypt
\end{lstlisting}

Dostępne argumenty polecenia przedstawiono w~tabeli \ref{tab:opcjeWC}.


\begin{table}[h!]
\centering
\caption{Opcje polecenia wc}
\setlength{\arrayrulewidth}{1pt}
\setlength{\tabcolsep}{6pt}
\renewcommand{\arraystretch}{1.2}
\begin{tabular}{ |p{0.15\textwidth}|p{0.4\textwidth}|}
\hline \rowcolor{gray}
\textbf{Argumenty} & \textbf{Opis} \\ \hline
\mbox{\lstinline[style=MyBashStyle]{-l}} & Oblicza liczbę linii \\ \hline
\mbox{\lstinline[style=MyBashStyle]{-w}} & Oblicza liczbę słów \\ \hline
\mbox{\lstinline[style=MyBashStyle]{-m}} & Oblicza liczbę znaków \\ \hline
\mbox{\lstinline[style=MyBashStyle]{-c}} & Oblicza liczbę znaków  \\ \hline
\end{tabular}
\label{tab:opcjeWC}
\end{table}

\begin{example}[Poruszanie się po systemie plików]

System operacyjny jest wyposażony w zestaw instrukcji, które umożliwiają poruszanie się po drzewie plików. Najważniejsze zestawiono w tabeli~\ref{tab:poruszanie} i~zilustrowano w przykładzie.

\begin{table}[h!]
\centering
\caption{Poruszanie się po systemie plików}
\setlength{\arrayrulewidth}{1pt}
\setlength{\tabcolsep}{6pt}
\renewcommand{\arraystretch}{1.2}
\begin{tabular}{ |p{0.15\textwidth}|p{0.4\textwidth}|}
\hline \rowcolor{gray}
\textbf{Polecenie} & \textbf{Opis} \\ \hline
\mbox{\lstinline[style=MyBashStyle]{pwd}} & Wyświetlenie nazwy katalogu bieżącego (ang. print working directory) \\ \hline
\mbox{\lstinline[style=MyBashStyle]{cd}}  & Zmiana katalogu bieżącego (ang. change directory) \\ \hline
\mbox{\lstinline[style=MyBashStyle]{cd..}} & Przejście do katalogu nadrzędnego \\ \hline
\mbox{\lstinline[style=MyBashStyle]{cd /}} & Przejście do katalogu głównego \\ \hline
%\mbox{\lstinline[style=MyBashStyle]{cd ~}} &	Przejście do katalogu domowego \\ \hline
\end{tabular}
\label{tab:poruszanie}
\end{table}
\end{example}



Wprowadzić następujące komendy w~wierszu poleceń.

\begin{lstlisting}[style=MyBashStyle]
# pwd			# Nazwa katalogu biezacego (dla naszego przypadku)
/
# cd /usr			# Przejscie do katalogu uzytkownika
# pwd
/usr
# cd ..			# Przejscie do katalogu nadrzednego
# pwd
/
# ls .			# Wylistowanie nazw plikow w biezacym katalogu
...
# ls -l 			# Wylistowanie ze szczegolami
\end{lstlisting}
\end{example}

\begin{example}[Utworzenie i kasowanie katalogów]

Pliki w systemie plików można tworzyć i usuwać. Katalogi można również modyfikować. Najczęściej używane polecenia służą do tworzenia, kopiowania, przenoszenia oraz usuwania katalogów. Zestawienie podstawowych poleceń podano w tabeli~\ref{tab:tworziusun}.

\begin{table}[h!]
\centering
\caption{Tworzenie i usuwanie katalogów i~plików}
\setlength{\arrayrulewidth}{1pt}
\setlength{\tabcolsep}{6pt}
\renewcommand{\arraystretch}{1.2}
\begin{tabular}{ |p{0.15\textwidth}|p{0.4\textwidth}|}
\hline \rowcolor{gray}
\textbf{Polecenie} & \textbf{Opis} \\ \hline
\mbox{\lstinline[style=MyBashStyle]{touch plik}} & Utworzenie pustego pliku lub zmiana daty modyfikacji istniejącego \\ \hline
\mbox{\lstinline[style=MyBashStyle]{cd}}  & Zmiana katalogu bieżącego (ang. change directory) \\ \hline
\mbox{\lstinline[style=MyBashStyle]{rm [-Rfi] plik}} & Usunięcie pliku: \mbox{\lstinline[style=MyBashStyle]{-i}} - żądanie potwierdzenie; \mbox{\lstinline[style=MyBashStyle]{-f}} - bezwarunkowe kasowanie pliku; \mbox{\lstinline[style=MyBashStyle]{-R}} - kasowanie zawartości katalogu z~podkatalogami  \\ \hline
\mbox{\lstinline[style=MyBashStyle]{mkdir katalog}} & Utworzenie katalogu o nazwie \mbox{\lstinline[style=MyBashStyle]{katalog}} \\ \hline
\mbox{\lstinline[style=MyBashStyle]{rmdir katalog}} &	Usunięcie katalogu o nazwie \mbox{\lstinline[style=MyBashStyle]{katalog}}  \\ \hline
\end{tabular}
\label{tab:tworziusun}
\end{table}

Sprawdzić działanie następujących komend:

\begin{lstlisting}[style=MyBashStyle]
# pwd			# Nazwa katalogu biezacego
/
# cd /home			# Przejdz do katalogu home
# mkdir katalog			# Utworzenie katalogu
# cd katalog			# Przejdz do katalogu
# mkdir podkatalog			# Utworzenie podkatalogu
# touch plik			# Utworzenie pustego pliku
# touch plik2			# Utworzenie pustego pliku
# rm plik2			# Usuniecie pustego pliku
# rmdir podkatalog			# Usuniecie podkatalogu
# cd .. 			# Wyjscie do katalogu nadrzednego
# rmdir katalog 			# Proba usuniecia podkatalogu
katalog/: Directory not empty
# rm -Ri katalog			# Usuniecie rekursywne z potwierdzeniem
rm: remove katalog/plik? (y/N) y
rm: remove directory katalog? (y/N) y
\end{lstlisting}

\end{example}


\begin{example}[Przenoszenie i~kopiowanie katalogów i~plików]

Pliki i~katalogi można kopiować i~przenosić. Zestawienie najważniejszych komend przedstawiono w~tabeli~\ref{tab:przenos}.

\begin{table}[h!]
\centering
\caption{Przenoszenie i~kopiowanie katalogów i~plików}
\setlength{\arrayrulewidth}{1pt}
\setlength{\tabcolsep}{6pt}
\renewcommand{\arraystretch}{1.2}
\begin{tabular}{ |p{0.23\textwidth}|p{0.4\textwidth}|}
\hline \rowcolor{gray}
\textbf{Polecenie} & \textbf{Opis} \\ \hline
\mbox{\lstinline[deletekeywords={if}]{mv [-if] zrodlo cel}} & Przenoszenie lub zmiana nazwy plików: \mbox{\lstinline[style=MyBashStyle]{-i}} - żądanie potwierdzenia, gdy plik docelowy może być nadpisany; \mbox{\lstinline[style=MyBashStyle]{-f}} - bezwarunkowe skopiowanie pliku \\ \hline
\mbox{\lstinline[style=MyBashStyle]{cp [-ifR] zrodlo cel}}  & Kopiowanie plików: \mbox{\lstinline[style=MyBashStyle]{-i}} - żądanie potwierdzenia, gdy plik docelowy może być nadpisany; \mbox{\lstinline[style=MyBashStyle]{-f}} - bezwarunkowe skopiowanie pliku; \mbox{\lstinline[style=MyBashStyle]{-R}} - kopiowanie zawartości katalogu z podkatalogami \\ \hline
\end{tabular}
\label{tab:przenos}
\end{table}

Sprawdzić działanie następujących komend:

\begin{lstlisting}[style=MyBashStyle]
# pwd
/home
# mkdir katalog
# cd katalog
# touch pliczek
# cd ..
# mv ./katalog/pliczek .			# Przeniesienie do katalogu biezacego
# mv pliczek plik.dat			# Zmiana nazwy pliku
# cp plik.dat katalog			# Skopiowanie pliku do katalogu
# cp -Ri katalog katalog2			# Skopiowanie katalogu do katalogu2 razem z zawartoscia
\end{lstlisting}
\end{example}


\begin{example}[Zmiana atrybutów pliku]

Pliki w systemie mają określonych właścicieli, grupy użytkowników, a także zestawy atrybutów do nich przypisanych. System umożliwia dostęp do plików w trybie odczytu, zapisu lub wykonania. Symboliczne oznaczenia praw dostępu do pliku są następujące:

\begin{myitemize}
\item \lstinline[style=MyBashStyle]{r} - prawo odczytu (ang. read)
\item \lstinline[style=MyBashStyle]{w} - prawo zapisu (ang. write)
\item \lstinline[style=MyBashStyle]{x} - prawo wykonania (ang. execute)
\end{myitemize}

Prawa te mogą być zdefiniowane dla właściciela pliku, grupy, do której on należy i wszystkich innych użytkowników.

\begin{myitemize}
\item \lstinline[style=MyBashStyle]{u} - właściciel pliku (ang. user)
\item \lstinline[style=MyBashStyle]{g} - grupa (ang. group)
\item \lstinline[style=MyBashStyle]{o} - inni użytkownicy (ang. other)
\end{myitemize}

Aby obejrzeć właściciela pliku oraz atrybuty wykonajmy następujące polecenie:

\begin{lstlisting}[style=MyBashStyle]
# ls -l /home/plik.dat
-rw-rw-rw- 1 root root 0 Oct 22 11:41 /home/plik.dat
\end{lstlisting}

W~terminalu wyświetlone zostały w~kolejności atrybuty dla właściciela (\lstinline[style=MyBashStyle]{rw-}), grupy (\lstinline[style=MyBashStyle]{rw-}) oraz innych użytkowników (\lstinline[style=MyBashStyle]{rw-}); wskazano także liczbę dowiązań \lstinline[style=MyBashStyle]{1}, nazwę właściciela pliku \lstinline[style=MyBashStyle]{root}, nazwę grupy \lstinline[style=MyBashStyle]{root}, rozmiar pliku \lstinline[style=MyBashStyle]{0}, datę utworzenia \lstinline[style=MyBashStyle]{Oct 22 11:41} oraz nazwę pliku \lstinline[style=MyBashStyle]{home/plik.dat}.


Atrybuty plików oraz ich właścicieli można zmieniać - zobacz tabela \ref{tab:zmien}.

\begin{table}[h!]
\centering
\caption{Tworzenie i usuwanie katalogów i~plików}
\setlength{\arrayrulewidth}{1pt}
\setlength{\tabcolsep}{6pt}
\renewcommand{\arraystretch}{1.2}
\begin{tabular}{ |p{0.15\textwidth}|p{0.4\textwidth}|}
\hline \rowcolor{gray}
\textbf{Polecenie} & \textbf{Opis} \\ \hline
\mbox{\lstinline[style=MyBashStyle]{chmod}} & Zmiana atrybutów dla pliku, bądź katalogu \\ \hline
\mbox{\lstinline[style=MyBashStyle]{chown}} & Zmiana właściciela (lub opcjonalnie grupy) dla pliku, bądź katalogu \\ \hline
\mbox{\lstinline[style=MyBashStyle]{chgrp}}  & Zmiana grupy dla pliku, bądź katalogu \\ \hline
\end{tabular}
\label{tab:zmien}
\end{table}

Zmiana atrybutów pliku odbywa się wg następujące składni:

\begin{lstlisting}[style=MyBashStyle]
# chmod wlasciciel akcja atrybuty
\end{lstlisting}

gdzie \lstinline[style=MyBashStyle]{wlasciciel} jest jednym ze skrótów literowych (\lstinline[style=MyBashStyle]{u}, \lstinline[style=MyBashStyle]{g}, \lstinline[style=MyBashStyle]{o}, bądź \lstinline[style=MyBashStyle]{a} - dla wszystkich użytkowników), atrybuty dotyczą oznaczeń (\lstinline[style=MyBashStyle]{r}, \lstinline[style=MyBashStyle]{w} lub \lstinline[style=MyBashStyle]{x}). Możliwe do wykonania akcje opisano w~tabeli~\ref{tab:zarzadzanie}.

\begin{table}[h!]
\centering
\caption{Zarządzanie prawami dostępu}
\setlength{\arrayrulewidth}{1pt}
\setlength{\tabcolsep}{6pt}
\renewcommand{\arraystretch}{1.2}
\begin{tabular}{ |p{0.15\textwidth}|p{0.4\textwidth}|}
\hline \rowcolor{gray}
\textbf{Polecenie} & \textbf{Opis} \\ \hline
\mbox{\lstinline[style=MyBashStyle]{+}} & Dodanie praw dostępu \\ \hline
\mbox{\lstinline[style=MyBashStyle]{-}} & Usunięcie praw dostępu \\ \hline
\mbox{\lstinline[style=MyBashStyle]{=}}  & Jawne ustawienie praw dostępu \\ \hline
\end{tabular}
\label{tab:zarzadzanie}
\end{table}

Wykonać następującą serię poleceń:

\begin{lstlisting}[style=MyBashStyle]
# pwd
/home
# ls -l
total 4
drwxrwxrwx   2 root      root           1024 Oct 22 11:43 katalog
drwxrwxrwx   2 root      root           1024 Oct 22 11:43 katalog2
-rw-rw-rw-   1 root      root              0 Oct 22 11:41 plik.dat
# chmod a+rwx plik.dat			# Dodanie praw dostepu dla wszystkich
# ls -l plik.dat
-rwxrwxrwx   1 root      root              0 Oct 22 11:41 plik.dat
# chmod go-wx plik.dat			# Odebranie praw dostepu
# ls -l plik.dat
-rwxr--rw--   1 root      root              0 Oct 22 11:41 plik.dat
\end{lstlisting}



\end{example}


\subsection{Obsługa procesów}

W systemie QNX każdy program jest uruchamiany jako proces. System zarządza procesami układając je w hierarchię rodzic-potomek. Proces, który uruchamia inny proces, nazywa się macierzystym, a proces uruchomiony - potomnym. Po wydaniu polecenia w konsoli, np. \lstinline[style=MyBashStyle]{ls}, proces powłoki powołuje do życia nowy proces \lstinline[style=MyBashStyle]{ls}. Powłoka jest więc w tej sytuacji procesem macierzystym, a proces \lstinline[style=MyBashStyle]{ls} jest procesem potomnym.


\begin{example}[Proces uruchomiony w~tle]

Procesy mogą być uruchamiane w pierwszym planie (ang. foreground) i w tle (ang. background). Domyślnie, procesy są uruchamiane w pierwszym planie. Strumień wejścia do programu stanowi klawiatura, natomiast wyniki są wyprowadzane na ekran, np.

\begin{lstlisting}[style=MyBashStyle]
# ls
katalog katalog2 plik.dat
\end{lstlisting}

Procesy tła możemy uruchamiać poprzez dodanie znaku (\lstinline[style=MyBashStyle]{&}):

\begin{lstlisting}[style=MyBashStyle]
# ls &
[1] 1011730
# katalog katalog2 plik.dat

[1] + Done	ls
\end{lstlisting}

Po uruchomieniu programu, na ekranie pojawia się nr zadania \lstinline[style=MyBashStyle]{[1]} oraz numer identyfikujący proces PID (ang. process identification number). Po wciśnięciu klawisza Enter, program kończy zadanie, a~powłoka wyświetla znak zachęty.
Proces uruchomiony w pierwszym planie można przesunąć do tła (bez podłączenia do klawiatury) i na odwrót. Podstawowe komendy, służące kontrolowaniu zadań zestawiono w~tabeli \ref{tab:kontrola2}.

\begin{table}[h!]
\centering
\caption{Kontrola zadań}
\setlength{\arrayrulewidth}{1pt}
\setlength{\tabcolsep}{6pt}
\renewcommand{\arraystretch}{1.2}
\begin{tabular}{ |p{0.15\textwidth}|p{0.4\textwidth}|}
\hline \rowcolor{gray}
\textbf{Polecenie} & \textbf{Opis} \\ \hline
\mbox{\lstinline[style=MyBashStyle]{polecenie &}} & Uruchomienie zadania w tle \\ \hline
\mbox{\lstinline[style=MyBashStyle]{jobs [-l]}} & Listowanie zadań pracujących w tle \\ \hline
\mbox{\lstinline[style=MyBashStyle]{Ctrl+Z}}  & Wstrzymanie bieżącego zadania \\ \hline
\mbox{\lstinline[style=MyBashStyle]{Ctrl+C}}  & Zakończenie bieżącego zadania \\ \hline
\mbox{\lstinline[style=MyBashStyle]{fg [PID]}}  & Przeniesienie procesu działającego w tle na pierwszy plan na podstawie numeru procesu\\ \hline
\mbox{\lstinline[style=MyBashStyle]{fg [jobID]}}  & Przeniesienie procesu działającego w tle na pierwszy plan na podstawie numeru zadania\\ \hline
\mbox{\lstinline[style=MyBashStyle]{bg [PID]}}  & Uruchomienie w tle wstrzymanego zadania na podstawie numeru procesu \\ \hline
\mbox{\lstinline[style=MyBashStyle]{bg [jobID]}}  & Uruchomienie w tle wstrzymanego zadania na podstawie numeru zadania \\ \hline
\end{tabular}
\label{tab:kontrola2}
\end{table}
\end{example}


\begin{example}[Procesy tła i~procesy pierwszoplanowe]

Proces \lstinline[style=MyBashStyle]{top} wyświetla statystyki wydajności systemu operacyjnego. Uruchomić proces \lstinline[style=MyBashStyle]{top} w~pierwszym planie, a~następnie nacisnąć kombinację \mbox{\lstinline[style=MyBashStyle]{Ctrl+Z}}, aby wstrzymać bieżący proces.

\begin{lstlisting}[style=MyBashStyle]
# top
22 processes; 64 threads;
CPU states: 99.9% idle, 0.0% user, 0.0% kernel
Memory: 0 total, 204M avail, page size 4K

      PID   TID PRI STATE    HH:MM:SS    CPU  COMMAND
  1040402     1  10 Rply      0:00:00   0.01% top
     4110     2  21 Rcv       0:00:01   0.00% io-pkt-v4-hc
     4106     1  10 Rcv       0:00:00   0.00% devc-con-hid
     4101     1  10 Rcv       0:00:00   0.00% pci-bios
     4102     2  21 Rcv       0:00:00   0.00% devb-eide
        1     9  10 Rcv       0:00:00   0.00% kernel
...

             Min        Max       Average
CPU idle:     99%        99%        99%
Mem Avail:   204MB      204MB      204MB
Processes:    22         22         22
Threads:      64         64         64

[1] + Stopped top 			# Po wcisnieciu Ctrl+Z
# jobs -l
[1] + 1040402 Stopped top
# fg %1			# Przeniesienie procesy na pierwszy plan; alternatywnie fg 1040402
22 processes; 64 threads;
CPU states: 99.9% idle, 0.0% user, 0.0% kernel
Memory: 0 total, 204M avail, page size 4K

      PID   TID PRI STATE    HH:MM:SS    CPU  COMMAND
  1040402     1  10 Rply      0:00:00   0.01% top
     4110     2  21 Rcv       0:00:01   0.00% io-pkt-v4-hc
 ...

[1] + Stopped top 			# Po wcisnieciu Ctrl+Z
# bg %1 			# Przeniesienie do tla
# jobs -l
[1] + 1040402 Running top
# fg %1 			# Przeniesienie do pierwszego planu
# 			# Po wcisnieciu Ctrl+C
\end{lstlisting}

\end{example}


\begin{example}[Procesy tła i~procesy pierwszoplanowe]

Uruchamiając i testując programy często zachodzi potrzeba zbierania informacji o stanie systemu. Statystyki dostarczają specjalizowane programy, których przegląd umieszczono w tabeli~\ref{tab:statystyki}.

\begin{table}[h!]
\centering
\caption{Statystyki stanu systemu}
\setlength{\arrayrulewidth}{1pt}
\setlength{\tabcolsep}{6pt}
\renewcommand{\arraystretch}{1.2}
\begin{tabular}{ |p{0.15\textwidth}|p{0.4\textwidth}|}
\hline \rowcolor{gray}
\textbf{Polecenie} & \textbf{Opis} \\ \hline
\mbox{\lstinline[style=MyBashStyle]{ps [-f]}} & Wyświetla listę procesów i ich status \\ \hline
\mbox{\lstinline[style=MyBashStyle]{top}} & Wyświetla statystyki wydajnościowe sytemu \\ \hline
\mbox{\lstinline[style=MyBashStyle]{pidin}}  & Wyświetla statystyki systemowe \\ \hline
\mbox{\lstinline[style=MyBashStyle]{hogs}}  & Wyświetla listę procesów, wg użycia procesora \\ \hline
\mbox{\lstinline[style=MyBashStyle]{shomem [-S]}}  & Wyświetla informacje nt. użytej pamięci \\ \hline
\end{tabular}
\label{tab:statystyki}
\end{table}

Obejrzeć tablicę procesów wywołując następujące polecenie:
\begin{lstlisting}[style=MyBashStyle,deletekeywords={ps}]
# ps -f
  UID        PID       PPID  C STIME TTY          TIME CMD
    0      45068          1  - Oct22 ?        00:00:00 inetd
    0       4109          1  - Oct22 ?        00:00:00 sh
    0    1118226       4119  - 17:29 ?        00:00:00 ps -f
    0       4116          1  - Oct22 ?        00:00:00 sh
    0       4118          1  - Oct22 ?        00:00:00 sh
    0       4119          1  - Oct22 ?        00:00:00 sh
\end{lstlisting}

Znaczenie kolumn po wydaniu polecenie \lstinline[style=MyBashStyle]{ps -f} opisano w~tabeli~\ref{tab:opispol}.

\begin{table}[h!]
\centering
\caption{Opis pól wyświetlany przez polecenie ps}
\setlength{\arrayrulewidth}{1pt}
\setlength{\tabcolsep}{6pt}
\renewcommand{\arraystretch}{1.2}
\begin{tabular}{ |p{0.15\textwidth}|p{0.4\textwidth}|}
\hline \rowcolor{gray}
\textbf{Oznaczenie} & \textbf{Opis} \\ \hline
\mbox{\lstinline[style=MyBashStyle]{UID}} & Numer identyfikacyjny użytkownika, który uruchomił proces \\ \hline
\mbox{\lstinline[style=MyBashStyle]{PID}} & Numer identyfikacyjny procesu (potomnego) \\ \hline
\mbox{\lstinline[style=MyBashStyle]{PPID}}  & Numer identyfikacyjny procesu nadrzędnego (macierzystego) \\ \hline
\mbox{\lstinline[style=MyBashStyle]{C}}  & Wykorzystanie procesora \\ \hline
\mbox{\lstinline[style=MyBashStyle]{STIME}}  & Czas uruchomienia procesu \\ \hline
\mbox{\lstinline[style=MyBashStyle]{TTY}}  & Nazwa terminala kontrolującego (niewspierana) \\ \hline
\mbox{\lstinline[style=MyBashStyle]{TIME}}  & Czas działania procesu \\ \hline
\mbox{\lstinline[style=MyBashStyle]{CMD}}  & Komenda, która uruchomiła proces \\ \hline
\end{tabular}
\label{tab:opispol}
\end{table}

Wylistować zawartość statystyk systemowych za pomocą polecenia \lstinline[style=MyBashStyle]{pidin}, \lstinline[style=MyBashStyle]{hogs} oraz \lstinline[style=MyBashStyle]{showmem}.

\begin{lstlisting}[style=MyBashStyle,deletekeywords={ps}]
# pidin | more
...
# hogs
...
# showmem -S
...
\end{lstlisting}

\end{example}


\begin{example}[Usuwanie procesów] Ważną grupą poleceń są komendy umożliwiające przerwanie działania procesów, bądź przesłanie im sygnału. Ich zestawienie przedstawiono w~tabeli \ref{tab:usuwanie}.

\begin{table}[h!]
\centering
\caption{Usuwanie procesów}
\setlength{\arrayrulewidth}{1pt}
\setlength{\tabcolsep}{6pt}
\renewcommand{\arraystretch}{1.2}
\begin{tabular}{ |p{0.45\textwidth}|p{0.4\textwidth}|}
\hline \rowcolor{gray}
\textbf{Polecenie} & \textbf{Opis} \\ \hline
\mbox{\lstinline[style=MyBashStyle]{kill [-nazwa_sygnalu| -nr_sygnalu] PID}} & Przesłanie sygnału do procesu o nr \mbox{\lstinline[style=MyBashStyle]{PID}} \\ \hline
\mbox{\lstinline[style=MyBashStyle]{slay [-nazwa_sygnalu| -nr_sygnalu] pro}} & Przesłanie sygnału do procesu o~nazwie \mbox{\lstinline[style=MyBashStyle]{pro}} \\ \hline
\end{tabular}
\label{tab:usuwanie}
\end{table}

Na działający proces można oddziaływać wysyłając do niego sygnały. Sygnały zostały stworzone z~myślą o sytuacjach wyjątkowych, np. awariach systemu, czy błędach w pracy programu. Mechanizm sygnałów - ideą przypomina mechanizm przerwań. Proces po otrzymaniu sygnału może wykonać pewną procedurę, lub pozostawić obsługę sygnału systemowi. Nazwy sygnałów i ich liczbowe odpowiedniki możemy podejrzeć poleceniem (\lstinline[style=MyBashStyle]{kill -l}). Trzy często używane sygnały przedstawiono w tabeli~\ref{tab:wybranesygnaly}.

\begin{table}[h!]
\centering
\caption{Usuwanie procesów}
\setlength{\arrayrulewidth}{1pt}
\setlength{\tabcolsep}{6pt}
\renewcommand{\arraystretch}{1.2}
\begin{tabular}{ |p{0.1\textwidth}|p{0.1\textwidth}|p{0.4\textwidth}|}
\hline \rowcolor{gray}
\textbf{Nazwa sygnału} & \textbf{Numer sygnału} & \textbf{Opis} \\ \hline
\mbox{\lstinline[style=MyBashStyle]{INT}} & \mbox{\lstinline[style=MyBashStyle]{2}} &  Przerwanie z klawiatury \\ \hline
\mbox{\lstinline[style=MyBashStyle]{KILL}} & \mbox{\lstinline[style=MyBashStyle]{9}} &  Żądanie natychmiastowego zakończenia procesu \\ \hline
\mbox{\lstinline[style=MyBashStyle]{TERM}} & \mbox{\lstinline[style=MyBashStyle]{15}} &  Żądanie normalnego zakończenia procesu \\ \hline
\end{tabular}
\label{tab:wybranesygnaly}
\end{table}


W poniższym przykładzie powołuje się do życia proces \lstinline[style=MyBashStyle]{cat}, a~następnie przesyła się do niego sygnał \lstinline[style=MyBashStyle]{KILL}. Przykład ten wymaga użycia dwóch konsol. Przełączenia pomiędzy konsolami dokonujemy kombinacją klawiszy \lstinline[style=MyBashStyle]{Ctrl+Alt+1} oraz \lstinline[style=MyBashStyle]{Ctrl+Alt+2}.

\begin{lstlisting}[style=MyBashStyle,caption=Konsola 1]
# cat
Terminated
#
\end{lstlisting}

\begin{lstlisting}[style=MyBashStyle,caption=Konsola 2,deletekeywords={cat,ps}]
# ps -f
  UID        PID       PPID  C STIME TTY          TIME CMD
    0      45068          1  - Oct23 ?        00:00:00 inetd
    0       4109          1  - Oct23 ?        00:00:00 sh
    0       4110          1  - Oct23 ?        00:00:00 sh
    0       4111          1  - Oct23 ?        00:00:00 sh
    0       4112          1  - Oct23 ?        00:00:00 sh
    0      81943       4112  - Oct23 ?        00:00:00 cat
    0     110616       4109  - 14:58 ?        00:00:00 ps -f
# kill -TERM 81943
\end{lstlisting}

\end{example}



\subsection{Zmienne}

Zmienne środowiskowe tworzą tzw. środowisko, w którym wykonują się polecenia. Każde polecenie uruchomione w systemie działa w pewnym ,,otoczeniu'' zmiennych środowiskowych. Zmienne pozwalają na przechowywanie wartości, bądź zdefiniowanych przez użytkownika, bądź danych systemowych (zmienne środowiskowe). Powłoka umożliwia tworzenie zmiennych, przypisywanie wartości i modyfikacje oraz ich usuwanie. Podstawowe operacje na zmiennych przedstawiono w tabeli~\ref{tab:operacje}.

\begin{table}[h!]
\centering
\caption{Podstawowe operacje na zmiennych środowiskowych}
\setlength{\arrayrulewidth}{1pt}
\setlength{\tabcolsep}{6pt}
\renewcommand{\arraystretch}{1.2}
\begin{tabular}{ |p{0.3\textwidth}|p{0.4\textwidth}|}
\hline \rowcolor{gray}
\textbf{Polecenie} & \textbf{Opis} \\ \hline
\mbox{\lstinline[style=MyBashStyle]{ZMIENNA=wartosc}} & Przypisanie wartości do \mbox{\lstinline[style=MyBashStyle]{ZMIENNA}}. Jeśli zmienna nie istnieje, to jest tworzona \\ \hline
\mbox{\lstinline[style=MyBashStyle]{$ZMIENNA}} & Użycie wartości zmiennej (substytucja) \\ \hline
\mbox{\lstinline[style=MyBashStyle]{set}}  & Umożliwia m.in. wyświetlenie wszystkich zmiennych \\ \hline
\mbox{\lstinline[style=MyBashStyle]{unset ZMIENNA}}  & Usuwa zmienną \mbox{\lstinline[style=MyBashStyle]{ZMIENNA}} z pamięci (ze środowiska) \\ \hline
\mbox{\lstinline[deletekeywords={export}]{export ZMIENNA[=wartosc]}}  & Powoduje, że \mbox{\lstinline[style=MyBashStyle]{ZMIENNA}} jest przekazywana do środowiska każdego uruchamianego polecenia \\ \hline
\end{tabular}
\label{tab:operacje}
\end{table}

\begin{example}[Podstawowe operacje na zmiennych] Należy przetestować działanie poniższych poleceń.

\begin{lstlisting}[style=MyBashStyle]
# A="foo"
# echo $A
foo
# unset A
# echo $A

# set
...
# echo $PATH
/proc/boot:/bin:/usr/bin:/sbin:/usr/sbin
#
\end{lstlisting}


\end{example}


\begin{example}[Zmienne i~polecenia] Należy przetestować działanie poniższych poleceń.

\begin{lstlisting}[style=MyBashStyle]
# A=1 echo "Zmienna A: " $A
Zmienna A:
# B=2
# echo "Zmienna B: " $B
Zmienna B: 2
#
\end{lstlisting}

Polecenia są wykonywane w swoim własnym środowisku. Środowisko to jest zainicjalizowane niektórymi wartościami otrzymanymi od środowiska macierzystego. Aby zmienna \lstinline[style=MyBashStyle]{A} przenosiła się do środowiska uruchamianych poleceń, należy na niej wykonać polecenie \lstinline[deletekeywords={export}]{export A}. W powyższym przykładzie wartość zmiennej \lstinline[style=MyBashStyle]{A} nie została wyświetlona, ponieważ jest ona definiowana tylko dla środowiska polecenia \lstinline[style=MyBashStyle]{echo}, a podstawienia wartości \lstinline[style=MyBashStyle]{$A} dokonuje wcześniej powłoka. Zmienna \lstinline[style=MyBashStyle]{B} natomiast jest zdefiniowana w~środowisku powłoki, zatem jej wartość \lstinline[style=MyBashStyle]{$B} jest znana w chwili wywołania \lstinline[style=MyBashStyle]{echo}.

\end{example}


\begin{example}[Wyświetlanie kodu zakończenia]
Kod zakończenia ostatnio wykonanego polecenia zapamiętywany jest w specjalnej zmiennej \lstinline[style=MyBashStyle]{$?}. Wyświetl wartość tej zmiennej po wywołaniu polecenia, które zakończyło się sukcesem oraz po zakończeniu polecenia, które nie mogło się powieść.

\begin{lstlisting}[style=MyBashStyle]
# echo "dobre polecenie"
dobre polecenie
# echo $?
0
# zlepolecenie
sh: zlepolecenie: cannot execute - No such file or directory
# echo $?
126
\end{lstlisting}
\end{example}


\begin{example}[Skrypt i~zmienne środowiskowe]
Skrypty po uruchomieniu działają w swoim własnym środowisku. Niektóre zmienne ze środowiska macierzystego (np. ze środowiska powłoki \lstinline[style=MyBashStyle]{sh}) są kopiowane do środowiska skryptu, inne nie. Poniższy przykład ilustruje oddziaływanie między skryptem a jego środowiskiem oraz środowiskiem macierzystym.

Przejść do katalogu \lstinline[style=MyBashStyle]{/home} i~napisać skrypt \lstinline[style=MyBashStyle]{piszA} o~treści:

\begin{lstlisting}[style=MyBashStyle,deletekeywords={echo}]
echo "Zmienna A: " $A
\end{lstlisting}

W terminalu powłoki przeprowadź następujące eksperymenty:

\begin{lstlisting}[style=MyBashStyle,deletekeywords={echo}]
# unset A
# sh piszA
Zmienna A:

# A=10
# sh piszA
Zmienna A:

# A=10 sh piszA
Zmienna A: 10

# export A=20
# sh piszA
Zmienna A: 20
\end{lstlisting}

\end{example}

\subsection{Przetwarzanie tekstu}

Powłoka zawiera specjalne skrypty, które służą do wyświetlania, modyfikacji i formatowania tekstów. Podstawowe komendy zestawiono w tabeli~\ref{tab:wyswietl}.

\begin{table}[h!]
\centering
\caption{Wyświetlanie tekstu}
\setlength{\arrayrulewidth}{1pt}
\setlength{\tabcolsep}{6pt}
\renewcommand{\arraystretch}{1.2}
\begin{tabular}{ |p{0.3\textwidth}|p{0.4\textwidth}|}
\hline \rowcolor{gray}
\textbf{Polecenie} & \textbf{Opis} \\ \hline
\mbox{\lstinline[style=MyBashStyle]{cat [plik]}} & Wyświetlanie całej treści pliku (lub standardowego wejścia) \\ \hline
\mbox{\lstinline[style=MyBashStyle]{more}} & Wyświetlanie ze stronicowaniem \\ \hline
\mbox{\lstinline[style=MyBashStyle]{less}} & Wyświetlanie ze stronicowaniem \\ \hline
\mbox{\lstinline[style=MyBashStyle]{head [-liczbalinii] [plik]}}  & Wyświetlenie kilku pierwszych wierszy tekstu \\ \hline
\mbox{\lstinline[style=MyBashStyle]{tail [-liczbalinii] [plik]}}  & Wyświetlenie kilku ostatnich wierszy tekstu \\ \hline
\end{tabular}
\label{tab:wyswietl}
\end{table}

Jednocześnie możliwe jest filtrowanie tekstu za pomocą poleceń zaprezentowanych w~tabeli~\ref{tab:filtruj}.

\begin{table}[h!]
\centering
\caption{Filtrowanie tekstu}
\setlength{\arrayrulewidth}{1pt}
\setlength{\tabcolsep}{6pt}
\renewcommand{\arraystretch}{1.2}
\begin{tabular}{ |p{0.25\textwidth}|p{0.4\textwidth}|}
\hline \rowcolor{gray}
\textbf{Polecenie} & \textbf{Opis} \\ \hline
\mbox{\lstinline[style=MyBashStyle]{grep wzor [plik]}} & Wyświetlanie tylko linii tekstu pasujących do wzorca \\ \hline
\mbox{\lstinline[style=MyBashStyle]{sort [plik]}} & Wyświetlanie posortowanej treści \\ \hline
\mbox{\lstinline[style=MyBashStyle]{wc [plik]}} & Zliczanie wierszy, słów, znaków, itp. \\ \hline
\end{tabular}
\label{tab:filtruj}
\end{table}

\subsection{Strumienie wejściowo-wyjściowe}

Uruchomiony program ma do dyspozycji trzy strumienie danych. Każdy z tych plików jest reprezentowany przez małą liczbę całkowitą, zwaną deskryptorem pliku (uchwytem do pliku) - patrz tabela~\ref{tab:strumienie}.

\begin{table}[h!]
\centering
\caption{Strumienie wejście-wyjście}
\setlength{\arrayrulewidth}{1pt}
\setlength{\tabcolsep}{6pt}
\renewcommand{\arraystretch}{1.2}
\begin{tabular}{ |p{0.15\textwidth}|p{0.15\textwidth}|p{0.4\textwidth}|}
\hline \rowcolor{gray}
\textbf{Strumień} & \textbf{Deskryptor} & \textbf{Opis} \\ \hline
\mbox{\lstinline[style=MyBashStyle]{stdin}} & \mbox{\lstinline[style=MyBashStyle]{0}} &  Standardowy strumień wejściowy  \\ \hline
\mbox{\lstinline[style=MyBashStyle]{stdout}} & \mbox{\lstinline[style=MyBashStyle]{1}} &  Standardowy strumień wyjściowy  \\ \hline
\mbox{\lstinline[style=MyBashStyle]{stderr}} & \mbox{\lstinline[style=MyBashStyle]{2}} &  Standardowy strumień dla komunikatów o~błędach  \\ \hline
\end{tabular}
\label{tab:strumienie}
\end{table}

Wiele programów działa w ten sposób, że pobierają tekst ze strumienia \lstinline[style=MyBashStyle]{stdin}, przetwarzają go wysyłając rezultat do strumienia \lstinline[style=MyBashStyle]{stdout}, a komunikaty o błędach do \lstinline[style=MyBashStyle]{stderr}, tak, jak przedstawiono to na rysunku~\ref{fig:strumienie}. Przy zwykłym uruchomieniu programu, strumień \lstinline[style=MyBashStyle]{stdin} przychodzi z klawiatury a strumienie \lstinline[style=MyBashStyle]{stdout} i \lstinline[style=MyBashStyle]{stderr} są wyświetlane w~konsoli.

\begin{figure}[!h]
\centering
\includegraphics[width=0.5\textwidth]{img/strumienie}
\caption{Strumienie wejście-wyjście}
\label{fig:strumienie}
\end{figure}

Źródło oraz wyjście danych można zmienić stosując tzw. przekierowanie strumieni. Jako strumień wejściowy do programu można wysłać zawartość pliku. Służy do tego zapis:

\begin{lstlisting}[style=MyBashStyle]
# polecenie < plik
\end{lstlisting}


\begin{example}[Przekierowanie strumienia wejściowego]

Polecenie \lstinline[style=MyBashStyle]{cat} wywołane bez żadnych argumentów kopiuje strumień \lstinline[style=MyBashStyle]{stdin} do strumienia \lstinline[style=MyBashStyle]{stdout}. Podaj zawartość pliku \lstinline[style=MyBashStyle]{piszA} jako strumień wejściowy do polecenia \lstinline[style=MyBashStyle]{cat}.

\begin{lstlisting}[style=MyBashStyle]
# cat < piszA
\end{lstlisting}

Strumień wyjściowy \lstinline[style=MyBashStyle]{stdout} możemy przekierować do pliku za pomocą zapisu:

\begin{lstlisting}[style=MyBashStyle]
# cat > mojplik
pierwszy wiersz
drugi wiersz
trzeci wiersz
#                          # Ctrl+C
# cat mojplik
\end{lstlisting}

W ten sposób wyjście z programu zostanie zapisane do pliku \lstinline[style=MyBashStyle]{plik}. Można również sprawić, aby wyjście zostało dopisane do pliku:

\begin{lstlisting}[style=MyBashStyle]
# cat >> mojplik
czwarty wiersz
piaty wiersz
#                          # Ctrl+C
# cat mojplik
\end{lstlisting}
\end{example}


\begin{example}[Przekierowanie strumienia wyjściowego]

Wylistuj zawartość bieżącego katalogu i~przekieruj listing do pliku \lstinline[style=MyBashStyle]{mojplik2}. Wyświetl zawartość pliku \lstinline[style=MyBashStyle]{mojplik2}.


\begin{lstlisting}[style=MyBashStyle]
# ls -la > mojplik2
# cat mojplik2
\end{lstlisting}

Wyjście \lstinline[style=MyBashStyle]{stderr} można przekierować do pliku w następujący sposób:

\begin{lstlisting}[style=MyBashStyle]
# polecenie 2> plikzbledem
# cat plikzbledem
\end{lstlisting}

lub

\begin{lstlisting}[style=MyBashStyle]
# polecenie 2>> plikzbledem
# cat plikzbledem
\end{lstlisting}
\end{example}

\begin{example}[Przekierowanie strumienia stderr]

Próba skopiowania nieistniejącego pliku spowoduje wyświetlenie komunikatu o błędzie. Przekieruj ten komunikat do pliku \lstinline[style=MyBashStyle]{plikzbledem2} po czym wyświetl zawartość pliku

\begin{lstlisting}[style=MyBashStyle]
# cp pliknieist . 2> plikzbledem2
# cat plikzbledem2
\end{lstlisting}
\end{example}

\begin{example}[Jednoczesne przekierowanie strumienia stdin i stdout]
Podaj zawartość pliku \lstinline[style=MyBashStyle]{piszA} na wejście polecenia \lstinline[style=MyBashStyle]{cat} a wyjście przekieruj do pliku wyjscie. Wyświetl zawartość pliku \lstinline[style=MyBashStyle]{wyjscie}.

\begin{lstlisting}[style=MyBashStyle]
# cat < piszA > wyjscie
# cat wyjscie
\end{lstlisting}
\end{example}

\begin{example}[Potoki i~filtrowanie wyników]
Polecenia można łączyć w potoki (ang. pipes). Dokonuje się tego poprzez zapis:

\begin{lstlisting}[style=MyBashStyle]
# polecenie1 | polecenie 2 | ...
\end{lstlisting}

\begin{figure}[!h]
\centering
\includegraphics[width=1.0\textwidth]{img/potoki}
\caption{Zasada działania potoków (pipes)}
\label{fig:potoki}
\end{figure}

Wyjście \lstinline[style=MyBashStyle]{stdout} programu1 jest podawane na wejście \lstinline[style=MyBashStyle]{stdin} programu2 itd. Kodem zakończenia potoku jest kod zwrócony przez ostatnie polecenie.

Polecenie \lstinline[style=MyBashStyle]{grep} filtruje przychodzący tekst przepuszczając tylko linie tekstu pasujące do wzorca. Należy "przepompować" zawartość pliku \lstinline[style=MyBashStyle]{piszA} poleceniem \lstinline[style=MyBashStyle]{cat} do polecenia \lstinline[style=MyBashStyle]{grep}, aby wyszukać linie zawierające literę \lstinline[style=MyBashStyle]{A}.

\begin{lstlisting}[style=MyBashStyle]
# cat piszA | grep 'A'
\end{lstlisting}

\end{example}

\begin{example}[Sortowanie i obcinanie wyników]

Wyświetl listę pięciu największych plików w bieżącym katalogu:

\begin{lstlisting}[style=MyBashStyle]
# ls -s | sort -n | tail -5
\end{lstlisting}
\end{example}

\begin{example}[Prosta sekwencja poleceń]

Polecenia można grupować w tzw. sekwencje. Służą do tego operatory \lstinline[style=MyBashStyle]{;}, \lstinline[style=MyBashStyle]{&&} i~\lstinline[style=MyBashStyle]{||}. Polecenia połączone operatorem \lstinline[style=MyBashStyle]{;} wykonują się jedno po drugim bezwarunkowo.

W jednej linii zdefiniuj zmienną o nazwie \lstinline[style=MyBashStyle]{FOO} i~wyświetl jej zawartość.

\begin{lstlisting}[style=MyBashStyle]
# FOO='To jest zmienna FOO';echo $FOO
\end{lstlisting}

\end{example}

\begin{example}[Sekwencja warunkowa - koniunkcja]
Operator \lstinline[style=MyBashStyle]{&&} umożliwia wykonanie następnego polecenia w sekwencji tylko, jeśli poprzednie wykonało się pomyślnie (kod wykonania \lstinline[style=MyBashStyle]{=0}). Polecenie \lstinline[style=MyBashStyle]{grep} zwraca \lstinline[style=MyBashStyle]{0} (sukces), jeśli poszukiwany wzór występuje w tekście. W przeciwnym przypadku zwraca wartość niezerową (błąd). Przeszukać zawartość pliku \lstinline[style=MyBashStyle]{piszA} w poszukiwaniu wzorca \lstinline[style=MyBashStyle]{A} i wyświetlić napis "Znaleziono", jeśli wzorzec wystąpił. Wykonać podobny zabieg dla wzorca \lstinline[style=MyBashStyle]{kawa}.

\begin{lstlisting}[style=MyBashStyle]
# cat piszA | grep 'A' && echo "Znaleziono"
# echo $?
0
# cat piszA | grep 'kawa' && echo "Znaleziono"
# echo $?
1
\end{lstlisting}
\end{example}

\begin{example}[Sekwencja warunkowa - alternatywa]
Powtórzyć poprzedni przykład używając operatora \lstinline[style=MyBashStyle]{||}, zamiast \lstinline[style=MyBashStyle]{&&}.
\begin{lstlisting}[style=MyBashStyle]
# cat piszA | grep 'A' || echo "Znaleziono"
# echo $?
0
# cat piszA | grep 'kawa' || echo "Znaleziono"
# echo $?
0
\end{lstlisting}
\end{example}

\subsection{Ćwiczenia}

\begin{myenumerate}
\item W systemie pomocy znajdź opis programu do archiwizacji danych tar. Spakować wszystkie pliki znajdujące się w katalogu bieżącym (tj. \lstinline[style=MyBashStyle]{/home}) do pojedynczego archiwum o nazwie \lstinline[style=MyBashStyle]{programy.tar} (opcje \lstinline[style=MyBashStyle]{-cvf}). Utworzyć w katalogu domowym, katalog o nazwie \lstinline[style=MyBashStyle]{Dokumenty}. Skopiować archiwum do katalogu \lstinline[style=MyBashStyle]{/home/Dokumenty} i~tam go rozpakować (opcje \lstinline[style=MyBashStyle]{-xvf}).
\end{myenumerate}

\cleardoublepage

\section{Wprowadzenie do QNX Momentics}


W celu opracowania programów pracujących pod kontrolą systemu operacyjnego czasu rzeczywistego (hard real time), będziemy potrzebowali Platformy Programistycznej QNX. W jej skład wchodzi pakiet QNX Momentics Tool Suite, składający się z elementów niezbędnych do rozwoju i uruchomienia oprogramowania pod QNX Neutrino - patrz rysunek~\ref{fig:qnxMomentics}. Do tej grupy należą kompilatory, linker, biblioteki i~inne komponenty systemu operacyjnego, zbudowane dla wszystkich architektur wspieranych przez QNX Neutrino. Posługując się QNX Momentics w systemie operacyjnym Windows i~Linux mamy do dyspozycji zintegrowane środowisko programistyczne na bazie projektu Eclipse.


\begin{figure}[!h]
\centering
\includegraphics[width=0.35\textwidth]{img/qnxMomentics}
\caption{Platforma rozwoju oprogramowania}
\label{fig:qnxMomentics}
\end{figure}

Dzięki platformie programistycznej możemy tworzyć oprogramowanie w konfiguracji cross development (host-target). Na maszynie typu host (Windows) będziemy dysponować platformą programistyczną QNX Momentics, natomiast na maszynie docelowej typu target (QNX Neutrino na maszynie wirtualnej) będziemy uruchamiać nasze programy. Komunikacja pomiędzy komputerem host i target odbywa się przez sieć, a wspomaga go proces \lstinline[style=MyBashStyle]{qconn}.

\begin{figure}[!h]
\centering
\includegraphics[width=0.5\textwidth]{img/konfiguracja}
\caption{Konfiguracja host-target}
\label{fig:konfiguracja}
\end{figure}

Niniejsze laboratorium będzie poswięcone kilka zagadnieniom:
\begin{myenumerate}
\item Podstawy obsługi QNX Momentics
\item Zarządzanie projektami C/C++
\item Edycja kodu źródłowego, kompilacja i~budowanie
\item Dostęp do platformy docelowej oraz uruchamianie aplikacji
\end{myenumerate}

\includepdf[scale=0.75,pages={1-8},pagecommand={\thispagestyle{fancy}{\subsection{Podstawy obsługi QNX Momentics}}},nup=2x4]{img/qnx1.pdf}
\includepdf[scale=0.75,pages={9-},pagecommand={\thispagestyle{fancy}{}},nup=2x4]{img/qnx1.pdf}


\includepdf[scale=0.75,pages={1-8},pagecommand={\thispagestyle{fancy}{\subsection{Zarządzanie projektami C/C++}}},nup=2x4]{img/qnx2.pdf}
\includepdf[scale=0.75,pages={9-},pagecommand={\thispagestyle{fancy}{}},nup=2x4]{img/qnx2.pdf}


\includepdf[scale=0.75,pages={1-8},pagecommand={\thispagestyle{fancy}{\subsection{Edycja kodu, kompilacja i~budowanie}}},nup=2x4]{img/qnx3.pdf}
\includepdf[scale=0.75,pages={9-},pagecommand={\thispagestyle{fancy}{}},nup=2x4]{img/qnx3.pdf}

\includepdf[scale=0.75,pages={1-8},pagecommand={\thispagestyle{fancy}{\subsection{Dostęp do platformy docelowej oraz uruchamianie aplikacji}}},nup=2x4]{img/qnx4.pdf}
\includepdf[scale=0.75,pages={9-},pagecommand={\thispagestyle{fancy}{}},nup=2x4]{img/qnx4.pdf}

%\clearpage
%\phantomsection
%{\label{viRef}
%\includepdf[scale=0.95,pages={1},pagecommand={\thispagestyle{fancy}},pagecommand={\thispagestyle{fancy}{\label{subsec:vi}}},lastpage=1,angle=90]{img/viRef.pdf}
%}

%\subsection{Zarządzanie projektami C}
%
%\subsection{Edycja kodu źródłowego, kompilacja i~budowanie}
%
%\subsection{Dostęp do platformy docelowej oraz uruchamianie aplikacji}
%\subsection{Ćwiczenia}

\cleardoublepage

\section{Wprowadzenie do programowania w~języku C}

\subsection{Wstęp}
Treść laboratorium zawiera krótkie wprowadzenie do języka programowania C jako tego, który będzie wiodący w dalszych etapach zajęć. Wprowadzenie to pozwoli zacząć programować w języku C, tak szybko, jak to możliwe. Nie należy jednak traktować tej części laboratorium, jako substytutu kursów języka C, prezentowanych na innych wykładach i laboratoriach oraz w podręcznikach poświęconych językowi C. Pozostałe podrozdziały omawiają również podstawy użytkowania kompilatora oraz narzędzi make obecnych w~systemie QNX.

\subsection{Kompilowanie i~uruchamianie programów}

\subsubsection{Kompilator qcc}

Kody źródłowe, napisane w języku C muszą zostać wstępnie przetworzone (preprocessing), skompilowane (compiling) i skonsolidowane (linking), aby utworzyć plik wykonywalny. Używając linii poleceń, napisany program można skompilować w systemie QNX za pomocą następujących komend:


Dla języka C:

\begin{lstlisting}[style=MyBashStyle]
qcc [opcje] [operandy]
\end{lstlisting}

Dla języka C++:

\begin{lstlisting}[style=MyBashStyle]
QCC [opcje] [operandy]
\end{lstlisting}

Wybrane opcje kompilatora qcc zestawiono w~tabeli~, natomiast operandy stanowią pliki źródłowe (\lstinline[style=MyBashStyle]{*.c}) oraz pliki typu (\lstinline[style=MyBashStyle]{*.o}).

\begin{table}[h!]
\centering
\caption{Wybrane opcje kompilatora \lstinline[style=MyBashStyle]{qcc}}
\setlength{\arrayrulewidth}{1pt}
\setlength{\tabcolsep}{6pt}
\renewcommand{\arraystretch}{1.2}
\begin{tabular}{ |p{0.20\textwidth}|p{0.4\textwidth}|}
\hline \rowcolor{gray}
\textbf{Opcje} & \textbf{Opis} \\ \hline
\mbox{\lstinline[style=MyBashStyle]{-c}} & Tylko kompilacja \\ \hline
\mbox{\lstinline[style=MyBashStyle]{-E}} & Preprocessing na standardowe wyjście \\ \hline
\mbox{\lstinline[style=MyBashStyle]{-g}} & Kompilacja z debugowaniem \\ \hline
\mbox{\lstinline[style=MyBashStyle]{-I path [:path ...]}} & Ustawia ścieżkę przeszukiwania dla dyrektyw \mbox{\lstinline[style=MyBashStyle]{#include}}  \\ \hline
\mbox{\lstinline[style=MyBashStyle]{-L path [:path ...]}} & Ustawia ścieżkę przeszukiwania dla bibliotek  \\ \hline
\mbox{\lstinline[style=MyBashStyle]{-lplik}} & Dołącza bibliotekę o nazwie lib\underline{plik}.a lub lib\underline{plik}.so.  \\ \hline
\mbox{\lstinline[style=MyBashStyle]{-O1}} & Kompilowanie z optymalizacją \mbox{\lstinline[style=MyBashStyle]{O1}}.  \\ \hline
\mbox{\lstinline[style=MyBashStyle]{-O2}} & Kompilowanie z optymalizacją \mbox{\lstinline[style=MyBashStyle]{O2}}.  \\ \hline
\mbox{\lstinline[style=MyBashStyle]{-O3}} & Kompilowanie z optymalizacją \mbox{\lstinline[style=MyBashStyle]{O3}}.  \\ \hline
\mbox{\lstinline[style=MyBashStyle]{-o outfile}} & Ustala nazwę pliku wyjściowego (wykonywalnego). Domyślnie \mbox{\lstinline[style=MyBashStyle]{a.out}}.  \\ \hline
\mbox{\lstinline[style=MyBashStyle]{-Wall}} & Wyświetla wszystkie ostrzeżenia kompilatora.  \\ \hline
\mbox{\lstinline[style=MyBashStyle]{-pedantic}} & Wyświetla wszystkie ostrzeżenia kompilatora wymagane przez ANSI C.  \\ \hline
\end{tabular}
\label{tab:opcjeqcc}
\end{table}


\begin{example}{[Konfiguracja środowiska pracy]}
W~trakcie tego laboratorium będziemy głównie pracować w~linii poleceń. Zanim przejdziemy do właściwych przykładów, musimy skonfigurować środowisko pracy.

\begin{myenumerate}
\item  Ustawiamy zmienne środowiskowe w~systemie Windows. W tym celu należy:
\begin{myitemize}
\item Otworzyć linię poleceń w~systemie Windows, tj. nacisnąć przycisk \lstinline[style=MyBashStyle]{Start}, a~w~okienku \lstinline[style=MyBashStyle]{Wyszukaj programy i pliki} wpisać nazwę \lstinline[style=MyBashStyle]{cmd}.
\item W linii poleceń, uruchomić skrypt \lstinline[style=MyBashStyle]{C:\qnx660\qnx660-env.bat}.
\end{myitemize}

\item Na pulpicie tworzymy katalog o~nazwie \lstinline[style=MyBashStyle]{tmp}.
\item W~linii poleceń przechodzimy do katalogu \lstinline[style=MyBashStyle]{tmp} wpisując komendę \lstinline[style=MyBashStyle]{cd C:\Users\prog_N\Desktop\tmp}, gdzie \lstinline[style=MyBashStyle]{N} jest numerem komputera, do którego zalogowany jest użytkownik.
\item Uruchamiamy maszynę wirtualną z~systemem operacyjnym QNX oraz sprawdzamy IP maszyny za pomocą polecenia \lstinline[style=MyBashStyle]{ifconfig}.
\item Uruchamiamy środowisko do programowania aplikacji QNX Momentics oraz konfigurujemy dostęp do platformy docelowej w~kartach \lstinline[style=MyBashStyle]{Target Navigator} oraz \lstinline[style=MyBashStyle]{Target File System Navigator}.
\end{myenumerate}
\end{example}

\begin{example}{[Pierwszy program...]} \label{ex:pierwszy}

Otworzyć edytor tekstu \lstinline[style=MyBashStyle]{Notatnik}. Wpisać treść poniższego kodu i~zapisać plik pod nazwą \lstinline[style=MyBashStyle]{hello.c} w~katalogu \lstinline[style=MyBashStyle]{tmp} na \lstinline[style=MyBashStyle]{Pulpicie}.

\lstinputlisting[caption=Pierwszy program...,style=MyCStyle,label=src:kod3]{src/lab3/hello.c}

W~linii poleceń (Windows) skompilować program \lstinline[style=MyBashStyle]{hello.c} wydając polecenie:

\begin{lstlisting}[style=MyBashStyle]
> qcc -Wall hello.c
> ls
...
\end{lstlisting}

Proces tworzenia pliku wykonywalnego z~nadaniem nazw plików wyjściowych można podzielić na dwa etapy: kompilacja i~budowanie.

\begin{lstlisting}[style=MyBashStyle]
> qcc -Wall -c hello.c
> qcc hello.o -o hello
> ls
...
\end{lstlisting}

Kod źródłowy \lstinline[style=MyBashStyle]{hello.c} możemy jednocześnie skompilować i~zbudować z~nadaniem nazwy plikowi wyjściowemu.

\begin{lstlisting}[style=MyBashStyle]
> qcc -Wall hello.c -o hello2
> ls
...
\end{lstlisting}

Ostatnim etapem tego przykładu będzie uruchomienie programu \lstinline[style=MyBashStyle]{hello} na maszynie z~systemem operacyjnym QNX. Poprzez widok \lstinline[style=MyBashStyle]{Target File System Navigator} w środowisku QNX Momentics przekopiować program wykonywalny \lstinline[style=MyBashStyle]{hello} z~katalogu \lstinline[style=MyBashStyle]{tmp} do katalogu \lstinline[style=MyBashStyle]{/home} na maszynie wirtualnej. Uruchomić program poprzez polecenie:

\begin{lstlisting}[style=MyBashStyle]
# ./hello

Hello world !!!

#
\end{lstlisting}
\end{example}

\subsubsection{Sterowanie procesem budowania programów}

Kompilacja i uruchamianie projektu, składającego się z wielu plików źródłowych, w których zależności są złożone, może być uciążliwa. Istnieją programy narzędziowe, które ułatwiają zarządzanie złożonymi programami. Jednym z nich jest narzędzie \lstinline[style=MyBashStyle]{make}, które przetwarza specjalny plik \lstinline[style=MyBashStyle]{Makefile}.

W pliku \lstinline[style=MyBashStyle]{Makefile} występują tzw. reguły (ang. rules) mające następującą formę:

\begin{lstlisting}[style=MyBashStyle]
target: prerequisite [prerequisites]
<tab> commands
\end{lstlisting}

Cel (ang. target) - jest zazwyczaj plikiem, który chcemy utworzyć. Zależność (ang. prerequisite) - do utworzenia celu wymagane są zazwyczaj pliki źródłowe; nazywamy je zależnościami. Polecenia (ang. commands) - są krokami (np. wywołania kompilatora lub polecenia powłoki), które należy wykonać, aby utworzyć cel.

\begin{example}{[Pierwszy plik Makefile]}
W~Notatniku napisać skrypt \lstinline[style=MyBashStyle]{Makefile} do uruchomienia programu zapisanego w~\lstinline[style=MyBashStyle]{hello.c} oraz zapisać go w~katalogu \lstinline[style=MyBashStyle]{tmp}.

\lstinputlisting[caption=Pierwszy skrypt Makefile,style=MyBashStyle,label=src:MakefilePierwszy]{src/lab3/MakefilePierwszy.make}

Wpisać w~wiersz poleceń (Windows) następujące wywołania:

\begin{lstlisting}[style=MyBashStyle,caption=Pierwszy plik Makefile]
> make -n 	# Wyswietla tylko polecenia, ale ich nie wykonuje
> make all
> ls
...
> make clean
> ls
...
\end{lstlisting}

\end{example}


\begin{example}{[Rozbudowany plik Makefile]} \label{ex:rozbudowany}
W bardziej rozbudowanych projektach stosuje się różnego typu zmienne, które ułatwiają proces konstrukcji pliku \lstinline[style=MyBashStyle]{Makefile}. Należą do nich zmienne definiowane przez użytkownika, zmienne standardowe (predefiniowane), np. dotyczące nazw kompilatorów i~flag wywołań oraz zmienne automatyczne, których wartości są obliczane, gdy reguła jest wykonywana. Wybrane zmienne standardowe i~automatyczne przedstawiono w tabelach~\ref{tab:zmiennestandardowe} oraz \ref{tab:zmienneautomatyczne}.

\begin{table}[h!]
\centering
\caption{Zmienne standardowe}
\setlength{\arrayrulewidth}{1pt}
\setlength{\tabcolsep}{6pt}
\renewcommand{\arraystretch}{1.2}
\begin{tabular}{ |p{0.15\textwidth}|p{0.4\textwidth}|}
\hline \rowcolor{gray}
\textbf{Argumenty} & \textbf{Opis} \\ \hline
\mbox{\lstinline[style=MyBashStyle]{CC}} & Nazwa kompilatora języka C \\ \hline
\mbox{\lstinline[style=MyBashStyle]{CXX}} & Nazwa kompilatora języka C++ \\ \hline
\mbox{\lstinline[style=MyBashStyle]{CFLAGS}} & Opcje kompilatora języka C \\ \hline
\mbox{\lstinline[style=MyBashStyle]{CXXFLAGS}} & Opcje kompilatora języka C++  \\ \hline
\mbox{\lstinline[style=MyBashStyle]{LFLAGS}} & Opcje dla linkera  \\ \hline
\end{tabular}
\label{tab:zmiennestandardowe}
\end{table}


\begin{table}[h!]
\centering
\caption{Zmienne automatyczne}
\setlength{\arrayrulewidth}{1pt}
\setlength{\tabcolsep}{6pt}
\renewcommand{\arraystretch}{1.2}
\begin{tabular}{ |p{0.15\textwidth}|p{0.4\textwidth}|}
\hline \rowcolor{gray}
\textbf{Argumenty} & \textbf{Opis} \\ \hline
\mbox{\lstinline[style=MyBashStyle]{<}} & Nazwa pliku pierwszej zależności \\ \hline
\mbox{\lstinline[style=MyBashStyle]{@}} & Nazwa pliku docelowego \\ \hline
\mbox{\lstinline[style=MyBashStyle]{^}} & Lista wszystkich zależności \\ \hline
\end{tabular}
\label{tab:zmienneautomatyczne}
\end{table}

Zapisać w~Notatniku i uruchomić skrypt \lstinline[style=MyBashStyle]{Makefile} ze zmiennymi standardowymi i automatycznymi.

\lstinputlisting[caption=Rozbudowany plik Makefile,style=MyBashStyle,label=src:MakefileDrugi]{src/lab3/MakefileDrugi.make}

W przykładzie zastosowano zmienne definiowane przez użytkownika, zmienne standardowe oraz zmienne automatyczne. Dodatkowo użyto reguły \lstinline[style=MyBashStyle]{.PHONY}, służącej do poinstruowania narzędzia \lstinline[style=MyBashStyle]{make}, że reguły \lstinline[style=MyBashStyle]{all} oraz \lstinline[style=MyBashStyle]{clean} są regułami specjalnymi, a nie nazwami plików.
\end{example}


\begin{example}{[Makefile w środowisku QNX Momentics]}
Oczywiście, możemy wykorzystać plik \lstinline[style=MyBashStyle]{Makefile} utworzony w~przykładzie \ref{ex:rozbudowany} oraz kod źródłowy~\ref{src:kod3} zapisany w~przykładzie~\ref{ex:pierwszy} w~środowisku QNX Momentics. W~tym celu należy:

\begin{myenumerate}
\item W środowisku QNX Momentics utworzyć pusty projekt \lstinline[style=MyBashStyle]{C Project}, który należy nazwać jako \lstinline[style=MyBashStyle]{hello}.
\item W~kolejnym oknie z~katalogu \lstinline[style=MyBashStyle]{Executable} jako typ projektu wybrać \lstinline[style=MyBashStyle]{Empty project}, a~jako kompilator \lstinline[style=MyBashStyle]{QNX QCC}. Pozostawić domyślną konfigurację. 
\item Przekopiować zawartość skryptu \ref{ex:rozbudowany} oraz kodu źródłowego~\ref{src:kod3} do bieżącego projektu. Zbudować projekt klikając prawym przyciskiem na projekt i~wciskając opcję \lstinline[style=MyBashStyle]{Build Project}.
\item Skonfigurować środowisko do uruchamiania programu i~uruchomić zbudowany program na maszynie QNX.
\item Usunąć pliki tymczasowe z~projektu klikając prawym przyciskiem na projekt i~wciskając opcję \lstinline[style=MyBashStyle]{Clean Project}.
\end{myenumerate}
\end{example}

\subsection{Podstawy języka C}

\subsubsection{Typy zmiennych}

Niektóre wbudowane typy zmiennych przedstawiono w~tabeli~\ref{tab:typyzmiennych}.

\begin{table}[h!]
\centering
\caption{Typy zmiennych}
\setlength{\arrayrulewidth}{1pt}
\setlength{\tabcolsep}{6pt}
\renewcommand{\arraystretch}{1.2}
\begin{tabular}{ |p{0.1\textwidth}|p{0.45\textwidth}|}
\hline \rowcolor{gray}
\textbf{Typ} & \textbf{Opis} \\ \hline
\mbox{\lstinline[style=MyBashStyle]{int}} & integer \\ \hline
\mbox{\lstinline[style=MyBashStyle]{short}} & short integer \\ \hline
\mbox{\lstinline[style=MyBashStyle]{long}} & long integer \\ \hline
\mbox{\lstinline[style=MyBashStyle]{float}} & single precision real (floating point) variable \\ \hline
\mbox{\lstinline[style=MyBashStyle]{double}} & double precision real (floating point) variable \\ \hline
\mbox{\lstinline[style=MyBashStyle]{char}} & character variable (single byte) \\ \hline
\end{tabular}
\label{tab:typyzmiennych}
\end{table}

\subsubsection{Pętle}

Większość programów zawiera pętle, umożliwiające powtarzanie określonych czynności. W języku C istnieje kilka różnych sposobów tworzenia pętli. Dwie najbardziej rozpowszechnione to pętla \lstinline[style=MyCStyle]{while} i~\lstinline[style=MyCStyle]{for}. Składnia poleceń podana jest poniżej.


\begin{lstlisting}[style=MyCStyle]
while (expression)
{
...block of statements to execute...
}
\end{lstlisting}

\begin{lstlisting}[style=MyCStyle]
for (expression_1; expression_2; expression_3)
{
...block of statements to execute...
}
\end{lstlisting}

Pętla \lstinline[style=MyCStyle]{while} jest kontynuowana do momentu, w~którym  wyrażenie logiczne jest prawdą, przy czym warunek ten jest sprawdzany po wejściu do pętli. Pętla \lstinline[style=MyCStyle]{for} jest równoważna następującej pętli \lstinline[style=MyCStyle]{while}.

\begin{lstlisting}[style=MyCStyle]
expression_1;
while (expression_2)
{
...block of statements...
expression_3;
}
\end{lstlisting}

Przykłady zastosowania pętli.

\begin{myitemize}
\item Pętla \lstinline[style=MyCStyle]{while}.
\end{myitemize}

\begin{lstlisting}[style=MyCStyle]
i = initial_i;

while (i < i_max)
{
...block of statements...
i = i + i_increment;
}
\end{lstlisting}

\begin{myitemize}
\item Pętla \lstinline[style=MyCStyle]{for}.
\end{myitemize}

\begin{lstlisting}[style=MyCStyle,caption=Pętla for - przykład]
for (i = initial_i; i <= i_max; i = i + i_increment)
{
...block of statements...
}
\end{lstlisting}


\subsubsection{Konstrukcje warunkowe}

Składnia wyrażeń warunkowych w~konstrukcji \lstinline[style=MyCStyle]{if} wygląda następująco: 

\begin{lstlisting}[style=MyCStyle]
if (conditional_1) 
{
...block of statements executed if conditional_1 is true...
}
else if (conditional_2)
{
...block of statements executed if conditional_2 is true...
}
else
{
...block of statements executed otherwise...
}

\end{lstlisting}

Innego typu często używaną konstrukcją warunkową jest \lstinline[style=MyCStyle]{switch}: 

\begin{lstlisting}[style=MyCStyle]
switch (expression)
{
	case const_expression_1:
	{
	...block of statements...
		break;
	}
	case const_expression_2:
	{
	...block of statements...
		break;
	}
	default:
	{
	...block of statements..
	}
}
\end{lstlisting}



\begin{example}{[Makefile z dołączoną biblioteką]}
Poniższy kod źródłowy oblicza wartości funkcji $sinus$ dla kątów od $0^{\circ}-360^{\circ}$. 

\lstinputlisting[caption=Drugi program...,style=MyCStyle,label=src:koddrugie]{src/lab3/sine.c}



\begin{myenumerate}
\item Napisać odpowiedni skrypt \lstinline[style=MyCStyle]{Makefile} w~\lstinline[style=MyCStyle]{Notatniku} oraz skompilować i~zbudować program z~linii poleceń. Za pomocą QNX Momentics przekopiować program wykonywalny na maszynę docelową. Sprawdzić poprawność działania programu uruchamiając go z~linii poleceń pod QNX-em. 

Uwaga: w~skrypcie \lstinline[style=MyCStyle]{Makefile} należy zapisać informację o~dołączeniu biblioteki matematycznej \mbox{\lstinline[style=MyCStyle]{libm.a}}. Biblioteka ta zawiera m.in. kody funkcji trygonometrycznych. 
\item  Alternatywnie zadanie można wykonać w środowisku QNX Momentics. Przekopiować kod źródłowy oraz skrypt \lstinline[style=MyCStyle]{Makefile} do IDE. Z~pliku \lstinline[style=MyCStyle]{Makefile} usunąć wpis o~dołączeniu biblioteki matematycznej. Zamiast informacji w~skrypcie, bibliotekę matematyczną należy dołączyć poprzez własności projektu. 
\end{myenumerate} 

\end{example} 


\subsubsection{Operatory relacji}

Wyrażenia warunkowe zawierają operacje logiczne, dzięki którym można porównać różnego typu wielkości. Lista operatorów porównania jest podana w~tabeli~\ref{tab:operatoryrelacji}. 

\begin{table}[h!]
\centering
\caption{Operatory relacji}
\setlength{\arrayrulewidth}{1pt}
\setlength{\tabcolsep}{6pt}
\renewcommand{\arraystretch}{1.2}
\begin{tabular}{ |p{0.15\textwidth}|p{0.4\textwidth}|}
\hline \rowcolor{gray}
\textbf{Typ} & \textbf{Opis} \\ \hline
\mbox{\lstinline[style=MyBashStyle]{<}} & smaller than \\ \hline
\mbox{\lstinline[style=MyBashStyle]{<=}} & smaller than or equal to \\ \hline
\mbox{\lstinline[style=MyBashStyle]{==}} & equal to \\ \hline
\mbox{\lstinline[style=MyBashStyle]{!=}} & not equal to \\ \hline
\mbox{\lstinline[style=MyBashStyle]{>=}} & greater than or equal to \\ \hline
\mbox{\lstinline[style=MyBashStyle]{>}} & greater than \\ \hline
\end{tabular}
\label{tab:operatoryrelacji}
\end{table}

\subsubsection{Operatory logiczne}

Często w~instrukcjach warunkowych stosowane są operatory logiczne przedstawione w~tabeli~\ref{tab:operatorylogiczne}. Te z~kolei dzielą się na jednoargumentowe (unarne) i dwuargumentowe (binarne).

\begin{table}[h!]
\centering
\caption{Operatory logiczne}
\setlength{\arrayrulewidth}{1pt}
\setlength{\tabcolsep}{6pt}
\renewcommand{\arraystretch}{1.2}
\begin{tabular}{ |p{0.15\textwidth}|p{0.4\textwidth}|}
\hline \rowcolor{gray}
\textbf{Typ} & \textbf{Opis} \\ \hline
\mbox{\lstinline[style=MyBashStyle]{&&}} & and \\ \hline
\mbox{\lstinline[style=MyBashStyle]{||}} & or \\ \hline
\mbox{\lstinline[style=MyBashStyle]{!}} & not \\ \hline
\end{tabular}
\label{tab:operatorylogiczne}
\end{table}

\subsubsection{Wskaźniki}

Język programowania C pozwala na odwoływanie do zmiennych poprzez ich adres w~pamięci. Daje to dużą elastyczność przy programowaniu, ale powoduje również trudności dla nowicjuszy programujących w~C. Wszystkie zmienne w~programie rezydują w~pamięci. Rozważmy prosty przykład: 

\begin{lstlisting}[style=MyCStyle]
float x;
x = 6.5;
\end{lstlisting}

Czasami chcemy uzyskać informację, gdzie zmienna rezydują w~pamięci. Uzyskujemy to poprzez umieszczenie operatora adresu \lstinline[style=MyCStyle]{&} przed nazwą zmiennej. Język C pozwala na definiowanie wskaźników (ang. pointers), które przechowują adresy zmiennych. Tę sytuację ilustruje kolejny przykład. 

\begin{lstlisting}[style=MyCStyle]
float x;
float* px;
x = 6.5;
px = &x;
\end{lstlisting}

W~powyższym przykładzie zdefiniowano wskaźnik \lstinline[style=MyCStyle]{px} wskazujący na typ \lstinline[style=MyCStyle]{float} i~przypisano mu wartość adresu zmiennej \lstinline[style=MyCStyle]{x}. Zawartość pamięci wskazywanej przez wskaźnik można uzyskać poprzez operator dereferencji \lstinline[style=MyCStyle]{*}. Tak więc \lstinline[style=MyCStyle]{*px} zawiera wartość zmiennej \lstinline[style=MyCStyle]{x}. Opisaną sytuację ilustruje również rysunek~\ref{fig:wskaznik}.

\begin{figure}[!h]
\centering
\includegraphics[width=0.5\textwidth]{img/pointer}
\caption{Idea wskaźnika}
\label{fig:wskaznik}
\end{figure}

\begin{example}{[Zastosowania wskaźników]}
W~poniższym przykładzie pokazano kilka możliwości zastosowania wskaźników. Skompilować, zbudować oraz uruchomić przykład.  

\lstinputlisting[caption=Zastosowania wskaźników,style=MyCStyle,label=src:pointer]{src/lab3/pointer.c}

\end{example}

\subsubsection{Tablice}

W~języku C typy danych można umieszczać w~tablicach. Składnia jest przy tym następująca: 

\begin{lstlisting}[style=MyCStyle]
typ nazwatablicy[wymiar];
\end{lstlisting}

W~języku~C tablice zaczynają się od indeksu \lstinline[style=MyCStyle]{0}. Pozostałe elementy zajmują sąsiednie komórki w~pamięci. Język~C traktuje nazwę tablicy jako wskaźnik do jej pierwszego elementu. Tak więc, jeśli \lstinline[style=MyCStyle]{v} jest tablicą, \lstinline[style=MyCStyle]{*v} ma tą samą wartość co element tablicy \lstinline[style=MyCStyle]{v[0]},\lstinline[style=MyCStyle]{*(v+1)} ma tą samą wartość, co element tablicy \lstinline[style=MyCStyle]{v[1]}. Sytuację tę ilustruje rysunek~\ref{fig:tablica}. 

\begin{figure}[!h]
\centering
\includegraphics[width=0.5\textwidth]{img/tablica}
\caption{Tablica a~wskaźnika}
\label{fig:tablica}
\end{figure}

Skompilować, zbudować oraz uruchomić przykład. 

\lstinputlisting[caption=Zastosowania wskaźników i~tablic,style=MyCStyle,label=src:table]{src/lab3/table.c}

\subsubsection{Funkcje}

Funkcje w~języku~C pozwalają na znaczne uproszczenie kodów źródłowych. W trakcie laboratorium zetknęliśmy się już z~funkcją główną \lstinline[style=MyCStyle]{main}, a~także z~funkcjami matematycznymi i~funkcjami wejścia/wyjścia. Oprócz zastosowań funkcji bibliotecznych, programista może implementować własne funkcje. Nagłówek i~ciało funkcji mogą przyjąć następującą postać:

\begin{lstlisting}[style=MyCStyle]
typ nazwa_funkcji ( lista_argumentow )
{
	...deklaracja lokalne...

	...operacje...

	return zwracana_wartosc;
}
\end{lstlisting}

Argumenty w~wywołaniach funkcji można przekazywać przez wartość. Oznacza to, że w~ciele funkcji istnieje kopia argumentu wywołania. Jakakolwiek zmiana zmiennej przekazanej do funkcji nie powoduje zmiany wartości zmiennej poza ciałem funkcji. Aby zmienić wartość zmiennej przekazanej w~wywołaniu w~ciele funkcji, należy przekazać ja poprzez wskaźnik. Problem ten ilustrują poniższe przykłady.

\begin{example}{[Przekazywanie argumentów przez wartość]} Skompilować, zbudować oraz uruchomić przykład. 
\lstinputlisting[caption=Przekazywanie argumentów przez wartość,style=MyCStyle,label=src:exchange1]{src/lab3/exchange1.c}
\end{example} 

\begin{example}{[Przekazywanie argumentów przez wskaźnik]} Skompilować, zbudować oraz uruchomić przykład. 
\lstinputlisting[caption=Przekazywanie argumentów przez wskaźnik,style=MyCStyle,label=src:exchange2]{src/lab3/exchange2.c}
\end{example} 

\subsubsection{Przekazywanie argumentów z wiersza poleceń}

Często zdarza się (szczególnie w~systemach UNIX–owych), że argumenty przetwarzane w~programie są przekazywane poprzez wiersz poleceń. Typowe przykłady to \lstinline[style=MyCStyle]{ls -la}, czy też \lstinline[style=MyCStyle]{tail -20}. Argumenty z~wiersza poleceń pozwalają na uzyskanie większej elastyczności naszych programów. Można je przekazywać do programu poprzez funkcję główną \lstinline[style=MyCStyle]{main()}. Ogólna składnia funkcji ma następującą postać: 

\begin{lstlisting}[style=MyCStyle]
main(int argc, char* argv[]);
\end{lstlisting}

gdzie \lstinline[style=MyCStyle]{argc}, określa liczbę argumentów wywołania (łącznie z~nazwą wywoływanego programu), a~\lstinline[style=MyCStyle]{argv} jest tablicą ciągów znakowych \lstinline[style=MyCStyle]{char*}, w~której przechowywane są argumenty wywołania. Sposób użycia ilustruje poniższy przykład.

\begin{example}{[Argumenty wiersza poleceń]} Skompilować, zbudować oraz uruchomić przykład. 
\lstinputlisting[caption=Argumenty wiersza poleceń,style=MyCStyle,label=src:cmd]{src/lab3/cmd.c}
\end{example}

\subsubsection{Operacje I/O (wejścia/wyjścia)}

Język C, poprzez swoje biblioteki, dostarcza różnorodnych funkcji, pozwalających na obsługę wejścia/wyjścia. Na poziomie liter, funkcją która pobiera znak ze standardowego wejścia \lstinline[style=MyCStyle]{stdin}, jest \lstinline[style=MyCStyle]{getchar()}, podczas gdy funkcja \lstinline[style=MyCStyle]{putchar()} zapisuje jeden znak do standardowego wyjścia \lstinline[style=MyCStyle]{stdout}. Użycie funkcji ilustruje poniższy kod źródłowy.

\begin{example}{[Operacje wejścia i~wyjścia]} Skompilować, zbudować oraz uruchomić przykład. 
\lstinputlisting[caption=Użycie funkcji getchar() i putchar(),style=MyCStyle,label=src:io]{src/lab3/io.c}
\end{example}

Znak \lstinline[style=MyCStyle]{EOF} jest wartością końca pliku, zdefiniowaną w~nagłówku biblioteki standardowej. Zakończenie danych (koniec pliku) uzyskujemy poprzez kombinację klawiszy \lstinline[style=MyCStyle]{Ctrl+D}. Bardziej zaawansowaną funkcją pozwalającą pisać sformatowany tekst do standardowego wyjścia jest poznana wcześniej funkcja \lstinline[style=MyCStyle]{printf}. Funkcją, która umożliwia czytanie ze standardowego wejścia jest \lstinline[style=MyCStyle,morekeywords={scanf}]{scanf}. Składnie obu poleceń przedstawiono poniżej.

\begin{lstlisting}[style=MyCStyle,morekeywords={scanf}]
printf("format", zmienne);
scanf("format", &zmienne);
\end{lstlisting}

Odpowiednikiem powyższych wyrażeń, pozwalających pisać do tablic zmiennych typu \lstinline[style=MyCStyle]{char} są poniższe funkcje.

\begin{lstlisting}[style=MyCStyle,morekeywords={scanf,sprintf}]
sprintf(string, "format", zmienne);
scanf(string, "format", &zmienne);
\end{lstlisting}

String w~składni oznacza nazwę tablicy zmiennych typu \lstinline[style=MyCStyle]{char}, bądź wskaźnik do jej pierwszego elementu.

\subsubsection{Operacje I/O na plikach}

Podobne instrukcje wejścia/wyjścia istnieją w~przypadku obsługi plików. Ogólną składnię I/O w~przypadku plików podano poniżej. 

\begin{lstlisting}[style=MyCStyle]
FILE *fp;			/* wskaznik do pliku */	
fp = fopen(nazwa, tryb);		/* otworzenie pliku */
fscanf(fp, "format", lista_zmiennych);  /* czytanie z pliku */
fprintf(fp, "format", lista_zmiennych);   /* pisanie do pliku */
fclose(fp);		/* zamkniecie pliku */	
\end{lstlisting}

Tryb w~instrukcji \lstinline[style=MyCStyle]{fopen} określa cel otwarcia pliku. Dozwolone tryby przedstawiono w~tabeli~\ref{tab:operacjenaplikach}. 

\begin{table}[h!]
\centering
\caption{Operacje na plikach}
\setlength{\arrayrulewidth}{1pt}
\setlength{\tabcolsep}{6pt}
\renewcommand{\arraystretch}{1.2}
\begin{tabular}{ |p{0.15\textwidth}|p{0.4\textwidth}|}
\hline \rowcolor{gray}
\textbf{Typ} & \textbf{Opis} \\ \hline
\mbox{\lstinline[style=MyBashStyle]{r}} & czytanie z~pliku \\ \hline
\mbox{\lstinline[style=MyBashStyle]{w}} & pisanie do pliku \\ \hline
\mbox{\lstinline[style=MyBashStyle]{a}} & dodanie zawartości do pliku \\ \hline
\end{tabular}
\label{tab:operacjenaplikach}
\end{table}

\begin{example}{[Operacje IO na plikach]} Skompilować, zbudować oraz uruchomić przykład. 
\lstinputlisting[caption=Zastosowanie funkcji fopen() i~fclose(),style=MyCStyle,label=src:iofile]{src/lab3/iofile.c}
\end{example}

\subsection{Ćwiczenia}


%\lstinline[style=MyBashStyle]{}
%\begin{lstlisting}[style=MyBashStyle,caption=Rozbudowany plik Makefile]
%\end{lstlisting}

\begin{myenumerate}
\item Zmodyfikować program obliczający tablicę funkcji \lstinline[style=MyBashStyle]{sinus}, tak, aby zawierał instrukcję sterującą \lstinline[style=MyBashStyle]{for}. Utworzyć funkcję \lstinline[style=MyBashStyle]{sinus} i przenieść jej kod do oddzielnego pliku źródłowego \lstinline[style=MyBashStyle]{sinus.c}. Dołączyć plik nagłówkowy \lstinline[style=MyBashStyle]{sinus.h} z~deklaracją funkcji. Napisać odpowiedni skrypt \lstinline[style=MyBashStyle]{Makefile}. Wyniki zapisać do pliku \lstinline[style=MyBashStyle]{sinus.dat}.
\item Napisz funkcję sprawdzającą ile par liczb całkowitych z przedziału \lstinline[style=MyBashStyle]{<a,b>} jest zawartych w~kole o~średnicy \lstinline[style=MyBashStyle]{8}. ($x^2+y^2\leq 8$). Wartości \lstinline[style=MyBashStyle]{a} i~\lstinline[style=MyBashStyle]{b} powinny być zadawane z klawiatury i przekazywane jako parametry funkcji.
\item Napisać bibliotekę operacji wektorowych, wykorzystując struktury. Zdefiniować wektor jako strukturę:

\begin{lstlisting}[style=MyCStyle]
struct Wektor
{
	double x, y, z;
};
\end{lstlisting}

Biblioteka operacji powinna pozwalać na dodawanie, odejmowanie, mnożenie przez liczbę wektora, liczenie iloczynów skalarnych i~wektorowych. Uzupełnić program  o odpowiednie funkcje testowe.
\item Uzupełnić poprzedni program o operacje transformacji wektorów z układu do układu. Napisać funkcje odpowiadające za elementarne obroty wokół osi \lstinline[style=MyBashStyle]{x}, \lstinline[style=MyBashStyle]{y}, \lstinline[style=MyBashStyle]{z}.
\end{myenumerate}




\cleardoublepage

\section{Procesy i zarządzanie procesami}

\subsection{Wstęp}

Przykłady procesów w życiu codziennym: proces fizyczny, chemiczny,
technologiczny, proces sądowy, administracyjny. Proces to przebieg powiązanych
ze sobą zmian. Większość procesów zachodzi w~określonym środowisku (np.
w~urzędzie), podlega pewnym ograniczeniom (np.  procedurom prawnym) i~wymaga
pewnych zasobów (np.  pieniędzy podatników). Pewne zdarzenia w~procesie
mogą występować sekwencyjnie, inne mogą nakładać się na siebie w~czasie. Jeśli
jesteśmy w~stanie wyodrębnić unikalny ciąg występujących po sobie zdarzeń
i~stwierdzić, że są ze sobą jakoś powiązane, to taką sekwencję często określa
się mianem wątku. Proces, podobnie jak fabuła powieści, może obejmować wiele
wątków biegnących równocześnie.

W~informatyce proces jest dynamicznym obiektem utworzonym przez system
operacyjny w~celu wykonania pewnego programu. Procesowi przydzielane są zasoby
takie jak czas procesora, obszar pamięci operacyjnej, pliki, urządzenia
peryferyjne. W~pamięci komputera zwykle znajduje się wiele procesów, które są
na różnych etapach wykonania i~współzawodniczą o~zasoby. W~obrębie jednego
procesu występuje jeden lub więcej wątków (w uproszczeniu wątek obrazuje ciąg
wykonywanych po sobie instrukcji procesora), wątki mogą być wykonywane
równolegle (tj. wątek A biegnie równolegle z~wątkiem B).

Często proces jest utożsamiany z~programem uruchomionym w~systemie operacyjnym.
Rzeczywiste aplikacje mogą się jednak składać z~wielu procesów, które
komunikują się ze sobą za pomocą mechanizmów komunikacji międzyprocesowej
(inter-process communication, IPC). Zagadnieniom komunikacji IPC poświęcono
odrębne rozdziały tej instrukcji.

Obiekt-proces (tak jak większość obiektów w~informatyce) posiada pewne atrybuty,
oraz podlega regułom określającym jego czas życia. W~niniejszym laboratorium
omówimy następujące zagadnienia:
\begin{myitemize}
  \item wyświetlanie i~modyfikowanie atrybutów procesów (atrybuty),
  \item tworzenie i usuwanie procesów (czas życia).
\end{myitemize}

\subsection{Wyświetlanie i modyfikowanie atrybutów procesów}

Atrybuty to informacje przechowywane przez obiekt-proces na rzecz systemu
operacyjnego. Pozwalają systemowi i~użytkownikowi identyfikować procesy oraz
np. określać jakie zasoby są aktualnie przydzielone poszczególnym procesom.

Każdy proces w systemie QNX ma przypisany unikalny identyfikator PID (process
ID), który jest liczbą całkowitą. Każdy proces (oprócz procesu ,,głównego''
\texttt{procnto}) ma również swój proces macierzysty. Identyfikator PPID
(parent process ID) jest identyfikatorem PID procesu macierzystego. Proces
\texttt{procnto} ma identyfikator PID=1 i~nie ma identyfikatora PPID. Wybrane
atrybuty i~funkcje systemowe umożliwiające dostęp do nich przedstawiono
w~tabeli \ref{tab:HE6LE}.

%%%%%%%%%%%%%%%%%%%%%%%%%%%%%%%%%%%%%%%%%%%%%%%%%%%%%%%%%%%%%%%%%%%%%%%%%%%%%
\begin{table}[h!]
  \centering
  \caption{Niektóre atrybuty procesów oraz funkcje biblioteki systemowej
           umożliwiające dostęp do nich}
  \label{tab:HE6LE}
  \begin{tabular}{|c|c|c|}
    \hline
    \textbf{Atrybuty procesu} & \textbf{Wyświetlanie} & \textbf{Modyfikacja} \\ \hline
    Identyfikator PID                   & \texttt{getpid()}           & -- \\ \hline
    Identyfikator PPID                  & \texttt{getppid()}          & -- \\ \hline
    Priorytet i~strategia szeregowania  & \texttt{sched\_getparam()}  & \texttt{sched\_setparam()} \\ \hline
  \end{tabular}
\end{table}
%%%%%%%%%%%%%%%%%%%%%%%%%%%%%%%%%%%%%%%%%%%%%%%%%%%%%%%%%%%%%%%%%%%%%%%%%%%%%

Na ogół mamy do czynienia z sytuacją, kiedy procesów gotowych do wykonania jest
więcej niż umożliwiają to dostępne w~danej chwili zasoby. Procedura szeregująca
(scheduler) rozstrzyga, któremu z procesów (wątków) zostanie w~danej chwili
przydzielony czas procesora. Jednym z~istotnych parametrów rzutujących na
kolejność przydzielania czasu procesora jest priorytet -- każdy z procesów (i
wątków -- patrz następne laboratoria) ma przyporządkowany priorytet (process
priority).  Jest on miarą pilności wykonania danego procesu względem innych.
W systemie QNX Neutrino 6 priorytet jest liczbą z zakresu od 0 (najniższy) do
255 (najwyższy). System nakłada ograniczenia na dopuszczalne zakresy
priorytetów dla programów uruchamianych przez poszczególnych użytkowników
(tabela \ref{tab:2N6W7}):

%%%%%%%%%%%%%%%%%%%%%%%%%%%%%%%%%%%%%%%%%%%%%%%%%%%%%%%%%%%%%%%%%%%%%%%%%%%%%
\begin{table}[h!]
  \centering
  \caption{Priorytety w~QNX Neutrino 6}
  \label{tab:2N6W7}
  \begin{tabular}{|c|c|c|}
    \hline
    \textbf{Priorytet} & \textbf{Użytkownik} \\ \hline
    0 & Proces jałowy \\ \hline
    1 -- 63 & Wątki zwykłego użytkownika \\ \hline
    1 -- 255 & Wątki użytkownika \texttt{root} \\ \hline
  \end{tabular}
\end{table}
%%%%%%%%%%%%%%%%%%%%%%%%%%%%%%%%%%%%%%%%%%%%%%%%%%%%%%%%%%%%%%%%%%%%%%%%%%%%%

W systemie QNX są dostępne trzy strategie szeregowania:
\begin{myitemize}
  \item szeregowanie FIFO (FIFO scheduling),
  \item szeregowanie karuzelowe (round robin scheduling),
  \item szeregowanie sporadyczne (sporadic scheduling).
\end{myitemize}
Są one omówione szczegółowo w dokumentacji QNX~\cite{qnx}. W tabeli
\ref{tab:A9C2X} przedstawiono ich symbole, wykorzystywane w~wywołaniach
systemowych.


%%%%%%%%%%%%%%%%%%%%%%%%%%%%%%%%%%%%%%%%%%%%%%%%%%%%%%%%%%%%%%%%%%%%%%%%%%%%%
\begin{table}[h!]
  \centering
  \caption{Strategie szeregowania w~QNX Neutrino 6}
  \label{tab:A9C2X}
  \begin{tabular}{|c|c|c|}
    \hline
    \textbf{Symbol}           & \textbf{Opis} \\ \hline
    \texttt{SCHED\_FIFO}      & Szeregowanie FIFO \\ \hline
    \texttt{SCHED\_RR}        & Szeregowanie karuzelowe \\ \hline
    \texttt{SCHED\_SPORADIC}  & Szeregowanie sporadyczne \\ \hline
  \end{tabular}
\end{table}
%%%%%%%%%%%%%%%%%%%%%%%%%%%%%%%%%%%%%%%%%%%%%%%%%%%%%%%%%%%%%%%%%%%%%%%%%%%%%

%%%%%%%%%%%%%%%%%%%%%%%%%%%%%%%%%%%%%%%%%%%%%%%%%%%%%%%%%%%%%%%%%%%%%%%%%%%%%
\begin{example}{[Lista procesów]}
  \label{ex:TYRDA}
  W terminalu wykonać polecenie \texttt{ps -l}. Sprawdzić w {\color{red}dokumentacji},
  co oznaczają poszczególne kolumny wyświetlonego raportu.
\end{example}
%%%%%%%%%%%%%%%%%%%%%%%%%%%%%%%%%%%%%%%%%%%%%%%%%%%%%%%%%%%%%%%%%%%%%%%%%%%%%

%%%%%%%%%%%%%%%%%%%%%%%%%%%%%%%%%%%%%%%%%%%%%%%%%%%%%%%%%%%%%%%%%%%%%%%%%%%%%
\begin{example}{[Wyświetlanie i modyfikacja wybranych atrybutów procesu]}
  \label{ex:TEPBG}
  \lstinputlisting[style=MyCStyle,label=src:4JKQ7]{src/lab4/attributes1.c}
\end{example}
%%%%%%%%%%%%%%%%%%%%%%%%%%%%%%%%%%%%%%%%%%%%%%%%%%%%%%%%%%%%%%%%%%%%%%%%%%%%%


\subsection{Tworzenie procesów}

W systemie QNX Neutrino, modułem odpowiedzialnym za dynamiczne tworzenie,
usuwanie oraz zarządzanie procesami jest zawarty w mikrojądrze (procnto)
manager procesów. W systemie QNX istnieją różne metody tworzenia procesów.
Część z nich pochodzi wprost od systemów UNIX-owych, opartych o standard POSIX,
inne funkcje do tworzenia procesów są charakterystyczne tylko dla systemu QNX.
Podstawowe funkcje do tworzenia procesów przedstawiono w tabeli
\ref{tab:IMJR3}. W ramach laboratorium omówimy cztery funkcje, służące
tworzeniu procesów, tj. \texttt{system()}, \texttt{fork()}, \texttt{exec()},
\texttt{spawn()}.

%%%%%%%%%%%%%%%%%%%%%%%%%%%%%%%%%%%%%%%%%%%%%%%%%%%%%%%%%%%%%%%%%%%%%%%%%%%%%
\begin{table}[h!]
  \centering
  \caption{Metody tworzenia procesów w systemie QNX}
  \label{tab:IMJR3}
  \begin{tabular}{|l|p{0.75\textwidth}|}
    \hline
    \textbf{Funkcja}    & \textbf{Opis}  \\ \hline
    \texttt{system()}   & Wywołanie programów, poleceń systemowych, bądź skryptów \\ \hline
    \texttt{fork()}     & Utworzenie kopii procesu bieżącego \\ \hline
    \texttt{exec()}     & Zastąpienie bieżącego procesu nowym procesem -- rodzina funkcji \\ \hline
    \texttt{spawn()}    & Utworzenie procesu potomnego -- rodzina funkcji \\ \hline
    \texttt{vfork()}    & Utworzenie procesu potomnego i zablokowanie procesu macierzystego \\ \hline
    \texttt{forkpty()}  & Utworzenie procesu potomnego w oknie pseudoterminala \\ \hline
    \texttt{popen()}    & Uruchomienie programu jako procesu potomnego z
                          jednoczesnym utworzeniem łącza pomiędzy procesem
                          bieżącym, a potomnym \\ \hline
  \end{tabular}
\end{table}
%%%%%%%%%%%%%%%%%%%%%%%%%%%%%%%%%%%%%%%%%%%%%%%%%%%%%%%%%%%%%%%%%%%%%%%%%%%%%

Jednym z~najprostszych sposobów uruchomienia innego procesu z~poziomu programu
w~języku C jest użycie funkcji \texttt{system()} (przykład \ref{ex:V86MA}).
Poleceniem tym można uruchomić program w~sposób podobny, jak to się czyni
wprost z~powłoki. Funkcja zwraca status zakończenia uruchomionego programu.

%%%%%%%%%%%%%%%%%%%%%%%%%%%%%%%%%%%%%%%%%%%%%%%%%%%%%%%%%%%%%%%%%%%%%%%%%%%%%
\begin{example}{[Wywołanie programu za pomocą polecenia system()]}
  \label{ex:V86MA}
  \lstinputlisting[style=MyCStyle,label=src:P1A10]{src/lab4/system1.c}
\end{example}
%%%%%%%%%%%%%%%%%%%%%%%%%%%%%%%%%%%%%%%%%%%%%%%%%%%%%%%%%%%%%%%%%%%%%%%%%%%%%

Funkcja \texttt{fork()} tworzy kopię procesu i~uruchamia ją jako proces
potomny. Utworzony proces potomny wykonuje się współbieżnie z procesem
tworzącym, posiada własny nr PID, a jego PPID wskazuje na proces tworzący.
Funkcja \texttt{fork()} tworzy deskryptor nowego procesu oraz kopię segmentu
kodu, danych i stosu. Modyfikacje zmiennych w procesie macierzystym nie są
widoczne w procesie potomnym i odwrotnie. Jeżeli operacja utworzenia procesu
potomnego zakończy się powodzeniem, to funkcja w procesie macierzystym zwraca
identyfikator (PID) procesu potomnego (wartość większa od 1), a w procesie
potomnym wartość 0. Jeżeli próba utworzenia procesu się nie powiedzie, to
funkcja \texttt{fork()} zwraca w procesie macierzystym wartość -1. Działanie
funkcji przedstawiono na rysunku, a jej użycie w przykładzie~\ref{ex:HP9M8}.

%%%%%%%%%%%%%%%%%%%%%%%%%%%%%%%%%%%%%%%%%%%%%%%%%%%%%%%%%%%%%%%%%%%%%%%%%%%%%
\begin{figure}
  \centering
  \begin{tikzpicture}
    \node[TBox6emCentered] (parent)   at (0.00,3.50)  {Proces macierzysty};
    \node                  (fork)     at (0.00,1.75)  {\texttt{fork()}};
    \node                  (x)        at (4.50,1.75)  {};
    \node[TBox6emCentered] (parent2)  at (0.00,0.00)  {Proces macierzysty};
    \node[TBox6emCentered] (child)    at (4.50,0.00)  {Proces potomny};

    \draw[->] (parent.south)  to                                                (fork.north);
    \draw[->] (fork.south)    to node[anchor=center,left]{Zwraca PID potomka}   (parent2.north);
    \draw[->] (fork.east)     to                                                (x.center)
                              to node[anchor=center, right]{Zwraca zero}        (child.north);
  \end{tikzpicture}
  \caption{Idea działania funkcji \texttt{fork()}}
  \label{fig:S278F}
\end{figure}
%%%%%%%%%%%%%%%%%%%%%%%%%%%%%%%%%%%%%%%%%%%%%%%%%%%%%%%%%%%%%%%%%%%%%%%%%%%%%

%%%%%%%%%%%%%%%%%%%%%%%%%%%%%%%%%%%%%%%%%%%%%%%%%%%%%%%%%%%%%%%%%%%%%%%%%%%%%
\begin{example}{[Wywołanie funkcji fork()]}
  \label{ex:HP9M8}
  \lstinputlisting[style=MyCStyle,label=src:Z5EXB]{src/lab4/fork1.c}

  Sprawdzić działanie programu, gdy zmienna CHILD = 8, a zmienna PARENT = 4.
\end{example}
%%%%%%%%%%%%%%%%%%%%%%%%%%%%%%%%%%%%%%%%%%%%%%%%%%%%%%%%%%%%%%%%%%%%%%%%%%%%%

\subsection{Obsługa zakończenia procesów}

Poprawne zakończenie procesów jest zagadnieniem, na które należy zwrócić
szczególną uwagę przy rozwoju wiarygodnego oprogramowania w systemie QNX.
Powodem tego jest fakt, że procesy na ogół współdziałają ze sobą, mogą być np.
procesami macierzystymi, bądź oczekiwać na określone zdarzenia, wygenerowane
przez inne procesy. W związku z tym, przed zakończeniem procesu należy zwolnić
zajęte przez ten proces zasoby, zakończyć z innymi procesami scenariusze
komunikacyjne i synchronizacyjne oraz zaczekać na zakończenie procesów
potomnych.

Zakończenie procesu następuje w następujących przypadkach:
\begin{myitemize}
  \item poprzez wywołanie funkcji exit(),
  \item funkcja main() wywołuje instrukcję return lub zakończyło się działanie
        ostatniej instrukcji kodu,
  \item proces zostanie zakończony przez system operacyjny lub inny proces.
\end{myitemize}

Normalne zakończenie procesu może być zainicjowane przez programistę, poprzez
wywołanie funkcji \texttt{exit()}

%%%%%%%%%%%%%%%%%%%%%%%%%%%%%%%%%%%%%%%%%%%%%%%%%%%%%%%%%%%%%%%%%%%%%%%%%%%%%
\begin{lstlisting}[style=MyCStyle]
 #include <stdlib.h>
 void exit( int status );
\end{lstlisting}
%%%%%%%%%%%%%%%%%%%%%%%%%%%%%%%%%%%%%%%%%%%%%%%%%%%%%%%%%%%%%%%%%%%%%%%%%%%%%

Funkcja ta powoduje zakończenie procesu bieżącego oraz zwrócenie statusu
zakończonego procesu.  Status jest dostępny dla procesu macierzystego. Na ogół
wartość statusu jest ustawiana na \texttt{EXIT\_SUCCESS}, gdy proces zakończył
się poprawnie, bądź na wartość \texttt{EXIT\_FAILURE}, bądź innych kod, gdy
wystąpił błąd.

W przypadku, gdy proces macierzysty uruchamia procesy potomne, należy unikać
sytuacji, gdy proces macierzysty kończy się wcześniej, niż jego procesy potomne
(osierocenie procesów). Proces macierzysty powinien zaczekać na zakończenie
swoich procesów potomnych. Do synchronizacji zakończenia procesów używa się
funkcji:


%%%%%%%%%%%%%%%%%%%%%%%%%%%%%%%%%%%%%%%%%%%%%%%%%%%%%%%%%%%%%%%%%%%%%%%%%%%%%
\begin{lstlisting}[style=MyCStyle]
  #include <sys/types.h>
  #include <sys/wait.h>
  pid_t wait( int * status );
\end{lstlisting}
%%%%%%%%%%%%%%%%%%%%%%%%%%%%%%%%%%%%%%%%%%%%%%%%%%%%%%%%%%%%%%%%%%%%%%%%%%%%%

Zmienna \texttt{status} jest statusem zakończenia procesu potomnego. Funkcja
zwraca PID zakończonego procesu, bądź -1, w przypadku gdy brak procesów
potomnych. \texttt{wait()} blokuje proces wywołujący, do momentu, gdy któryś z procesów
potomnych zakończył się poprawnie (np. wywołał funkcję \texttt{exit()}), bądź
wystąpił błąd.

%%%%%%%%%%%%%%%%%%%%%%%%%%%%%%%%%%%%%%%%%%%%%%%%%%%%%%%%%%%%%%%%%%%%%%%%%%%%%
\begin{figure}
  \centering
  \begin{tikzpicture}
    \node[TBox6emCentered]  (parent)  at (0.00, 3.50) {Proces macierzysty};
    \node                   (fork)    at (0.00, 2.00) {\texttt{fork()}};
    \node                   (wait)    at (0.00, 0.75) {\texttt{wait()}};
    \node[TBox6emCentered]  (child)   at (4.50, 2.00) {Proces potomny};
    \node[circle,draw=black](o)       at (0.00,-0.25) {};
    \node                   (exit)    at (4.50,-0.25) {\texttt{exit()}};
    \node                   (end1)    at (0.00,-1.00) {};
    \node[circle,fill=black](end2)    at (4.50,-1.00) {};
    \node[right of = end2]  (end2t)   {Koniec};

    \draw[->] (parent.south)  to (fork.north);
    \draw[->] (fork.east)     to (child.west);
    \draw[->] (fork.south)    to (wait.north);
    \draw[->] (wait.south)    to node[anchor=center,left]{Oczekiwanie} (o.north);
    \draw[->] (child.south)   to (exit.north);
    \draw[->,dotted] (exit.west)     to (o.east);
    \draw[->] (o.south)       to node[anchor=center,left]{Dalszy bieg} (end1.south);
    \draw[->] (exit.south)    to (end2);
  \end{tikzpicture}
  \caption{Schemat poprawnego zakończenia procesu}
  \label{fig:Z6B8G}
\end{figure}
%%%%%%%%%%%%%%%%%%%%%%%%%%%%%%%%%%%%%%%%%%%%%%%%%%%%%%%%%%%%%%%%%%%%%%%%%%%%%

Ilustrację omawianych zagadnień stanowi niniejszy przykład.

%%%%%%%%%%%%%%%%%%%%%%%%%%%%%%%%%%%%%%%%%%%%%%%%%%%%%%%%%%%%%%%%%%%%%%%%%%%%%
\begin{example}{[Wywołanie funkcji fork() wraz z obsługą zakończenia procesów]}
  \label{ex:11SSB}
  \lstinputlisting[style=MyCStyle,label=src:92W6D]{src/lab4/process1.c}
\end{example}
%%%%%%%%%%%%%%%%%%%%%%%%%%%%%%%%%%%%%%%%%%%%%%%%%%%%%%%%%%%%%%%%%%%%%%%%%%%%%


Zastosowanie funkcji \texttt{wait()} zapewnia, że proces macierzysty nie zakończy się
przed procesem potomnym. W przypadku, gdy proces potomny zakończy się przed
wywołaniem przez macierzysty funkcji \texttt{wait()}, proces potomny zwalnia
wszystkie zasoby, z wyjątkiem deskryptora procesu, czyli miejsca w pamięci
operacyjnej, gdzie przechowywane są wszystkie informacje potrzebne do
zarządzania procesem. Status zakończenia procesu potomnego jest przekazywany do
procesu macierzystego w deskryptorze procesu potomnego. Taki stan procesu
potomnego nazywa się stanem ,,zombie''.

Inną funkcją, pozwalającą na oczekiwanie na zakończenie konkretnego procesu
potomnego jest funkcja \texttt{waitpid()}.

%%%%%%%%%%%%%%%%%%%%%%%%%%%%%%%%%%%%%%%%%%%%%%%%%%%%%%%%%%%%%%%%%%%%%%%%%%%%%
\begin{lstlisting}[style=MyCStyle]
#include <sys/types.h>
#include <sys/wait.h>

pid_t waitpid( pid_t pid, int *status, int options );
\end{lstlisting}
%%%%%%%%%%%%%%%%%%%%%%%%%%%%%%%%%%%%%%%%%%%%%%%%%%%%%%%%%%%%%%%%%%%%%%%%%%%%%

Funkcja zwraca PID zakończonego procesu, bądź -1, w przypadku, gdy brak jest
procesów potomnych. Funkcja zwraca również status zakończonego procesu poprzez
argument \texttt{status}. Jedną z użytecznych opcji funkcji jest flaga
\texttt{WNOHANG}, która powoduje, że proces macierzysty, w przypadku braku
potomnych, natychmiast wraca z funkcji i kontynuuje swoje działanie i w tym
przypadku możemy użyć takiej kombinacji do cyklicznego sprawdzania, czy proces
potomny się zakończył.


\subsection{Zastąpienie procesu bieżącego innymi procesami}

Funkcje z rodziny exec() zastępują bieżący proces, nowym procesem, na podstawie
pliku wykonywalnego, którego nazwa jest parametrem funkcji. W odróżnieniu od
funkcji \texttt{fork()},  w przypadku użytkowania funkcji z rodziny
\texttt{exec()}, kody procesów macierzystych i potomnych mogą być umieszczone w
oddzielnych plikach źródłowych. Taka konfiguracja zapewnia łatwiejsze
uruchamianie i testowanie programów. W systemie QNX zdefiniowano następującą
rodzinę funkcji: \texttt{execl()}, \texttt{execle()}, \texttt{execlp()},
\texttt{execlpe()}, \texttt{execv()}, \texttt{execve()}, \texttt{execvp()},
\texttt{execvpe()}. Działanie tych funkcji jest podobne, natomiast różnią się
listą parametrów formalnych. W trakcie laboratorium będziemy używać
najprostszych funkcji \texttt{execl()} oraz \texttt{execv()}, których sygnatury
wyglądają następująco:

%%%%%%%%%%%%%%%%%%%%%%%%%%%%%%%%%%%%%%%%%%%%%%%%%%%%%%%%%%%%%%%%%%%%%%%%%%%%%
\begin{lstlisting}[style=MyCStyle]
#include <process.h>

int execl(  const char * path,
            const char * arg0,
            const char * arg1,
            ...
            const char * argn,
            NULL );
int execv( const char * path,
           char * const argv[] );
\end{lstlisting}
%%%%%%%%%%%%%%%%%%%%%%%%%%%%%%%%%%%%%%%%%%%%%%%%%%%%%%%%%%%%%%%%%%%%%%%%%%%%%

gdzie \texttt{path} jest ścieżką do pliku wykonywalnego, a argumenty
\texttt{arg[0]} do \texttt{arg[n]} -- nazwą i argumentami programu. Na ostatnim
miejscu podaje się wartość \texttt{NULL}, na oznaczenie zakończenia listy
parametrów wywoływanego programu.

Funkcja \texttt{execv()} pobiera tylko dwa argumenty i daje możliwość
elastycznego budowania listy parametrów. Drugi z argumentów jest tablicą
napisów, która musi być zakończona pustym wskaźnikiem \texttt{((char *)0)}.
Wartość pola \texttt{argv[0]} musi wskazywać na nazwę procesu, który chcemy
uruchomić.

%%%%%%%%%%%%%%%%%%%%%%%%%%%%%%%%%%%%%%%%%%%%%%%%%%%%%%%%%%%%%%%%%%%%%%%%%%%%%
\begin{example}{[Wywołanie funkcji execl()]}
  \label{ex:WHF3V}
  \lstinputlisting[style=MyCStyle,label=src:5Y0UB]{src/lab4/execv1.c}
\end{example}
%%%%%%%%%%%%%%%%%%%%%%%%%%%%%%%%%%%%%%%%%%%%%%%%%%%%%%%%%%%%%%%%%%%%%%%%%%%%%

\subsection{Tworzenie procesów funkcją spawn()}

Posługiwanie się funkcjami \texttt{fork()} i \texttt{exec()} do tworzenia
procesów współbieżnych bywa niewygodne. System operacyjny QNX udostępnia
rodzinę \texttt{spawn()}, która służy do tworzenia procesów potomnych na
podstawie pliku wykonywalnego, którego nazwa jest jednym z argumentów funkcji.
Do tej grupy zaliczamy następujące funkcje: \texttt{spawn()},
\texttt{spawnl()}, \texttt{spawnv()}, \texttt{spawnle()}, \texttt{spawnlp()},
\texttt{spawnlpe()}, \texttt{spawnve()}, \texttt{spawnvp()},
\texttt{spawnvpe()}. Ponownie omówimy tylko dwie wybrane funkcje
\texttt{spawnl()} i~\texttt{spawnv()}, których deklaracje wyglądają
następująco:

%%%%%%%%%%%%%%%%%%%%%%%%%%%%%%%%%%%%%%%%%%%%%%%%%%%%%%%%%%%%%%%%%%%%%%%%%%%%%
\begin{lstlisting}[style=MyCStyle]
#include <process.h>

int spawnl( int mode,
            const char * path,
            const char * arg0,
            const char * arg1,
            ...
            const char * argn,
            NULL );
int spawnv( int mode,
            const char * path,
            char * const argv[] );
\end{lstlisting}
%%%%%%%%%%%%%%%%%%%%%%%%%%%%%%%%%%%%%%%%%%%%%%%%%%%%%%%%%%%%%%%%%%%%%%%%%%%%%
gdzie path jest ścieżką do pliku wykonywalnego, argumenty \texttt{arg[0]} do
\texttt{arg[n]} stanowią nazwę i argumenty programu. Ostatnim parametrem
funkcji jest string \texttt{NULL}, który służy do zakończenia listy parametrów
wywoływanego programu. Parametr mode jest trybem wykonania procesu. Tryb ten
mówi o sposobie wywołania procesu potomnego i metodzie zachowania procesu
macierzystego, gdy proces potomny zostanie zainicjowany.

Funkcja \texttt{spawnv()} pobiera trzy argumenty i daje możliwość elastycznego
budowania listy parametrów. Trzeci argument jest tablicą napisów, która musi być
zakończona pustym wskaźnikiem \texttt{((char *)0)}. Wartość pola
\texttt{argv[0]} musi wskazywać na nazwę procesu, który chcemy uruchomić.

%%%%%%%%%%%%%%%%%%%%%%%%%%%%%%%%%%%%%%%%%%%%%%%%%%%%%%%%%%%%%%%%%%%%%%%%%%%%%
\begin{table}
  \caption{Tryby wywołania funkcji \texttt{spawn()}}
  \label{tab:V0CUU}
  \begin{tabular}{|l|p{0.75\textwidth}|}
    \hline
    \textbf{Tryb}       & \textbf{Opis}
    \\ \hline
    \texttt{P\_WAIT}    & Proces macierzysty czeka na zakończenie procesu
                          potomnego i później jest kontynuowany.
    \\ \hline
    \texttt{P\_NOWAIT}  & Proces macierzysty i potomny są wykonywane
                          współbieżnie. Można używać funkcji \texttt{wait()}.
    \\ \hline
    \texttt{P\_NOWAITO} & Proces macierzysty i potomny są wykonywane
                          współbieżnie. Nie wolno używać funkcji wait() do
                          uzyskania kodu wyjścia. Relacja pokrewieństwa między
                          nimi zostaje przerwana.
    \\ \hline
    \texttt{P\_OVERLAY} & Proces macierzysty jest zastępowany przez proces
                          potomny. Wywołanie w tym trybie jest równoważne
                          wywołaniu funkcji execl().
    \\ \hline
  \end{tabular}
\end{table}
%%%%%%%%%%%%%%%%%%%%%%%%%%%%%%%%%%%%%%%%%%%%%%%%%%%%%%%%%%%%%%%%%%%%%%%%%%%%%

Ilustrację wywołania funkcji stanowi przykład:

%%%%%%%%%%%%%%%%%%%%%%%%%%%%%%%%%%%%%%%%%%%%%%%%%%%%%%%%%%%%%%%%%%%%%%%%%%%%%
\begin{example}{[Wywołanie funkcji spawnl()]}
  \label{ex:VSCD0}
  Utworzyć  dwa programy o nazwie macierzysty i potomny, następnie wywołać
  program macierzysty.
  \lstinputlisting[style=MyCStyle,label=src:4HNG2]{src/lab4/parent1.c}
  \lstinputlisting[style=MyCStyle,label=src:YZY1R]{src/lab4/child1.c}
\end{example}
%%%%%%%%%%%%%%%%%%%%%%%%%%%%%%%%%%%%%%%%%%%%%%%%%%%%%%%%%%%%%%%%%%%%%%%%%%%%%


\subsection{Ćwiczenia}

\begin{myenumerate}
  \item Napisać program, który tworzy dwa procesy. Każdy z procesów powinien
        utworzyć proces potomny. Wyświetlić identyfikatory procesów rodziców
        i~potomków.
  \item Napisać program, który tworzy proces macierzysty P1 i  potomny P2.
        Proces P2 uruchamia inny proces P3 za pomocą funkcji \texttt{execv()}.
        Parametry wywołania procesu P3 są przekazywane z wiersza poleceń.
  \item Utworzyć proces macierzysty mac, który stworzy za pomocą funkcji
        \texttt{fork()} 5 procesów potomnych oraz zaczekać na ich zakończenie.
        Niech każdy z procesów wyświetla co jedną sekundę swój numer PID, numer
        identyfikacyjny (w zależności od kolejności utworzenia) oraz czas
        działania. Czas działania poszczególnych procesów podawać jako
        argumenty do programu głównego (np.  mac 20 5 3 8 9). Na zakończenie
        procesu potomnego o nr i wywołać funkcję \texttt{exit(i)}. Proces
        macierzysty powinien czekać na zakończenie potomnych i wyświetlić
        informację, który z procesów potomnych się zakończył.
        %%%%%%%%%%%%%%%%%%%%%%%%%%%%%%%%%%%%%%%%%%%%%%%%%%%%%%%%%%%%%%%%%%%%%
        \begin{figure}[!h]
          \centering
          \begin{tikzpicture}
            \node[circle,fill=black] (o)  at ( 0.00, 0.00) {};
            \node[circle,fill=black] (c1) at (-1.00,-1.00) {};
            \node[circle,fill=black] (c2) at (-0.50,-1.00) {};
            \node[circle,fill=black] (c3) at (-0.00,-1.00) {};
            \node[circle,fill=black] (c4) at ( 0.50,-1.00) {};
            \node[circle,fill=black] (c5) at ( 1.00,-1.00) {};
            \node[] at (3.00,0.00) {Proces macierzysty};
            \node[] at (3.00,-1.00) {Procesy potomne};

            \draw[-]     (o) to (c1);
            \draw[-]     (o) to (c2);
            \draw[-]     (o) to (c3);
            \draw[-]     (o) to (c4);
            \draw[-]     (o) to (c5);
          \end{tikzpicture}
          \caption{Struktura 1-poziomowa}
          \label{fig:YT6VA}
        \end{figure}
        %%%%%%%%%%%%%%%%%%%%%%%%%%%%%%%%%%%%%%%%%%%%%%%%%%%%%%%%%%%%%%%%%%%%%

  \item Zadanie jest analogiczne do poprzedniego, z tym, że struktura procesów
        ma wyglądać jak drzewo przedstawione na rysunku \ref{fig:XDQ8Q}.
        %%%%%%%%%%%%%%%%%%%%%%%%%%%%%%%%%%%%%%%%%%%%%%%%%%%%%%%%%%%%%%%%%%%%%
        \begin{figure}[!h]
          \centering
          \begin{tikzpicture}
            \node[circle,fill=black] (c1) at ( 0.00, 2.25) {};
            \node[circle,fill=black] (c2) at ( 0.00, 1.50) {};
            \node[circle,fill=black] (c3) at ( 0.00, 0.75) {};
            \node[circle,fill=black] (c4) at ( 0.00, 0.00) {};
            \node[] at (0.00,2.75) {Proces macierzysty};

            \draw[-]     (c1) to (c2) to (c3) to (c4);
          \end{tikzpicture}
          \caption{Struktura 3-poziomowa}
          \label{fig:XDQ8Q}
        \end{figure}
        %%%%%%%%%%%%%%%%%%%%%%%%%%%%%%%%%%%%%%%%%%%%%%%%%%%%%%%%%%%%%%%%%%%%%
        Każdy z procesów, oprócz ostatniego tworzy jeden proces potomny.
\end{myenumerate}


\cleardoublepage

%% \label{??:7HOCC}
%% \label{??:3UDS1}
%% \label{??:BYA03}
%% \label{??:LDB7J}
%% \label{??:HXCTX}
%% \label{??:V35FU}
%% \label{??:V285P}
%% \label{??:0DR40}
%% \label{??:TFLMK}
%% \label{??:OD9EZ}
%% \label{??:87ILN}
%% \label{??:HY94T}
%% \label{??:GIWGJ}
%% \label{??:S0ENF}
%% \label{??:5O33M}
%% \label{??:9LCEW}
%% \label{??:FMIVT}
%% \label{??:8O5ZW}
%% \label{??:5GR5X}
%% \label{??:3KG12}
%% \label{??:O26C4}
%% \label{??:QXXWZ}
%% \label{??:FKPAJ}
%% \label{??:IUWYS}
%% \label{??:AKF4R}
%% \label{??:T7RU4}
%% \label{??:1QZGH}


\section{Zarządzanie wątkami – tworzenie, kończenie, atrybuty wątków}

\subsection{Wprowadzenie}

Podczas rozwoju oprogramowania, np. czasu rzeczywistego, wbudowanego, graficznego, często zachodzi potrzeba przetwarzania współbieżnego, bądź równoległego, w przypadku implementacji na komputerach równoległych z~pamięcią wspólną. Współbieżność (równoległość) tę można osiągnąć poprzez używanie wątków POSIX (biblioteka \lstinline[style=MyBashStyle]{pthread}), zaimplementowanych w~systemie QNX i~opartych na standardzie \lstinline[style=MyBashStyle]{IEEE POSIX 1003.1c}.

Dotychczas zajmowaliśmy się procesami mającymi po jednym wątku. Pojęcie procesu można rozszerzyć do istnienia wielu wątków sterowania i~rozpatrywać proces jako zbiór wątków i zasobów. Wątek w~tym przypadku będzie traktowany jako elementarna, niezależna jednostka, podlegająca szeregowaniu (scheduling). Część zasobów procesu jest wspólna dla jego wszystkich wątków. Należą do nich:

\begin{myitemize}
\item Wspólna przestrzeń adresowa, w~szczególności zmienne statyczne i sterta.
\item Pliki i~urządzenia wejścia-wyjścia.
\item Kanały, kolejki i połączenia.
\end{myitemize}

Wątki mają również swoje prywatne atrybuty i~zasoby. Spośród ważniejszych można wymienić:

\begin{myitemize}
\item Identyfikator wątku TID (thread identifier).
\item Priorytet.
\item Stos.
\item Zestaw rejestrów.
\item Atrybuty służące szeregowaniu wątków.
\item Maska sygnałów.
\item Lokalne zmienne wątku TLS (thread local storage).
\item Procedura zakończenia wątku.
\end{myitemize}

Zalety stosowania wielowątkowego modelu programowania (multithreaded programming model):

\begin{myitemize}
\item Mniejszy koszt tworzenia, kończenia, w~porównaniu z~procesami.
\item Zwykle szybszy czas przełączenia wątków niż procesów.
\item Wszystkie wątki dzielą tę samą przestrzeń adresową. W~wielu przypadkach, komunikacja między wątkami jest łatwiejsza i~wydajniejsza, niż komunikacja międzyprocesorowa.
\item Korzyści wydajnościowe, w~przypadku przetwarzania równoległego na maszynach z pamięcią wspólną SMP (symmetric multiprocessing) - cecha szczególnie istotna, ze względu na fakt istnienia relatywnie tanich procesorów wielordzeniowych (multicore processors).
\end{myitemize}

Biblioteka pthread zawiera wiele funkcji, które umożliwiają zarządzanie wątkami. Ich prototypy zostały zdefiniowane w pliku nagłówkowym \lstinline[style=MyBashStyle]{pthread.h}.  Procedury można podzielić zgrubnie na trzy grupy:

\begin{myitemize}
\item Zarządzanie wątkami – tworzenie, anulowanie, odłączanie, dołączanie wątków, operowanie na atrybutach wątków.
\item Mechanizmy synchronizacji wątków (np. muteksy, zmienne warunkowe, bariery).
\item Mechanizmy komunikacji wątków.
\end{myitemize}

Biblioteka \lstinline[style=MyBashStyle]{pthread} zawiera ponad 60 procedur, zdefiniowanych dla języka~C. W~trakcie niniejszego laboratorium skupimy się na zbiorze funkcji, które są na ogół najczęściej używane i~będą użyteczne dla początkującego programisty wątków POSIX.

\subsection{Zarządzanie wątkami}
\subsubsection{Tworzenie wątków}

Funkcja \lstinline[style=MyCStyle]{main()} posiada jeden wątek (domyślny). Dodatkowe wątki w~obrębie procesu muszą być jawnie utworzone przez programistę. Funkcja \lstinline[style=MyCStyle]{pthread_create()} służy utworzeniu nowego wątku. Prototyp funkcji wygląda następująco:


\begin{lstlisting}[style=MyCStyle]
int pthread_create( pthread_t* thread,
		const pthread_attr_t* attr,
		void* (*start_routine)(void* ),
		void* arg );
\end{lstlisting}

gdzie

\begin{myitemize}
\item \lstinline[style=MyCStyle]{thread} jest unikalnym identyfikatorem wątku (TID), nadawanym przez system operacyjny.
\item \lstinline[style=MyCStyle]{attr} - atrybuty utworzonego wątku; \lstinline[style=MyCStyle]{NULL}, jeśli przyjęte są wartości domyślne.
\item \lstinline[style=MyCStyle]{start_routine} - funkcja, która będzie wykonywana przez utworzony wątek, o~sygnaturze w~prototypie.
\item \lstinline[style=MyCStyle]{arg} - argument przekazywany jako parametr do wątku (typu \lstinline[style=MyCStyle]{void*}); bądź \lstinline[style=MyCStyle]{NULL}, jeśli nie przekazujemy żadnego argumentu.
\end{myitemize}

Funkcja zwraca \lstinline[style=MyCStyle]{0}, gdy sukces i \lstinline[style=MyCStyle]{-1}, gdy wystąpił błąd. Maksymalna liczba możliwych do utworzenia wątków jest zależna od implementacji biblioteki. Utworzone wątki mogą tworzyć nowe wątki, bez ograniczeń, związanych z hierarchią wątków.

Do pobierania identyfikatora wątku służy funkcja:

\begin{lstlisting}[style=MyCStyle]
pthread_t pthread_self( void )
\end{lstlisting}

Obie funkcje \lstinline[style=MyCStyle]{pthread_create}, \lstinline[style=MyCStyle]{pthread_self} użyjemy w poniższym przykładzie.

\begin{example}{[Utworzenie wątku z~pobraniem identyfikatora]}
Należy skompilować, zbudować i~uruchomić poniższy kod.

\lstinputlisting[caption=Utworzenie wątku z~pobraniem identyfikatora,style=MyCStyle,label=src:firstThread]{src/lab5/firstThread.c}

Po uruchomieniu programu, wątek główny i~wątek utworzony wykonują współbieżnie (równolegle) swoje zadania, wyświetlając nr aktualnie wykonywanego wątku i~nr kroku. W~trakcie wykonywania programu pobrać informację o identyfikatorach wątków za pomocą polecenia:

\begin{lstlisting}[style=MyCStyle]
# pidin -p watek
\end{lstlisting}

gdzie \lstinline[style=MyCStyle]{watek} jest nazwą wykonywanego procesu. Wynik działania programu powinien wyglądać następująco:

\begin{lstlisting}[style=MyCStyle]
     pid tid name               prio STATE       Blocked
  237594   1 ./watek             10r NANOSLEEP
  237594   2 ./watek             10r NANOSLEEP
\end{lstlisting}

\end{example}


\subsubsection{Kończenie wątków}

Wątek może być zakończony w~następujący sposób:

\begin{myitemize}
\item Następuje powrót z funkcji wykonywanej przez wątek.
\item Wątek wykonuje funkcję \lstinline[style=MyCStyle]{pthread_exit()}.
\item Wątek jest anulowany przez inny wątek, za pomocą funkcji \lstinline[style=MyCStyle]{pthread_cancel()}.
\item Następuje zakończenie procesu macierzystego, poprzez wykonanie np. funkcji \lstinline[style=MyCStyle]{exec}, bądź \lstinline[style=MyCStyle]{exit}.
\end{myitemize}

Aby jawnie zakończyć  wątek, używamy funkcji:

\begin{lstlisting}[style=MyCStyle]
void pthread_exit( void* value_ptr );
\end{lstlisting}

gdzie wartość \lstinline[style=MyCStyle]{value_ptr} jest kodem powrotu wątku. W~przypadku, gdy funkcja \lstinline[style=MyCStyle]{main()} zakończy swoje wykonywanie wywołaniem \lstinline[style=MyCStyle]{pthread_exit()}, to wykonywanie wątków będzie kontynuowane do ich zakończenia, w~przeciwnym przypadku, wątki będą automatycznie zakończone, wraz z~zakończeniem procesu macierzystego. Gdy wątek jest dołączalny (atrybut \lstinline[style=MyCStyle]{PTHREAD_CREATE_JOINABLE} ustawiony), to wraz z~wywołaniem funkcji \lstinline[style=MyCStyle]{pthread_exit()} wątek kończy swoje działanie, ale zwalnia zasoby dopiero po wywołaniu funkcji \lstinline[style=MyCStyle]{pthread_join()} przez inny wątek. Gdy atrybut jest niedołączalny, to zwalnia wszystkie zasoby, tuż po wywołaniu funkcji  \lstinline[style=MyCStyle]{pthread_exit()}.


\begin{example}{[Utworzenie i~zakończenie wątku wraz z~przekazaniem argumentów]}
Należy przeanalizować i~przetestować działanie programu.

\lstinputlisting[caption=Utworzenie i zakończenie wątku wraz z przekazaniem argumentów,style=MyCStyle,label=src:secondThread]{src/lab5/secondThread.c}


Wyświetlić informacje o wykonywanych wątkach poleceniem:

\begin{lstlisting}[style=MyCStyle]
# pidin -p watek
     pid tid name               prio STATE       Blocked
  282650   1 ./watek             10r DEAD
  282650   2 ./watek             10r NANOSLEEP
  282650   3 ./watek             10r NANOSLEEP
  282650   4 ./watek             10r NANOSLEEP
  282650   5 ./watek             10r NANOSLEEP
\end{lstlisting}

Pomimo, iż proces macierzysty (wątek główny) się zakończył, to utworzone wątki wciąż pracują, co zapewnia funkcja  \lstinline[style=MyCStyle]{pthread_exit()}.
\end{example}

\subsubsection{Łączenie wątków}

Funkcja \lstinline[style=MyCStyle]{pthread_exit()} powoduje zakończenie wątku, który ją wywołał, jednak nie zwalnia automatycznie zasobów zajętych przez wątek, jak deskryptory plików, muteksy, itd. W~przypadku wątków dołączalnych, zasoby te są zwalniane w momencie dołączenia wątku bieżącego do innego wątku poprzez wywołanie funkcji:

\begin{lstlisting}[style=MyCStyle]
int pthread_join( pthread_t thread, void** value_ptr );
\end{lstlisting}

gdzie \lstinline[style=MyCStyle]{thread} jest numerem wątku, na którego zakończenie czekamy, a \lstinline[style=MyCStyle]{value_ptr} kodem powrotu, zwracanym przez zakończony wątek (wartość przekazana do \lstinline[style=MyCStyle]{pthread_exit()} lub \lstinline[style=MyCStyle]{PHTREAD_CANCELED}, jeśli wątek został anulowany), bądź \lstinline[style=MyCStyle]{NULL}. Funkcja zwraca \lstinline[style=MyCStyle]{0}, gdy sukces i~\lstinline[style=MyCStyle]{-1}, jeśli wystąpi błąd. Funkcja \lstinline[style=MyCStyle]{pthread_join()} blokuje wątek, który wywołał funkcję, do momentu, gdy wątek o~identyfikatorze \lstinline[style=MyCStyle]{thread} zakończy swoje działanie. Gdy wskazany wątek już się zakończył, wątek bieżący nie jest blokowany i~odbiera status w~funkcji \lstinline[style=MyCStyle]{pthread_join()}. Sytuacje te przedstawiają rysunki~\ref{fig:laczenie1} oraz~\ref{fig:laczenie2}.

\begin{figure}[!h]
\centering
\includegraphics[width=0.85\textwidth]{img/laczenie1}
\caption{Łączenie wątków. Wątek 1 czeka na wątek 2}
\label{fig:laczenie1}
\end{figure}
\begin{figure}[!h]
\centering
\includegraphics[width=0.85\textwidth]{img/laczenie2}
\caption{Łączenie wątków. Wątek 2 czeka na wątek 1}
\label{fig:laczenie2}
\end{figure}

\begin{example}{[Łączenie wątków]}
Przykład ilustruje omawiane zagadnienia łączenia wątków, wraz z~przekazaniem kodu powrotu z~utworzonych wątków do wątku głównego.

\lstinputlisting[caption=Łączenie wątków,style=MyCStyle,label=src:thirdThread]{src/lab5/thirdThread.c}

\end{example}


\subsubsection{Atrybuty wątków}

Biblioteka \lstinline[style=MyCStyle]{pthread} dostarcza mechanizmu do regulowania własności wątków. Atrybuty wątków są przekazywane jako parametry w~strukturze \lstinline[style=MyCStyle]{pthread_attr_t* attr} funkcji \lstinline[style=MyCStyle]{pthread_create()}, w~chwili tworzenia wątku. W~przypadku, gdy przekazujemy wskaźnik \lstinline[style=MyCStyle]{NULL}, atrybuty są ustawiane domyślnie. Jeśli chcemy utworzyć wątek, z~atrybutami innymi, niż domyślne, jawnie dobranymi przez użytkownika należy:


\begin{myitemize}
\item Utworzyć obiekt \lstinline[style=MyCStyle]{attr} typu \lstinline[style=MyCStyle]{pthread_attr_t}.
\item Zainicjować zmienną \lstinline[style=MyCStyle]{attr} wywołaniem funkcji \lstinline[style=MyCStyle]{pthread_attr_init(&attr)}.
\item Zmodyfikować obiekt z~atrybutami, tak, aby zawierał pożądane wartości.
\item Przekazać wskaźnik do struktury \lstinline[style=MyCStyle]{attr}, w~trakcie tworzenia wątku za pomocą \lstinline[style=MyCStyle]{pthread_create()}.
\item Zwolnić zasoby wykorzystywane przez atrybut, przez wywołanie funkcji \mbox{\lstinline[style=MyCStyle]{pthread_attr_destroy(&attr)}}.
\end{myitemize}

Wybrane atrybuty wątku przedstawiono w tabeli~\ref{tab:atrybuty1}, a~funkcje służące do ich ustawiania w~tabeli~\ref{tab:atrybuty2}.

\begin{table}[h!]
\centering
\caption{Wybrane atrybuty wątku i ich wartości domyślne}
\setlength{\arrayrulewidth}{1pt}
\setlength{\tabcolsep}{6pt}
\renewcommand{\arraystretch}{1.2}
\begin{tabular}{ |p{0.45\textwidth}|p{0.4\textwidth}| }
\hline \rowcolor{gray}
\textbf{Atrybut} & \textbf{Wartość domyślna} \\ \hline
Dołączalność (\mbox{\lstinline[style=MyCStyle]{detachstate}}) & \mbox{\lstinline[style=MyCStyle]{PTHREAD_CREATE_JOINABLE}} \\ \hline
Dziedziczenie atrybutów (\mbox{\lstinline[style=MyCStyle]{inherit-scheduling}}) & \mbox{\lstinline[style=MyCStyle]{PTHREAD_INHERIT_SCHED}} \\ \hline
Strategia szeregowania (\mbox{\lstinline[style=MyCStyle]{schedpolicy}}) & \mbox{\lstinline[style=MyCStyle]{PTHREAD_INHERIT_SCHED}} \\ \hline
Parametry szeregowania (\mbox{\lstinline[style=MyCStyle]{schedparam}}) & Dziedziczone z procesu macierzystego \\ \hline
Rywalizacja wątku o zasoby (\mbox{\lstinline[style=MyCStyle]{contentionscope}}) & \mbox{\lstinline[style=MyCStyle]{PTHREAD_SCOPE_SYSTEM}} \\ \hline
Rozmiar stosu (\mbox{\lstinline[style=MyCStyle]{stacksize}}) & 4kB \\ \hline
Adres stosu (\mbox{\lstinline[style=MyCStyle]{stackaddr}}) & \mbox{\lstinline[style=MyCStyle]{NULL}} \\ \hline
\end{tabular}
\label{tab:atrybuty1}
\end{table}


Objaśnienia:

\begin{myitemize}
\item Dołączalność - informacja, czy wątek ma zwolnić zasoby natychmiast (\lstinline[style=MyCStyle]{PTHREAD_CREATE_DETACHED}), czy po wywołaniu przez proces macierzysty funkcji \lstinline[style=MyCStyle]{pthread_join (PTHREAD_CREATE_JOINABLE)}.
\item Dziedziczenie atrybutów - domyślnie są dziedziczone z~wątku macierzystego.
\item Strategia szeregowania: \lstinline[style=MyCStyle]{SCHED_FIFO}, \lstinline[style=MyCStyle]{SCHED_RR}, \lstinline[style=MyCStyle]{SCHED_SPORADIC}, \lstinline[style=MyCStyle]{SCHED_OTHER} - szczegółowe objaśnienia w~dokumentacji.
\item Parametry szeregowania - informacje, dot. szeregowania wątków. Modyfikowana jest struktura \lstinline[style=MyCStyle]{const struct sched_param * param}.
\item Rywalizacja wątku o~zasoby - wątki mogą rywalizować o~zasoby z~wątkami systemowymi z~innych procesów (ang. system contention scope).  Wątki mogą również rywalizować o~zasoby tylko w~obrębie procesu, który je utworzył (process contention scope). Domyślnie ustawiana jest pierwsza opcja.
\item Rozmiar stosu - informacja o~rozmiarze stosu do przechowywania zmiennych lokalnych.
\item Adres stosu - gdy adres stosu jest ustawiony na \lstinline[style=MyCStyle]{NULL}, to będzie ustalany i~zwalniany automatycznie przez system operacyjny. Pamięć na stos może być też przydzielona przez programistę i~wtedy jest on odpowiedzialny za jej zwolnienie.
\end{myitemize}

Po inicjalizacji struktury z atrybutami, możemy pobierać i ustawiać atrybuty związane z wątkami.

\begin{table}[h!]
\centering
\caption{Funkcje od pobierania i ustawiania atrybutów}
\setlength{\arrayrulewidth}{1pt}
\setlength{\tabcolsep}{6pt}
\renewcommand{\arraystretch}{1.2}
\begin{tabular}{ |p{0.25\textwidth}|p{0.35\textwidth}|p{0.35\textwidth}| }
\hline \rowcolor{gray}
\textbf{Atrybut} & \textbf{Pobierz atrybut} & \textbf{Ustaw atrybut} \\ \hline
Dołączalność & \mbox{\lstinline[style=MyCStyle]{pthread_attr_getdetachstate()}} & \mbox{\lstinline[style=MyCStyle]{pthread_attr_setdetachstate()}} \\ \hline
Dziedziczenie atrybutów & \mbox{\lstinline[style=MyCStyle]{pthread_attr_getinheritsched()}} & \mbox{\lstinline[style=MyCStyle]{pthread_attr_setinheritsched()}} \\ \hline
Strategia szeregowania & \mbox{\lstinline[style=MyCStyle]{pthread_attr_getschedpolicy()}} & \mbox{\lstinline[style=MyCStyle]{pthread_attr_setschedpolicy()}} \\ \hline
Parametry szeregowania & \mbox{\lstinline[style=MyCStyle]{pthread_attr_getschedparam()}} & \mbox{\lstinline[style=MyCStyle]{pthread_attr_setschedparam()}} \\ \hline
Rywalizacja o~zasoby & \mbox{\lstinline[style=MyCStyle]{pthread_attr_getscope()}} & \mbox{\lstinline[style=MyCStyle]{pthread_attr_setscope()}} \\ \hline
Rozmiar stosu & \mbox{\lstinline[style=MyCStyle]{pthread_attr_getstacksize()}} & \mbox{\lstinline[style=MyCStyle]{pthread_attr_setstacksize()}} \\ \hline
Adres stosu & \mbox{\lstinline[style=MyCStyle]{pthread_attr_getstackaddr()}} & \mbox{\lstinline[style=MyCStyle]{pthread_attr_setstackaddr()}} \\ \hline
\end{tabular}
\label{tab:atrybuty2}
\end{table}


\begin{example}{[Atrybuty wątków]}
Przykład ilustruje omawiane zagadnienia łączenia wątków, wraz z~przekazaniem kodu powrotu z~utworzonych wątków do wątku głównego.

\lstinputlisting[caption=Ustawianie atrybutu dołączalności,style=MyCStyle,label=src:attThread]{src/lab5/attThread.c}
\end{example}

\subsubsection{Ustalanie priorytetu, strategii i parametrów szeregowania wątków}


\underline{Dziedziczenie atrybutów}


\noindent\textbf{Ustawianie atrybutów}. Wątki potomne domyślnie dziedziczą priorytet i~własności dotyczące szeregowania z~wątku macierzystego. Funkcja \lstinline[style=MyCStyle]{pthread_attr_setinheritsched} umożliwia zmianę tego stanu:

\begin{lstlisting}[style=MyCStyle]
int pthread_attr_setinheritsched(pthread_attr_t * attr, int inheritsched );
\end{lstlisting}

gdzie \lstinline[style=MyCStyle]{attr} jest wskaźnikiem na strukturę definiującą atrybuty wątku, a~wartość \lstinline[style=MyCStyle]{inheritsched} jest jedną z~dwóch wartości:

\begin{myitemize}
\item \lstinline[style=MyCStyle]{PTHREAD_INHERIT_SCHED} - wątek potomny dziedziczy strategię szeregowania od wątku rodzica - wartość domyślna.
\item \lstinline[style=MyCStyle]{PTHREAD_EXPLICIT_SCHED} - strategia szeregowania jest ustawiona przez strukturę \lstinline[style=MyCStyle]{attr}.
\end{myitemize}

\noindent\textbf{Pobieranie atrybutów}. Funkcją, która umożliwia pobieranie parametrów dot. dziedziczenia własności z~wątku macierzystego ma następującą sygnaturę:

\begin{lstlisting}[style=MyCStyle]
int pthread_attr_getinheritsched(const pthread_attr_t* attr,int* inheritsched );
\end{lstlisting}

gdzie wartość \lstinline[style=MyCStyle]{inheritsched} jest wskaźnikiem do miejsca, gdzie funkcja przechowa atrybut.

\noindent\underline{Strategia szeregowania}

\noindent\textbf{Ustawianie atrybutów}. Ustawianie strategii szeregowania realizujemy funkcją:

\begin{lstlisting}[style=MyCStyle]
int pthread_attr_setschedpolicy(pthread_attr_t* attr, int policy);
\end{lstlisting}

gdzie \lstinline[style=MyCStyle]{attr} jest wskaźnikiem na strukturę definiującą atrybuty wątku, a~wartość \lstinline[style=MyCStyle]{policy} jest jedną z~wartości zdefiniowanych w~tabeli~\ref{tab:sched}.

\begin{table}[h!]
\centering
\caption{Strategie szeregowania w QNX Neutrino}
\setlength{\arrayrulewidth}{1pt}
\setlength{\tabcolsep}{6pt}
\renewcommand{\arraystretch}{1.2}
\begin{tabular}{ |p{0.15\textwidth}|p{0.2\textwidth}|p{0.4\textwidth}| }
\hline \rowcolor{gray}
\textbf{Numer} & \textbf{Symbol} & \textbf{Opis} \\ \hline
\mbox{\lstinline[style=MyCStyle]{0}} & \mbox{\lstinline[style=MyCStyle]{SCHED_NOCHANGE}} & Brak zmiany strategii szeregowania \\ \hline
\mbox{\lstinline[style=MyCStyle]{1}} & \mbox{\lstinline[style=MyCStyle]{SCHED_FIFO}} & Szeregowanie FIFO \\ \hline
\mbox{\lstinline[style=MyCStyle]{2}} & \mbox{\lstinline[style=MyCStyle]{SCHED_RR}} & Szeregowanie karuzelowe \\ \hline
\mbox{\lstinline[style=MyCStyle]{3}} & \mbox{\lstinline[style=MyCStyle]{SCHED_OTHER}} & wskazuje wartość \mbox{\lstinline[style=MyCStyle]{SCHED_RR}} \\ \hline
\mbox{\lstinline[style=MyCStyle]{4}} & \mbox{\lstinline[style=MyCStyle]{SCHED_SPORADIC}} & Szeregowanie sporadyczne \\ \hline
\end{tabular}
\label{tab:sched}
\end{table}

\noindent\textbf{Pobieranie atrybutów}. Pobieranie strategii szeregowania można zrealizować funkcją:

\begin{lstlisting}[style=MyCStyle]
int pthread_attr_getschedpolicy(const pthread_attr_t* attr,int* policy );
\end{lstlisting}

gdzie \lstinline[style=MyCStyle]{policy} jest wskaźnikiem do miejsca przechowywania ustawionej wartości strategii szeregowania.

\noindent\underline{Parametry szeregowania}

\noindent\textbf{Ustawianie atrybutów}. Aby ustawić atrybuty z~tego zbioru należy zastosować funkcję:

\begin{lstlisting}[style=MyCStyle]
int pthread_attr_setschedparam(pthread_attr_t * attr,const struct sched_param * param );
\end{lstlisting}

gdzie \lstinline[style=MyCStyle]{attr} jest wskaźnikiem na strukturę definiującą atrybuty wątku, a~zmienna \lstinline[style=MyCStyle]{param} jest wskaźnikiem na strukturę \lstinline[style=MyCStyle]{sched_param}, która definiuje parametry szeregowania watków.

\noindent\textbf{Pobieranie atrybutów}. Tę funkcję realizujemy następująco:

\begin{lstlisting}[style=MyCStyle]
int pthread_attr_getschedparam(const pthread_attr_t * attr, struct sched_param * param );
\end{lstlisting}

gdzie \lstinline[style=MyCStyle]{param} jest wskaźnikiem na strukturę \lstinline[style=MyCStyle]{sched_param}.

\begin{example}{[Zarządzanie atrybutami szeregowania wątków]}
Przykład pokazuje w~jaki sposób używać atrybutów, w~celu utworzenia wątku ze zdefiniowanymi przez programistę parametrami i~strategią szeregowania.

\lstinputlisting[caption=Pobieranie i ustawianie atrybutów szeregowania wątków,style=MyCStyle,label=src:sched]{src/lab5/sched.c}

Przykładowy wynik działania programu powinien wyglądać następująco:

\begin{lstlisting}[style=MyCStyle]
Priorytet ustawiony przy starcie procesu (watku glownego) 10.
Domyslna strategia ROUND ROBIN, priorytet: 10
Domyslna strategia FIFO, priorytet: 8
Tworze watek z nowymi atrybutami:
	Step = 0
	Step = 1
	Step = 2
	Step = 3
	Step = 4
Watek glowny dolaczyl watek potomny.
Watek glowny zakonczony. Wyjscie.
\end{lstlisting}

\end{example}


\subsection{Ćwiczenia}

\begin{myenumerate}
\item Napisać program, który tworzy $8$ wątków i~przekazuje do niego strukturę następującego typu:

\begin{lstlisting}[style=MyCStyle]
struct thread_data
  {
    int	cnt;          /* numer watku */
    int sum;          /* suma kontrolna */
    const char *msg;  /* wiadomosc */
  }
\end{lstlisting}

gdzie

\begin{myitemize}
  \item \lstinline[style=MyCStyle]{cnt} - numer wątku (nadajemy sami wg kolejności tworzenia: \lstinline[style=MyCStyle]{cnt=0,1,2,...}).
\item \lstinline[style=MyCStyle]{sum} - ,,suma kontrolna'', będąca sumą numerów wątków (np. dla pierwszego wątku \mbox{\lstinline[style=MyCStyle]{sum=0}}, dla drugiego \lstinline[style=MyCStyle]{sum=0+1}, dla trzeciego \lstinline[style=MyCStyle]{sum=0+1+2}, itd.).
\item \lstinline[style=MyCStyle]{msg} - wskaźnik do wiadomości otrzymanej z~programu głównego.
\end{myitemize}

W programie głównym utworzyć tablicę o~następującej zawartości:

\begin{lstlisting}[style=MyCStyle]
messages[0] = "English: Hello World!";
messages[1] = "French: Bonjour, le monde!";
messages[2] = "Spanish: Hola al mundo";
messages[3] = "Polski: Witaj swiecie!";
messages[4] = "German: Guten Tag, Welt!";
messages[5] = "Russian: Zdravstvytye, mir!";
messages[6] = "Japan: Sekai e konnichiwa!";
messages[7] = "Latin: Orbis, te saluto!";
\end{lstlisting}

Odpowiednie pola tablicy będą przekazywane do wątków. Każdy z~wątków powinien wyświetlać otrzymaną strukturę danych. Należy zadbać o~poprawne zakończenie wątków.

\item Zaproponować współbieżną (równoległą - gdy dysponujemy komputerem wieloprocesorowym) wersję programu do obliczania liczby $\pi$ za pomocą wątków. Matematycznie wiadomo, że:

\begin{equation}\nonumber
\displaystyle\int_0^1{\frac{4}{1+x^2}dx}=\pi
\end{equation}

Całkę oznaczoną możemy aproksymować jako sumę:

\begin{equation}\nonumber
\sum_{i=0}^N\frac{4}{1+x_i^2}\Delta x\approx\pi
\end{equation}

Ilustrację graficzną aproksymacji liczby $\pi$ można znaleźć na rysunku~\ref{fig:piApprox}.

\begin{figure}[!h]
\centering
\includegraphics[width=0.4\textwidth]{img/piApprox}
\caption{Idea aproksymacji liczby $\pi$}
\label{fig:piApprox}
\end{figure}

\lstinputlisting[caption=Kod źródłowy sekwencyjnej wersji do obliczania liczby $\pi$,style=MyCStyle,label=src:piApprox]{src/lab5/piApprox.c}

\item Rozszerzyć poprzedni przykład o możliwość zmiany priorytetów i strategii szeregowania wątków.
\end{myenumerate}


\cleardoublepage

\section{Zarządzanie wątkami – tworzenie, kończenie, atrybuty wątków}


\cleardoublepage

\section{Potoki nienazwane i nazwane}
\label{sec:TQIXQ}

Współpracujące procesy i wątki mogą komunikować się ze sobą poprzez
zastosowanie mechanizmów komunikacji międzyprocesowej IPC (ang. Inter-Process
Communication). System operacyjny QNX Neutrino oferuje następujące mechanizmy
komunikacji:
\begin{myitemize}
  \item potoki (łącza) nienazwane (pipes) i nazwane (kolejki FIFO),
  \item komunikaty (message passing) i impulsy (pulses) – specyficzne dla QNX,
  \item sygnały (signals),
  \item kolejki wiadomości (message queues),
  \item pamięć współdzielona (shared memory),
  \item gniazdka (sockets).
\end{myitemize}

Niektóre z metod komunikacji są wprost implementowane w jądrze systemu
operacyjnego QNX (komunikaty i sygnały), inne stanowią zewnętrzne procesy, bądź
są częścią menadżera procesów. Projektant aplikacji może wybrać odpowiednią
metodę komunikacji, po uwzględnieniu wymagań dotyczących działania aplikacji,
takich jak:
\begin{myitemize}
  \item Czy wymagane jest stosowanie standardu POSIX?
  \item Jaka jest częstotliwość i wielkość wysyłanych wiadomości?
  \item Czy niezbędna jest komunikacja sieciowa?
  \item Czy wolno używać komunikacji blokującej?
\end{myitemize}

W tym laboratorium omówimy jeden z podstawowych rodzajów komunikacji: potoki.

\subsection{Czym jest potok?}
\label{sec:JT0OO}

Mechanizm potoków wywodzi się z~systemu UNIX i jest jedną z~najstarszych metod
komunikacji między procesami. Użytkownicy popularnych systemów operacyjnych na
ogół spotykają się z~potokami na poziomie wykonywania komend w~wierszu poleceń.
Kiedy wpisujemy sekwencję poleceń:
%%%%%%%%%%%%%%%%%%%%%%%%%%%%%%%%%%%%%%%%%%%%%%%%%%%%%%%%%%%%%%%%%%%%%%%%%%%%%
\begin{lstlisting}[style=MyBashStyle]
# cmd1 | cmd2
\end{lstlisting}
%%%%%%%%%%%%%%%%%%%%%%%%%%%%%%%%%%%%%%%%%%%%%%%%%%%%%%%%%%%%%%%%%%%%%%%%%%%%%
strumień wyjściowy procesu \texttt{cmd1} jest przekazywany na wejście procesu
\texttt{cmd2}. Przykładem użycia potoków może być wywołanie w~wierszu poleceń
następującej sekwencji:
%%%%%%%%%%%%%%%%%%%%%%%%%%%%%%%%%%%%%%%%%%%%%%%%%%%%%%%%%%%%%%%%%%%%%%%%%%%%%
\begin{lstlisting}[style=MyBashStyle]
# ls -l | wc -w
\end{lstlisting}
%%%%%%%%%%%%%%%%%%%%%%%%%%%%%%%%%%%%%%%%%%%%%%%%%%%%%%%%%%%%%%%%%%%%%%%%%%%%%

Zasada działania potoku przy ,,sprzęganiu'' komend \texttt{cmd1}
i~\texttt{cmd2}, została pokazana na rysunku \ref{fig:1CP3G}.
%%%%%%%%%%%%%%%%%%%%%%%%%%%%%%%%%%%%%%%%%%%%%%%%%%%%%%%%%%%%%%%%%%%%%%%%%%%%%
\begin{figure}[!h]
  \centering
  \begin{tikzpicture}
    \node[draw=black]        (in1)  at (-5.50, 0.00) {KLAWIATURA};
    \node[circle,draw=black] (cmd1) at (-2.50, 0.00) {cmd1};
    \node[draw=black]        (pipe) at ( 0.00, 0.00) {POTOK};
    \node[circle,draw=black] (cmd2) at ( 2.50, 0.00) {cmd2};
    \node[draw=black]        (out2) at ( 5.50, 0.00) {MONITOR};
    \draw[->]                (in1.east)  to node[above=1ex]{stdin}  (cmd1.west);
    \draw[->]                (cmd1.east) to node[above=1ex]{stdout} (pipe.west);
    \draw[->]                (pipe.east) to node[above=1ex]{stdin}  (cmd2.west);
    \draw[->]                (cmd2.east) to node[above=1ex]{stdout} (out2.west);
  \end{tikzpicture}
  \caption{Użycie potoku do przekazywania strumieni danych}
  \label{fig:1CP3G}
\end{figure}
%%%%%%%%%%%%%%%%%%%%%%%%%%%%%%%%%%%%%%%%%%%%%%%%%%%%%%%%%%%%%%%%%%%%%%%%%%%%%
Symbole \texttt{stdin} i~\texttt{stdout} oznaczają odpowiednio wejściowy
i~wyjściowy strumień danych procesu. W~typowej sytuacji \texttt{stdin} pobiera
dane z~klawiatury a~\texttt{stdout} wyprowadza tekst na konsolę (monitor).
Użycie potoku umożliwia skierowanie strumienia \texttt{stdout} jednego procesu
do strumienia \texttt{stdin} innego procesu.

Potok jest nienazwanym plikiem, który stosuje się jako kanał komunikacyjny dla
dwóch lub więcej współpracujących procesów. Jeden z procesów pisze do potoku,
inny z niego czyta. Pojemność potoku jest ograniczona. Kiedy na przykład proces
\texttt{cmd1} pisze do potoku szybciej niż proces \texttt{cmd2} czyta, potok
może się napełnić i wtedy proces \texttt{cmd1} jest blokowany, aż do momentu,
gdy potok zostanie (przynajmniej częściowo) opróżniony przez \texttt{cmd2}.
Podobnie, gdy proces \texttt{cmd2} próbuje odebrać dane, które nie nadeszły, to
zostaje zablokowany, aż do momentu, gdy będą one dostępne.

W trakcie laboratorium poznamy niskopoziomowe funkcje dostępu do plików,
dowiemy się jaki jest wewnętrzny mechanizm przekazywania danych pomiędzy
procesami na~bazie potoków, a~także w~jaki sposób możemy zastosować potoki
do~komunikacji pomiędzy wieloma procesami.

\subsection{Niskopoziomowe funkcje dostępu do plików}
\label{sec:F6P4D}

Do obsługi potoków z~poziomu języka C można użyć niskopoziomowych funkcji
dostępu do plików. Cztery podstawowe funkcje z~tej rodziny przedstawiono
w~tabeli \ref{tab:67EKH}.
%%%%%%%%%%%%%%%%%%%%%%%%%%%%%%%%%%%%%%%%%%%%%%%%%%%%%%%%%%%%%%%%%%%%%%%%%%%%%
\begin{table}[!h]
  \centering
  \caption{Ważniejsze funkcje dostępu do plików}
  \label{tab:67EKH}
  \begin{tabular}{|l|l|}
    \hline
    \texttt{open()}   & Otwarcie lub utworzenie pliku \\ \hline
    \texttt{close()}  & Zamknięcie pliku \\ \hline
    \texttt{read()}   & Odczyt z pliku \\ \hline
    \texttt{write()}  & Zapis do pliku \\ \hline
  \end{tabular}
\end{table}
%%%%%%%%%%%%%%%%%%%%%%%%%%%%%%%%%%%%%%%%%%%%%%%%%%%%%%%%%%%%%%%%%%%%%%%%%%%%%
Typowy schemat postępowania z~plikiem sprowadza się do otwarcia pliku
(\texttt{open()}), dokonania serii odczytów/zapisów (\texttt{read()},
\texttt{write()}) a następnie zamknięcia pliku (\texttt{close()}).

Do otwarcia pliku można użyć funkcji \texttt{open()}.
%%%%%%%%%%%%%%%%%%%%%%%%%%%%%%%%%%%%%%%%%%%%%%%%%%%%%%%%%%%%%%%%%%%%%%%%%%%%%
\begin{lstlisting}[style=MyCStyle]
#include <sys/types.h>
#include <sys/stat.h>
#include <fcntl.h>

int open(const char *path, int oflags, mode_t mode);
\end{lstlisting}
%%%%%%%%%%%%%%%%%%%%%%%%%%%%%%%%%%%%%%%%%%%%%%%%%%%%%%%%%%%%%%%%%%%%%%%%%%%%%
Funkcja \texttt{open()} przyjmuje następujące argumenty:
\begin{myitemize}
  \item \texttt{path} -- ścieżka do pliku, który chcemy otworzyć,
  \item \texttt{oflags} -- flagi określające m.in. tryb dostępu do otwieranego
        pliku (tabela \ref{tab:97FHR}),
  \item \texttt{mode} -- jeśli w~argumencie \texttt{oflags} wybierzemy flagę
        \texttt{O\_CREAT}, musimy dostarczyć również zestaw parametrów
        specyfikujących prawa dostępu do nowo tworzonego pliku (tabela
        \ref{tab:VT9O5}).
\end{myitemize}
Funkcja \texttt{open()} zwraca deskryptor otwartego pliku (nieujemna liczba
całkowita) lub $-1$ w~przypadku niepowodzenia.

Flagi, których można użyć w~argumencie \texttt{oflags} przedstawiono w~tabeli
\ref{tab:97FHR}.
%%%%%%%%%%%%%%%%%%%%%%%%%%%%%%%%%%%%%%%%%%%%%%%%%%%%%%%%%%%%%%%%%%%%%%%%%%%%%
\begin{table}[!h]
  \centering
  \caption{Wybrane flagi, które można przekazać przez argument \texttt{oflags}}
  \label{tab:97FHR}
  \begin{tabular}{|l|l|}
    \hline
    \textbf{Flaga}      & \textbf{Opis} \\ \hline
    \texttt{O\_RDONLY}  & otwarcie tylko do odczytu \\ \hline
    \texttt{O\_RDWR}    & otwarcie dla odczytu i~zapisu \\ \hline
    \texttt{O\_WRONLY}  & otwarcie tylko do zapisu \\ \hline
    \texttt{O\_CREAT}   & utworzenie pliku, gdy nie istnieje \\ \hline
    \texttt{O\_APPEND}  & dopisywanie na końcu istniejącego pliku \\ \hline
    \texttt{O\_NONBLOC} & zapis i odczyt będą działać nieblokująco \\ \hline
  \end{tabular}
\end{table}
%%%%%%%%%%%%%%%%%%%%%%%%%%%%%%%%%%%%%%%%%%%%%%%%%%%%%%%%%%%%%%%%%%%%%%%%%%%%%

Flagi, których można użyć w~argumencie \texttt{mode} przedstawiono w~tabeli
\ref{tab:VT9O5}.
%%%%%%%%%%%%%%%%%%%%%%%%%%%%%%%%%%%%%%%%%%%%%%%%%%%%%%%%%%%%%%%%%%%%%%%%%%%%%
\begin{table}[!h]
  \centering
  \caption{Wybrane flagi, które można przekazać przez argument \texttt{mode}.
           Umożliwiają one zdefiniowanie praw dostępu do nowo tworzonego pliku
           (\texttt{O\_CREAT}) odpowiednio dla właściciela pliku, członków
           wyróżnionej grupy oraz pozostałych użytkowników}
  \label{tab:VT9O5}
  \begin{tabular}{|l|l|l|l|}
    \hline
    \textbf{User}     & \textbf{Group}    & \textbf{Others}   & \textbf{Permission} \\ \hline
    \texttt{S\_IRUSR} & \texttt{S\_IRGRP} & \texttt{S\_IROTH} & \texttt{r}          \\ \hline
    \texttt{S\_IWUSR} & \texttt{S\_IWGRP} & \texttt{S\_IWOTH} & \texttt{w}          \\ \hline
    \texttt{S\_IXUSR} & \texttt{S\_IXGRP} & \texttt{S\_IXOTH} & \texttt{x}          \\ \hline
    \texttt{S\_IRWXU} & \texttt{S\_IRWXG} & \texttt{S\_IRWXO} & \texttt{rwx}        \\ \hline
  \end{tabular}
\end{table}
%%%%%%%%%%%%%%%%%%%%%%%%%%%%%%%%%%%%%%%%%%%%%%%%%%%%%%%%%%%%%%%%%%%%%%%%%%%%%

Aby zamknąć plik, używamy funkcji \texttt{close()}:
%%%%%%%%%%%%%%%%%%%%%%%%%%%%%%%%%%%%%%%%%%%%%%%%%%%%%%%%%%%%%%%%%%%%%%%%%%%%%k
\begin{lstlisting}[style=MyCStyle]
#include <unistd.h>

int close(int filedes);
\end{lstlisting}
%%%%%%%%%%%%%%%%%%%%%%%%%%%%%%%%%%%%%%%%%%%%%%%%%%%%%%%%%%%%%%%%%%%%%%%%%%%%%
Funkcja \texttt{close()} przyjmuje jeden argument:
\begin{myitemize}
  \item \texttt{filedes} -- deskryptor pliku do zamknięcia, ten sam, który był
        zwrócony przez funkcję \texttt{open()}.
\end{myitemize}

Odczyt danych z~pliku realizujemy przez wywołanie funkcji \texttt{read()}
%%%%%%%%%%%%%%%%%%%%%%%%%%%%%%%%%%%%%%%%%%%%%%%%%%%%%%%%%%%%%%%%%%%%%%%%%%%%%k
\begin{lstlisting}[style=MyCStyle]
#include <unistd.h>

ssize_t read(int filedes, void *buf, size_t nbyte);
\end{lstlisting}
%%%%%%%%%%%%%%%%%%%%%%%%%%%%%%%%%%%%%%%%%%%%%%%%%%%%%%%%%%%%%%%%%%%%%%%%%%%%%
Argumenty przyjmowane przez funkcję \texttt{read()} to:
\begin{myitemize}
  \item \texttt{filedes} -- deskryptor otwartego pliku, z~którego chcemy czytać,
  \item \texttt{buf} -- wskaźnik do bufora, do którego funkcja ma zapisać przeczytane dane,
  \item \texttt{nbyte} -- liczba bajtów, którą chcemy przeczytać.
\end{myitemize}
Funkcja \texttt{read()} odczytuje \texttt{nbyte} bajtów z~pliku określonego
przez deskryptor \texttt{filedes} i~umieszcza dane w~buforze wskazywanym przez
\texttt{buf}. Gdy wywołanie funkcji się powiedzie, funkcja zwraca liczbę
przeczytanych bajtów umieszczonych w~buforze. W~przypadku błędu, funkcja zwraca
$-1$. Liczba bajtów zwracana przez funkcję może być mniejsza od \texttt{nbyte}
w~następujących przypadkach
\begin{myitemize}
  \item liczba bajtów w~pliku jest mniejsza niż \texttt{nbyte},
  \item działanie funkcji \texttt{read()} zostało przerwane przez sygnał,
  \item plik jest potokiem i~posiada w~danym momencie mniej bajtów danych niż
        \texttt{nbyte}.
\end{myitemize}

Zapis danych do pliku realizujemy poprzez wywołanie funkcji \texttt{write()}
%%%%%%%%%%%%%%%%%%%%%%%%%%%%%%%%%%%%%%%%%%%%%%%%%%%%%%%%%%%%%%%%%%%%%%%%%%%%%k
\begin{lstlisting}[style=MyCStyle]
#include <unistd.h>

ssize_t read(int filedes, const void *buf, size_t nbyte);
\end{lstlisting}
%%%%%%%%%%%%%%%%%%%%%%%%%%%%%%%%%%%%%%%%%%%%%%%%%%%%%%%%%%%%%%%%%%%%%%%%%%%%%
Argumenty przyjmowane przez funkcję \texttt{write()} to:
\begin{myitemize}
  \item \texttt{filedes} -- deskryptor otwartego pliku, do~którego chcemy pisać,
  \item \texttt{buf} -- wskaźnik do bufora w~pamięci, z którego funkcja czerpie
        dane przeznaczone do zapisu,
  \item \texttt{nbyte} -- liczba bajtów, którą chcemy zapisać.
\end{myitemize}
Funkcja \texttt{write()} zapisuje \texttt{nbyte} bajtów do pliku określonego
przez deskryptor \texttt{filedes} z~bufora wskazywanego przez \texttt{buf}.
Jeśli flaga \texttt{O\_APPEND} była ustawiona podczas otwierania pliku, to dane
są dopisywane na końcu istniejącego pliku. W przeciwnym przypadku zapis do
pliku następuje od miejsca wskazanego przez bieżącą pozycję pliku. Po dokonaniu
zapisu, wskaźnik bieżącej pozycji jest przesuwany o~liczbę zapisanych bajtów.
Liczba zapisanych bajtów może być mniejsza niż \texttt{nbyte} gdy:
\begin{myitemize}
  \item brakuje miejsca na nośniku,
  \item działanie funkcji \texttt{write()} zostało przerwane przez sygnał.
\end{myitemize}


%%%%%%%%%%%%%%%%%%%%%%%%%%%%%%%%%%%%%%%%%%%%%%%%%%%%%%%%%%%%%%%%%%%%%%%%%%%%%k
\begin{example}{[Utworzenie nowego pliku i~pisanie do pliku]}
  Uruchomić program, który tworzy nowy plik i~zapisuje do niego dane. Jako
  argument wywołania podać nazwę pliku.
  \lstinputlisting[style=MyCStyle,label=src:FXPNJ]{src/lab7/write1.c}
\end{example}
%%%%%%%%%%%%%%%%%%%%%%%%%%%%%%%%%%%%%%%%%%%%%%%%%%%%%%%%%%%%%%%%%%%%%%%%%%%%%k

%%%%%%%%%%%%%%%%%%%%%%%%%%%%%%%%%%%%%%%%%%%%%%%%%%%%%%%%%%%%%%%%%%%%%%%%%%%%%k
\begin{example}{[Kopiowanie plików]}
  Uruchomić program. Jako argumenty wywołania podać nazwę pliku do skopiowania,
  nazwę pliku docelowego i~liczbę bajtów do skopiowania.
  \lstinputlisting[style=MyCStyle,label=src:DNAMC]{src/lab7/copy1.c}
\end{example}
%%%%%%%%%%%%%%%%%%%%%%%%%%%%%%%%%%%%%%%%%%%%%%%%%%%%%%%%%%%%%%%%%%%%%%%%%%%%%k

\subsection{Potoki nienazwane}
\label{sec:NAMCF}

Komunikacja za pomocą potoków nienazwanych ma zastosowanie do procesów
pozostających w~relacji pokrewieństwa (np. proces potomny i~macierzysty). Przy
tworzeniu potoku nienazwanego tworzone są deskryptory plików reprezentujących
potok, które są dziedziczone przez procesy potomne. Komunikacja jest
jednokierunkowa. Deskryptor, który jest nieużywany przez dany proces, powinien
być w~przez ten proces zamknięty.

Utworzenie potoku wymaga wywołania funkcji \texttt{pipe()}:
%%%%%%%%%%%%%%%%%%%%%%%%%%%%%%%%%%%%%%%%%%%%%%%%%%%%%%%%%%%%%%%%%%%%%%%%%%%%%k
\begin{lstlisting}[style=MyCStyle]
#include <unistd.h>
int pipe(int filedes[2]);
\end{lstlisting}
%%%%%%%%%%%%%%%%%%%%%%%%%%%%%%%%%%%%%%%%%%%%%%%%%%%%%%%%%%%%%%%%%%%%%%%%%%%%%k
Funkcja \texttt{pipe} przyjmuje jeden argument
\begin{myitemize}
  \item \texttt{filedes} -- tablica przechowująca deskryptory plików do odczytu
        i~zapisu (końcówki potoku).
\end{myitemize}
Funkcja zwraca zero, gdy operacja się powiedzie, bądź $-1$ w~przeciwnym
przypadku. Element \texttt{filedes[0]} jest deskryptorem ,,pliku'' do odczytu
z~potoku, element \texttt{filedes[1]} jest deskryptorem ,,pliku'' do zapisu.
Operacje odczytu i~zapisu można przeprowadzić przy użyciu opisanych wcześniej
funkcji \texttt{read()} i \texttt{write()}. Zasadę działania potoku
nienazwanego przedstawiono na rysunku \ref{fig:OF6P4}.
%%%%%%%%%%%%%%%%%%%%%%%%%%%%%%%%%%%%%%%%%%%%%%%%%%%%%%%%%%%%%%%%%%%%%%%%%%%%%
\begin{figure}[!h]
  \centering
  \begin{tikzpicture}
    \node[circle,draw=black,fill=black] (proc1)   at (-3.50, 0.00) {o};
    \node[draw=black,fill=gray]         (pipe)    at ( 0.00, 0.00) {POTOK};
    \node[circle,draw=black,fill=black] (proc2)   at ( 3.50, 0.00) {o};
    \node[]                             (proc1t)  at (-3.50,-0.75)  {Proces Macierzysty};
    \node[]                             (proc2t)  at ( 3.50,-0.75)  {Proces Potomny};
    \draw[->]                (proc1.east) to node[above=1ex]{\texttt{filedes[1]}} (pipe.west);
    \draw[->]                (pipe.east) to node[above=1ex]{\texttt{filedes[0]}}  (proc2.west);
  \end{tikzpicture}
  \caption{Użycie potoku do przekazywania strumieni danych}
  \label{fig:OF6P4}
\end{figure}
%%%%%%%%%%%%%%%%%%%%%%%%%%%%%%%%%%%%%%%%%%%%%%%%%%%%%%%%%%%%%%%%%%%%%%%%%%%%%

System QNX zarządza potokami poprzez menadżera potoków. Menadżer potoków
przydziela ograniczoną pamięć dla łącza danych. Bufor ten domyślnie ma rozmiar
5120 bajtów. Funkcją umożliwiającą pobranie informacji o~zmiennych
konfiguracyjnych dotyczących plików (w~tym potoków) jest \texttt{fpathconf(int
filedes, int name)}, gdzie \texttt{filedes} jest deskryptorem pliku, a
\texttt{name} jest kodem numerycznym stanowiącym nazwę odpytywanej zmiennej. W
naszym przypadku \texttt{name = \_PC\_PIPE\_BUF}.

%%%%%%%%%%%%%%%%%%%%%%%%%%%%%%%%%%%%%%%%%%%%%%%%%%%%%%%%%%%%%%%%%%%%%%%%%%%%%
\begin{example}{[Potoki nienazwane w~obrębie jednego procesu]}
  Uruchomić program. Przetestować rozmiar bufora \texttt{BUFSIZE} równy $8$.
  Zwrócić uwagę na fakt, że komunikaty czytane są w~kolejności, w~jakiej
  zostały zapisane (na bazie FIFO).
  \lstinputlisting[style=MyCStyle,label=src:QJT0O]{src/lab7/pipe1.c}
\end{example}
%%%%%%%%%%%%%%%%%%%%%%%%%%%%%%%%%%%%%%%%%%%%%%%%%%%%%%%%%%%%%%%%%%%%%%%%%%%%%

%%%%%%%%%%%%%%%%%%%%%%%%%%%%%%%%%%%%%%%%%%%%%%%%%%%%%%%%%%%%%%%%%%%%%%%%%%%%%
\begin{example}{[Potoki nienazwane w~obrębie procesu macierzystego i~potomnego (fork)]}
  Uruchomić program, w~którym proces macierzysty pisze do potoku, natomiast
  proces potomny odczytuje dane z~potoku.
  \lstinputlisting[style=MyCStyle,label=src:4TQIX]{src/lab7/pipe2.c}
\end{example}
%%%%%%%%%%%%%%%%%%%%%%%%%%%%%%%%%%%%%%%%%%%%%%%%%%%%%%%%%%%%%%%%%%%%%%%%%%%%%

Proces potomny może wywołać funkcję z~rodziny \texttt{exec()}, która zastępuje
bieżący proces innym procesem. Możliwa jest wtedy komunikacja pomiędzy procesem
macierzystym, a nowo wywołanym procesem. Jednak wywołany proces nie dziedziczy
od macierzystego deskryptorów plików, zatem musimy mu je przekazać jako
parametry wywołania procesu. Sytuację tę ilustruje przykład \ref{ex:MLOE9}.
%%%%%%%%%%%%%%%%%%%%%%%%%%%%%%%%%%%%%%%%%%%%%%%%%%%%%%%%%%%%%%%%%%%%%%%%%%%%%
\begin{example}{[Potoki nienazwane i~funkcja \texttt{exec()}]}
  \label{ex:MLOE9}
  Skompilować i~uruchomić program \textbf{konsument.c} i~\textbf{producent.c}.
  \lstinputlisting[style=MyCStyle,label=src:EZI8J]{src/lab7/konsument.c}
  \lstinputlisting[style=MyCStyle,label=src:LNIZJ]{src/lab7/producent.c}
\end{example}
%%%%%%%%%%%%%%%%%%%%%%%%%%%%%%%%%%%%%%%%%%%%%%%%%%%%%%%%%%%%%%%%%%%%%%%%%%%%%

Do tej pory mieliśmy do czynienia z~sytuacją, gdy komunikujące się procesy
wymieniały informacje bez udziału standardowych strumieni. Często wygodnie jest
połączyć deskryptory plików, uzyskane z~funkcji \texttt{pipe()} ze standardowym
wejściem $0$ oraz standardowym wyjściem $1$. Aby to zrobić, potrzebujemy dwóch
spokrewnionych ze~sobą funkcji, które umożliwiają duplikowanie deskryptorów
plików:
%%%%%%%%%%%%%%%%%%%%%%%%%%%%%%%%%%%%%%%%%%%%%%%%%%%%%%%%%%%%%%%%%%%%%%%%%%%%%
\begin{lstlisting}[style=MyCStyle]
#include <unistd.h>
int dup( int filedes );
int dup2( int filedes, int filedes 2);
\end{lstlisting}
%%%%%%%%%%%%%%%%%%%%%%%%%%%%%%%%%%%%%%%%%%%%%%%%%%%%%%%%%%%%%%%%%%%%%%%%%%%%%
gdzie
\begin{myitemize}
  \item \texttt{filedes} -- deskryptor pliku, który chcemy zduplikować,
  \item \texttt{filedes2} -- nowy deskryptor pliku, który chcemy przypisać do \texttt{filedes}.
\end{myitemize}

Nowy deskryptor wskazuje na obiekt \texttt{filedes}, przekazany jako argument
wywołania funkcji. W~szczególności ma te same atrybuty i~tryby dostępu, jak
również wskazuje na tę samą pozycję w~pliku. Funkcja \texttt{dup()}
automatycznie generuje wartość nowego deskryptora pliku (duplikatu). Funkcja
\texttt{dup2()} pozostawia to zadanie programiście, tj. programista dostarcza
nowy deskryptor jako \texttt{filedes2}. Gdy deskryptor \texttt{filedes2} jest
już otwarty, wskazywany przez niego plik zostanie najpierw zamknięty.

W dalszych przykładach skojarzymy końce potoku nienazwanego ze strumieniami
\texttt{stdin} i~\texttt{stdout}. Zabieg taki umożliwia realizację z~poziomu
języka C wywołania sekwencji procesów w~postaci
\begin{lstlisting}[style=MyBashStyle]
# program1 | program2
\end{lstlisting}

%%%%%%%%%%%%%%%%%%%%%%%%%%%%%%%%%%%%%%%%%%%%%%%%%%%%%%%%%%%%%%%%%%%%%%%%%%%%%
\begin{example}{[Potoki nienazwane i~przekierowania strumieni]}
  Skompilować, uruchomić i~przeanalizować program. Porównać wynik programu
  z~wywołaniem: \texttt{ls -l | wc -w}
  \lstinputlisting[style=MyCStyle,label=src:7FHR6]{src/lab7/redirect1.c}
\end{example}
%%%%%%%%%%%%%%%%%%%%%%%%%%%%%%%%%%%%%%%%%%%%%%%%%%%%%%%%%%%%%%%%%%%%%%%%%%%%%

Mechanizm potoków nienazwanych jest dość skomplikowany. Programista musi dbać
o~tworzenie potoków (\texttt{pipe()}), zarządzanie procesami (\texttt{fork()},
\texttt{exec()}), czy przekierowanie strumieni standardowych (\texttt{dup2()}).
Istnieją funkcje, które ułatwiają ten proces. Należą do nich \texttt{popen()}
i~\texttt{pclose()}.
%%%%%%%%%%%%%%%%%%%%%%%%%%%%%%%%%%%%%%%%%%%%%%%%%%%%%%%%%%%%%%%%%%%%%%%%%%%%%
\begin{lstlisting}[style=MyCStyle]
FILE* popen( const char *command, const char *mode );
\end{lstlisting}
%%%%%%%%%%%%%%%%%%%%%%%%%%%%%%%%%%%%%%%%%%%%%%%%%%%%%%%%%%%%%%%%%%%%%%%%%%%%%
gdzie
\begin{myitemize}
  \item \texttt{command} -- komenda, którą chcemy uruchomić jako proces potomny,
  \item \texttt{mode} -- tryb otwierania potoku (\texttt{\char`\"r\char`\"}, bądź \texttt{\char`\"w\char`\"}).
\end{myitemize}
Funkcja \texttt{popen()} uruchamia proces wskazany przez \texttt{command}
i~tworzy potok między procesem wywołującym a~procesem \texttt{command}.
W~zależności od ustawionego trybu, zwracany wskaźnik na strukturę \texttt{FILE}
może być użyty do odczytu bądź zapisu. W~przypadku błędu, funkcja zwraca
\texttt{NULL}.

Aby zamknąć potok stowarzyszony ze strumieniem \texttt{stream} używamy funkcji
\texttt{pclose()}:
%%%%%%%%%%%%%%%%%%%%%%%%%%%%%%%%%%%%%%%%%%%%%%%%%%%%%%%%%%%%%%%%%%%%%%%%%%%%%
\begin{lstlisting}[style=MyCStyle]
FILE* pclose( FILE* stream );
\end{lstlisting}
%%%%%%%%%%%%%%%%%%%%%%%%%%%%%%%%%%%%%%%%%%%%%%%%%%%%%%%%%%%%%%%%%%%%%%%%%%%%%
gdzie
\begin{myitemize}
  \item \texttt{stream} -- wskaźnik na strumień otwarty wcześniej funkcją \texttt{popen()}.
\end{myitemize}
Funkcja \texttt{pclose()} zamyka potok i~jednocześnie czeka na zakończenie
podprocesów utworzonych przez \texttt{popen()}. Funkcja zwraca status
zakończenie, bądź $-1$, w przypadku błędu.

\begin{example}{[Przykład pisania do potoku nienazwanego \texttt{popen()}, \texttt{pclose()}]}
  Skompilować, uruchomić program.
  \lstinputlisting[style=MyCStyle,label=src:7EKH1]{src/lab7/popen1.c}
\end{example}

\subsection{Potoki nazwane}
\label{sec:XPNJV}

Potoki nazwane, określane również mianem plików FIFO, pozwalają na uniknięcie
ograniczeń, którymi obarczone są potoki nienazwane. Pliki FIFO istnieją
w~przestrzeni nazw plików i~mają atrybuty, prawa dostępu, właścicieli, takie,
jak zwykłe pliki. Do obsługi potoków nazwanych używa się głównie funkcji
plikowych, jedyna różnica, to inicjalizacja potoku:
%%%%%%%%%%%%%%%%%%%%%%%%%%%%%%%%%%%%%%%%%%%%%%%%%%%%%%%%%%%%%%%%%%%%%%%%%%%%%
\begin{lstlisting}[style=MyCStyle]
#include <sys/types.h>
#include <sys/stat.h>

int mkfifo( const char* path, mode_t mode );
\end{lstlisting}
%%%%%%%%%%%%%%%%%%%%%%%%%%%%%%%%%%%%%%%%%%%%%%%%%%%%%%%%%%%%%%%%%%%%%%%%%%%%%
Funkcja \texttt{mkfifo()} przyjmuje następujące argumenty
\begin{myitemize}
  \item \texttt{path} -- nazwa pliku FIFO,
  \item \texttt{mode} -- prawa dostępu do pliku.
\end{myitemize}
Funkcja \texttt{mkfifo()} tworzy nowy plik specjalny FIFO o~nazwie podanej
przez parametr \texttt{path}. Plik FIFO będzie miał uprawnienia określone przez
argument \texttt{mode}. Funkcja zwraca $0$ w~przypadku powodzenia i~$-1$
w~przypadku błędu.

Otwieranie (\texttt{open()}) pliku FIFO do zapisu blokuje proces bieżący do
czasu, gdy inny proces nie otworzy tego pliku do odczytu. Podobna sytuacja ma
miejsce, gdy proces otwiera plik do odczytu -- jest on blokowany dotąd, aż inny
proces otworzy plik do zapisu. Możliwe jest także nieblokujące wywołanie
funkcji \texttt{open()} dla pliku FIFO (należy użyć flagi \texttt{O\_NONBLOCK}).

\begin{example}{[Tworzenie i~używanie plików FIFO z~wiersza poleceń]}
\begin{lstlisting}[style=MyBashStyle]
# mkfifo /tmp/fifo
# ls -l /tmp/fifo
prw-rw-r--  1 root root 0 Sep 12 12:24 fifo
\end{lstlisting}
Litera \textbf{p} w~atrybutach (\texttt{prw-rw-r}) oznacza, że jest to potok
nazwany (named pipe). Plik FIFO moze być odczytywany i~zapisywany za~pomocą
standardowych poleceń.
\end{example}

\begin{example}{[Zapis/odczyt pliku FIFO z~wiersza poleceń]}
Używając dwu konsoli wpisać następujące polecenia:

Konsola 1:
\begin{lstlisting}[style=MyBashStyle]
# cat < /tmp/fifo
\end{lstlisting}

Konsola 2:
\begin{lstlisting}[style=MyBashStyle]
# cat > /tmp/fifo
\end{lstlisting}

Wpisać w~konsoli 2 kilka linii tekstu i zakończyć pisanie (Ctl+D). Przełączyć
się na konsolę 1 i obejrzeć rezultat. Zakończyć polecenie na konsoli 1 (Ctl+C).
Usunąć plik FIFO jak zwykły plik.
\end{example}

Pliki FIFO można oczywiście obsługiwać z~poziomu języka C. Poniższy przykład
ilustruje sposób tworzenia, otwierania i~zamykania pliku FIFO.
%%%%%%%%%%%%%%%%%%%%%%%%%%%%%%%%%%%%%%%%%%%%%%%%%%%%%%%%%%%%%%%%%%%%%%%%%%%%%
\begin{example}{[Tworzenie i~używanie potoków nazwanych w~C]}
  Uruchomić program z~argumentami wywołania \texttt{O\_RDONLY} lub
  \texttt{O\_WRONLY}. Zaobserwować zachowanie się procesu. Sprawdzić obecność
  pliku FIFO w~systemie plików po wywołaniu programu. Uruchomić program w~dwóch
  konsolach, w pierwszej z~flagą \texttt{O\_RDONLY}, w~drugiej z~flagą
  \texttt{O\_WRONLY}. Sprawdzić co się stanie, gdy do flag dodamy opcję
  \texttt{O\_NONBLOCK}.
  \lstinputlisting[style=MyCStyle,label=src:5A3PM]{src/lab7/fifo1.c}
\end{example}
%%%%%%%%%%%%%%%%%%%%%%%%%%%%%%%%%%%%%%%%%%%%%%%%%%%%%%%%%%%%%%%%%%%%%%%%%%%%%

Następny przykład jest ilustracją zastosowania potoków nazwanych do
implementacji bardzo prostego jednokierunkowgo komunikatora. W~przykładzie mamy
proces-serwer, który odbiera komunikaty od klienta i wyprowadza je na
standardowe wyjście. Programy tworzą szkielet prostego systemu buforowania.
%%%%%%%%%%%%%%%%%%%%%%%%%%%%%%%%%%%%%%%%%%%%%%%%%%%%%%%%%%%%%%%%%%%%%%%%%%%%%
\begin{example}{[Prosty system buforowania na~bazie potoków nazwanych]}
  Skompilować, uruchomić i~przeanalizować działanie klienta i serwera.
  \lstinputlisting[style=MyCStyle,label=src:CP3GC]{src/lab7/client1.c}
  \lstinputlisting[style=MyCStyle,label=src:5RXYW]{src/lab7/server1.c}
\end{example}
%%%%%%%%%%%%%%%%%%%%%%%%%%%%%%%%%%%%%%%%%%%%%%%%%%%%%%%%%%%%%%%%%%%%%%%%%%%%%

\subsection{Ćwiczenia}
\label{sec:T9O59}

\begin{myenumerate}
  \item Napisać program, który otwiera plik podany jako argument wywołania i
    zlicza liczbę znaków występujących w pliku. Przetestować program na małym
    pliku z kilkoma liniami tekstu.
  \item Napisać następujące programy:
    \begin{myenumerate}
      \item Program tylko pisze do potoku w pętli nieskończonej po jednym
        znaku. Dane nie są nigdy odczytywane z potoku. Pokazać całkowitą liczbę
        bajtów zapisaną do potoku. Co się dzieje ze stanem procesu?
      \item Program tylko odczytuje z potoku w pętli nieskończonej po jednym
        znaku. Dane nie są nigdy zapisywane do potoku. Co się dzieje ze stanem
        procesu?
      \item O zachowaniu się procesów w sytuacjach a. i b. decyduje flaga
        \texttt{O\_NONBLOCK}, którą można kontrolować za pomocą funkcji
        \texttt{fcntl()}. Domyślnie flaga jest wyzerowana, informację o jej
        stanie można pobrać przez wywołanie funkcji:
        \texttt{fcntl(fd,F\_GETFL,O\_NONBLOCK);} Stan procesów w przypadku a. i
        b. można zmienić poprzez ustawienie flagi \texttt{O\_NONBLOCK}
        wywołaniem: \texttt{fcntl(fd,F\_SETFL,O\_NONBLOCK);} Jak się
        zachowują oba programy ?
    \end{myenumerate}
    \item Napisać dwa programy p1 i p2. Pierwszy z nich ma wypisywać na
      standardowe wyjście napis (funkcja \texttt{fprintf (stdout, "Tekst1")}).
      Drugi program ma czytać ze standardowego wejścia (funkcja \texttt{fscanf
      (stdin, \&zmienna)}) i wyświetlać przetworzony tekst. Wywołać i przetestować
      programy pojedynczo. Wywołać programy przez potok (p1 | p2). Zastosować
      funkcję \texttt{join()} do połączenia obu programów.
    \item Programy \textbf{klient.c} i \textbf{serwer.c} tworzą prosty system
      wymiany komunikatów. Zaimplementować program, w którym klient wysyła do
      serwera nazwę pliku do utworzenia oraz dane, a serwer tworzy plik i
      zapisuje do niego przekazane informacje.
\end{myenumerate}

\cleardoublepage



%% \label{??:0PLQB}
%% \label{??:KJUXZ}
%% \label{??:TALG5}
%% \label{??:SU8SV}
%% \label{??:4CQZR}
%% \label{??:H0FFY}
%% \label{??:LXOQ8}
%% \label{??:RYU75}
%% \label{??:K0HB8}
%% \label{??:HS3H9}
%% \label{??:FWSDE}
%% \label{??:0EIOG}
%% \label{??:2998B}
%% \label{??:RW1MV}
%% \label{??:U68A4}
%% \label{??:MWRG3}
%% \label{??:5J36G}
%% \label{??:D2BST}
%% \label{??:BHQ9N}
%% \label{??:9P2K2}
%% \label{??:A5I7M}
%% \label{??:BINKS}
%% \label{??:4EJKR}
%% \label{??:74OPC}
%% \label{??:GTZSE}
%% \label{??:ABAD3}
%% \label{??:7CO66}
%% \label{??:HAC6N}
%% \label{??:8TRF7}
%% \label{??:LE8IP}
%% \label{??:DD44N}
%% \label{??:T1LYK}
%% \label{??:1DMDC}
%% \label{??:7KIYM}
%% \label{??:UYV3F}
%% \label{??:GVW2J}
%% \label{??:5Q1NS}

\section{Mechanizmy komunikacji w systemie QNX Neutrino - komunikaty}

\subsection{Wprowadzenie}

W systemie QNX Neutrino istnieje wiele różnych metod komunikacji międzyprocesorowej. Przesyłanie komunikatów (message passing) jest podstawową formą IPC (interprocess communication) w systemie QNX. Ten unikalny dla Neutrino mechanizm został zaimplementowany wprost w mikrojądrze systemu operacyjnego i stanowi prymityw do komunikacji dwukierunkowej modułów (np. menadżera procesów, modułu systemu plików) z mikrojądrem. Pozwala to na odseparowanie procesów od mikrojądra. W przypadku awarii oprogramowania, procesy modułów nie mają bezpośredniego wpływu na działanie mikrojądra. Taka architektura (patrz rysunek~\ref{fig:microkernel}) zapewnia wysoką wydajność, konfigurowalność i skalowalność systemu, dostosowaną do ograniczeń nakładanych na system. Inne mechanizmy IPC (np. omawiane wcześniej potoki i~pliki FIFO) są nadbudową tej formy komunikacji międzyprocesorowej. 


\begin{figure}[!h]
\centering
\includegraphics[width=0.5\textwidth]{img/microkernel}
\caption{Modularna architektura QNX Neutrino}
\label{fig:microkernel}
\end{figure}

W zastosowaniach, często spotyka się aplikacje oparte o model klient-serwer, jak na rysunku\ref{fig:clientserver}
. Początkowo serwer czeka na wiadomości od procesów-klientów. Procesy klienta wysyłają (1) dane do serwera, a następnie zostają zablokowane w oczekiwaniu na odpowiedź. W tym czasie serwer otrzymuje wiadomości (2), przetwarza je i odpowiada klientom (3), które kontynuują swoje działanie.

\begin{figure}[!h]
\centering
\includegraphics[width=0.65\textwidth]{img/clientserver}
\caption{Model aplikacji typu klient-serwer}
\label{fig:clientserver}
\end{figure}

Ten model wymiany informacji może być zrealizowany poprzez przesyłanie komunikatów (message passing), na który składają się następujące mechanizmy: 

\begin{myitemize}
\item Tworzenie kanałów i połączeń.
\item Wysyłanie, odbieranie i potwierdzanie komunikatów.
\item Impulsy (krótkie, nieblokujące nadawcy komunikaty).
\item Przesyłanie komunikatów poprzez sieć oraz rejestrowanie nazw procesów, w ramach usługi GNS. 
\end{myitemize} 


\subsection{Tworzenie kanałów i połączeń}

W systemie QNX Neutrino przesyłanie komunikatów nie następuje bezpośrednio pomiędzy procesami, czy wątkami. Medium pośredniczącym w przekazywaniu komunikatów jest kanał komunikacyjny. Sytuację pokazano na rysunku~\ref{fig:channels}. Proces serwer, który odbiera komunikaty, tworzy kanał i oczekuje w nim wiadomości od klientów. Procesy klient dołącza się do kanału poprzez połączenia i przez nie wysyła do serwera komunikaty. Klient może mieć wiele połączeń, do różnych serwerów, natomiast serwer używa wyłącznie jednego kanału komunikacyjnego do odbierania wiadomości od klientów. 


\begin{figure}[!h]
\centering
\includegraphics[width=0.65\textwidth]{img/channels}
\caption{Kanały i połączenia}
\label{fig:channels}
\end{figure}

Funkcje służące obsłudze kanałów i połączeń przedstawiono w tabeli~\ref{tab:channels}. 

\begin{table}[h!]
\centering
\caption{Obsługa kanałów i połączeń}
\setlength{\arrayrulewidth}{1pt}
\setlength{\tabcolsep}{6pt}
\renewcommand{\arraystretch}{1.2}
\begin{tabular}{ |p{0.25\textwidth}|p{0.5\textwidth}|}
\hline \rowcolor{gray}
\textbf{Funkcja} & \textbf{Opis} \\ \hline
\mbox{\lstinline[style=MyCStyle]{ChannelCreate()}} & Tworzenie kanału do odbioru wiadomości \\ \hline
\mbox{\lstinline[style=MyCStyle]{ChannelDestroy()}} & Kasowanie kanału \\ \hline
\mbox{\lstinline[style=MyCStyle]{ChannelAttach()}} & Tworzenie połączenia do wysyłania wiadomości \\ \hline
\mbox{\lstinline[style=MyCStyle]{ChannelDetach()}} & Kasowanie połączenia \\ \hline
\end{tabular}
\label{tab:channels}
\end{table}

Pierwszą czynnością, którą należy wykonać przy implementacji serwera jest utworzenie kanału komunikacyjnego. Kanał komunikacyjny jest własnością procesu, który go utworzył. Wątki, które chcą się skomunikować z danym kanałem mogą być zawarte w tym samym procesie, mogą należeć do innego procesu w obrębie tego samego węzła, bądź innego węzła w sieci. Każdorazowo jednak muszą one utworzyć z kanałem połączenie. Kanał komunikacyjny tworzymy za pomocą funkcji:

\begin{lstlisting}[style=MyCStyle]
#include <sys/neutrino.h>
int ChannelCreate( unsigned flags );
int ChannelCreate_r( unsigned flags );
\end{lstlisting}

gdzie \lstinline[style=MyCStyle]{flags} stanowią opcje, które mogą być używane do zmiany własności kanału (tabela~\ref{tab:flags}). 

\begin{table}[h!]
\centering
\caption{Opcje tworzenia kanałów}
\setlength{\arrayrulewidth}{1pt}
\setlength{\tabcolsep}{6pt}
\renewcommand{\arraystretch}{1.2}
\begin{tabular}{ |p{0.3\textwidth}|p{0.5\textwidth}|}
\hline \rowcolor{gray}
\textbf{Opcje} & \textbf{Opis} \\ \hline
\mbox{\lstinline[style=MyCStyle]{_NTO_CHF_COID_DISCONNECT}} & Dostarcz impuls, kiedy dowolne połączenie do kanału jest zamykane. \\ \hline
\mbox{\lstinline[style=MyCStyle]{_NTO_CHF_DISCONNECT}} & Dostarcz impuls, kiedy wszystkie połączenia do kanału są zamknięte.  \\ \hline
\mbox{\lstinline[style=MyCStyle]{_NTO_CHF_FIXED_PRIORITY}} & Nie stosuje dziedziczenia priorytetów. \\ \hline
\mbox{\lstinline[style=MyCStyle]{_NTO_CHF_NET_MSG}} & Zarezerwowane dla menadżera zasobów (ionet) \\ \hline
\mbox{\lstinline[style=MyCStyle]{_NTO_CHF_REPLY_LEN}} & Długość odpowiedzi ma być zawarta w strukturze \mbox{\lstinline[style=MyCStyle]{_msg_info}}, którą wypełnia funkcja \mbox{\lstinline[style=MyCStyle]{MsgReceivev()}}. \\ \hline
\mbox{\lstinline[style=MyCStyle]{_NTO_CHF_SENDER_LEN}} & Długość komunikatu ma być zawarta w strukturze \mbox{\lstinline[style=MyCStyle]{_msg_info}}, którą wypełnia funkcja \mbox{\lstinline[style=MyCStyle]{MsgReceivev()}}. \\ \hline
\mbox{\lstinline[style=MyCStyle]{_NTO_CHF_THREAD_DEATH}} & Dostarcz do kanału impuls, gdy zakończy się dowolny wątek posiadający kanał. \\ \hline
\mbox{\lstinline[style=MyCStyle]{_NTO_CHF_UNBLOCK}} & Dostarcz do kanału impuls, gdy wątek wysyłający będący w stanie \mbox{\lstinline[style=MyCStyle]{REPLY_BLOCKED}} odblokuje się przed wysłaniem mu odpowiedzi funkcją \mbox{\lstinline[style=MyCStyle]{MsgReply()}}.  \\ \hline
\end{tabular}
\label{tab:flags}
\end{table}

Funkcje \lstinline[style=MyCStyle]{ChannelCreate()} i \lstinline[style=MyCStyle]{ChannelCreate_r()}  tworzą kanał komunikacyjny, który może być użyty do odbierania komunikatów lub impulsów. Zwracają identyfikator kanału \lstinline[style=MyCStyle]{CHID} (channel ID) lub \lstinline[style=MyCStyle]{-1} i ustawiają kod błędu. Komunikaty i impulsy są odbierane w kanale komunikacyjnym i ustawiane w kolejkę, zgodnie z wartościami priorytetów. Domyślnie, kiedy proces (wątek) odbiera komunikat z kanału, jego priorytet jest ustawiany w ten sposób, aby był równy priorytetowi procesu (wątku) wysyłającego komunikat. Metoda ta, zwana dziedziczeniem priorytetów (priority inheritance) zapobiega inwersji priorytów. Po otrzymaniu komunikatu, wątek może odłączyć się od kanału poprzez wywołanie \lstinline[style=MyCStyle]{MsgReceive()} z kanału \lstinline[style=MyCStyle]{-1}.  Dziedziczenie priorytetów można wyłączyć, poprzez ustawienie flagi \lstinline[style=MyCStyle]{_NTO_CHF_FIXED_PRIORITY}. Różnica między funkcjami \lstinline[style=MyCStyle]{ChannelCreate()}  i \lstinline[style=MyCStyle]{ChannelCreate_r()} polega na tym, że pierwsza z nich ustawia globalną zmienną \lstinline[style=MyCStyle]{errno}, umożliwiającą odczytanie kodu błędu, a także samego opisu błędu przez funkcję  \lstinline[style=MyCStyle]{char* strerror( int errno );}. 

Aby skasować kanał należy wywołać funkcję:

\begin{lstlisting}[style=MyCStyle]
#include <sys/neutrino.h>
int ChannelDestroy( int chid );
int ChannelDestroy_r( int chid );
\end{lstlisting}


gdzie \lstinline[style=MyCStyle]{chid} - jest numerem kanału zwróconym przez funkcję  \lstinline[style=MyCStyle]{ChannelCreate()}, bądź  \lstinline[style=MyCStyle]{ChannelCreate_r()}.
 
Kiedy kanał jest kasowany, to wszystkie wątki, które są w stanie zablokowany na operacjach  \lstinline[style=MyCStyle]{MsgSend()} i~ \lstinline[style=MyCStyle]{MsgReceive()} zostaną odblokowane. Odbieranie wiadomości, bądź impulsów z kanałów po jego zamknięciu zakończy się niepowodzeniem. 

\begin{example}{[Utworzenie kanału komunikacyjnego]} Skompilować, zbudować oraz uruchomić przykład. 
\lstinputlisting[caption=Kanał komunikacyjny,style=MyCStyle,label=src:channel]{src/lab8/channel.c}
\end{example} 

Ustanowienie połączenia między procesem (wątkiem), a~kanałem wymaga wyłania funkcji \lstinline[style=MyCStyle]{ConnectAttach()}: 

 
\begin{lstlisting}[style=MyCStyle]
#include <sys/neutrino.h>
int ConnectAttach( uint32_t nd,
                   pid_t pid,
                   int chid,
                   unsigned index,
                   int flags );

int ConnectAttach_r( uint32_t nd,
                     pid_t pid,
                     int chid,
                     unsigned index,
                     int flags );
\end{lstlisting}

gdzie 

\begin{myitemize}
\item[] \lstinline[style=MyCStyle]{nd} - jest numerem węzła, na którym uruchomiony jest proces posiadający kanał lub \lstinline[style=MyCStyle]{0}, bądź \lstinline[style=MyCStyle]{ND_LOCAL_NODE}~, w~przypadku węzła bieżącego.
\item[] \lstinline[style=MyCStyle]{pid} - numer PID procesu serwera, tzn. zawierającego kanał komunikacyjny.
\item[] \lstinline[style=MyCStyle]{chid} - jest numerem kanału zwróconym przez funkcję \lstinline[style=MyCStyle]{ChannelCreate()}.
\item[] \lstinline[style=MyCStyle]{index} - najmniejszy akceptowalny numer połączenia; przekazanie flagi \lstinline[style=MyCStyle]{_NTO_SIDE_CHANNEL} powoduje wybranie numeru połączenia z innego zakresu, niż deskryptory plików. Rekomenduje się używanie tej flagi podczas wywołania funkcji. 
\item[] \lstinline[style=MyCStyle]{flags} - flagi modyfikujące działanie; jeśli zawierają \lstinline[style=MyCStyle]{_NTO_COF_CLOEXEC}, to połączenie jest zamykane, kiedy proces wywołuje funkcję z rodziny \lstinline[style=MyCStyle]{exec()}. 
\end{myitemize}

Funkcja zwraca identyfikator połączenia \lstinline[style=MyCStyle]{COID} (connection \lstinline[style=MyCStyle]{ID}), bądź \lstinline[style=MyCStyle]{-1} w przypadku niepowodzenia. Podobnie, jak poprzednio, funkcje \lstinline[style=MyCStyle]{ConnectAttach()} i \lstinline[style=MyCStyle]{ConnectAttach_r()} różnią się ustawianiem zmiennej globalnej \lstinline[style=MyCStyle]{errno}.
 
Zamykanie połączenia realizujemy funkcjami: 

\begin{lstlisting}[style=MyCStyle]
#include <sys/neutrino.h>
int ConnectDetach( int coid );
int ConnectDetach_r( int coid );
\end{lstlisting}

gdzie 

\begin{myitemize}
\item[] \lstinline[style=MyCStyle]{coid} - jest numerem kanału, który chcemy skasować. 
\end{myitemize} 

Funkcja zwraca wartość dodatnią, w przypadku sukcesu i \lstinline[style=MyCStyle]{-1}, w przypadku niepowodzenia. W przypadku wywołanai funkcji, wątki, które są zablokowane na połączeniu, zostaną odblokowane, a funkcja \lstinline[style=MyCStyle]{MsgSend()} zwróci kod błędu. 

\begin{example}{[Utworzenie połączenia]} Skompilować, zbudować oraz uruchomić przykład. Dlaczego wynikiem działania programu jest błąd?
\lstinputlisting[caption=Połączenie,style=MyCStyle,label=src:connect]{src/lab8/connect.c}
\end{example} 

\subsection{Wysyłanie, odbieranie i odpowiadanie na komunikaty}

Komunikacja pomiędzy klientem a serwerem za pomocą przesyłania komunikatów (spotkań) jest komunikacją dwukierunkową, synchroniczną i składa się z trzech etapów: wysyłanie komunikatu przez klienta do serwera, odbiór komunikatu przez serwer i przesłanie przez serwer odpowiedzi do klienta. System operacyjny QNX Neutrino dostarcza całej grupy funkcji, służących do obsługi opisanych trzech etapów komunikacji. Można jednak napisać w pełni funkcjonalne aplikacje typu klient-serwer posługując się minimalnym zestawem funkcji takich, jak: \lstinline[style=MyCStyle]{ChannelCreate()}, \lstinline[style=MyCStyle]{ChannelDestroy()}, \lstinline[style=MyCStyle]{ConnectAttach()}, \lstinline[style=MyCStyle]{ConnectDetach()}, \lstinline[style=MyCStyle]{MsgReply()}, \lstinline[style=MyCStyle]{MsgSend()}, \lstinline[style=MyCStyle]{MsgReceive()}. 

W trakcie komunikacji za pomocą przesyłania komunikatów mogą się zdarzyć dwa scenariusze: 

\begin{myenumerate} 
\item Wywołanie funkcji \lstinline[style=MyCStyle]{MsgSend()} następuje po wywołaniu funkcji \lstinline[style=MyCStyle]{MsgReceive()}. Serwer jest blokowany do momentu nadejścia od nadawcy wiadomości i pozostaje w stanie \lstinline[style=MyCStyle]{RECEIVE blocked} do momentu odebrania wiadomości wysłanej przez klienta. W trakcie przetwarzania danych przez serwer, klient pozostaje w stanie \lstinline[style=MyCStyle]{REPLY blocked}, aż do czasu, gdy otrzyma odpowiedź od serwera. Poza omówionymi stanami, procesy klienta i serwera pozostają w stanie \lstinline[style=MyCStyle]{READY}. Sytuację tę ilustruje rysunek~\ref{fig:Msg1}. 

\begin{figure}[!h]
\centering
\includegraphics[width=0.75\textwidth]{img/Msg1}
\caption{Ilustracja przesyłania komunikatów - przypadek 1}
\label{fig:Msg1}
\end{figure}

\item Wywołanie funkcji \lstinline[style=MyCStyle]{MsgSend()} następuje przed wywołaniem funkcji \lstinline[style=MyCStyle]{MsgReceive()}. Klient w~stanie \lstinline[style=MyCStyle]{SEND blocked} pozostaje zablokowany  do momentu kiedy serwer rozpocznie odbieranie komunikatu. Jak poprzedni, w trakcie przetwarzania danych przez serwer, klient pozostaje w stanie \lstinline[style=MyCStyle]{REPLY blocked}, aż do czasu, gdy otrzyma odpowiedź od serwera. Poza omówionymi stanami, procesy klienta i serwera pozostają w stanie \lstinline[style=MyCStyle]{READY}. Sytuację tę ilustruje rysunek~\ref{fig:Msg2}. 

\begin{figure}[!h]
\centering
\includegraphics[width=0.75\textwidth]{img/Msg2}
\caption{Ilustracja przesyłania komunikatów - przypadek 2}
\label{fig:Msg2}
\end{figure}
\end{myenumerate} 

Wysyłanie komunikatu realizujemy poprzez następujące wywołanie: 

\begin{lstlisting}[style=MyCStyle]
#include <sys/neutrino.h>

int MsgSend( int coid,
             const void* smsg,
             int sbytes,
             void* rmsg,
             int rbytes );

int MsgSend_r( int coid,
               const void* smsg,
               int sbytes,
               void* rmsg,
               int rbytes );
\end{lstlisting}

gdzie 

\begin{myitemize}
\item[] \lstinline[style=MyCStyle]{coid} - jest numerem kanału, do którego chcemy wysyłać wiadomości. 
\item[] \lstinline[style=MyCStyle]{smsg} - wskaźnik do bufora zawierającego wiadomość do wysłania. 
\item[] \lstinline[style=MyCStyle]{sbytes} - liczba bajtów określająca pojemność bufora do wysłania.
\item[] \lstinline[style=MyCStyle]{rmsg} - wskaźnik do bufora zawierającego odpowiedź. 
\item[] \lstinline[style=MyCStyle]{rbytes} - liczba bajtów określająca pojemność bufora na odpowiedź. 
\end{myitemize}

Funkcja \lstinline[style=MyCStyle]{MsgSend()} wysyła komunikat do kanału określonego przez \lstinline[style=MyCStyle]{coid}. Funkcja ta służy zarówno do wysyłania komunikatu, jak i odbierania wiadomości do serwera. \lstinline[style=MyCStyle]{MsgSend()} zwraca \lstinline[style=MyCStyle]{status}, który jest przekazywany przez \lstinline[style=MyCStyle]{MsgReply()} lub \lstinline[style=MyCStyle]{-1} w przypadku błędu. Działanie funkcji zależy od stanu procesu serwera. Przejścia stanów procesu-klienta zilustrowano na rysunku~\ref{fig:states1}. 

\begin{figure}[!h]
\centering
\includegraphics[width=0.5\textwidth]{img/states1}
\caption{Przejścia stanów procesu klienta}
\label{fig:states1}
\end{figure}

Serwer może odebrać komunikat z kanału poprzez wywołanie następującej funkcji: 

\begin{lstlisting}[style=MyCStyle]
#include <sys/neutrino.h>

int MsgReceive( int chid,
                void * msg,
                int bytes,
                struct _msg_info * info );

int MsgReceive_r( int chid,
                  void * msg,
                  int bytes,
                  struct _msg_info * info );
\end{lstlisting}

gdzie

\begin{myitemize}
\item[] \lstinline[style=MyCStyle]{chid} - numer kanału zwrócony przez funkcję \lstinline[style=MyCStyle]{ChannelCreate()}.
\item[] \lstinline[style=MyCStyle]{msg} - wskaźnik do bufora zawierającego odebraną wiadomość. 
\item[] \lstinline[style=MyCStyle]{bytes} - liczba bajtów określająca pojemność bufora do odbioru komunikatu.
\item[] \lstinline[style=MyCStyle]{info} - wskaźnik \lstinline[style=MyCStyle]{NULL}, bądź wskaźnik do struktury \lstinline[style=MyCStyle]{_msg_info}, zawierającej dodatkowe informacje o komunikacie.  
\end{myitemize}

Funkcja odbiera wiadomość, bądź impuls od klienta. W przypadku komunikatu, funkcja zwraca identyfikator nadawcy \lstinline[style=MyCStyle]{rcvid > 0} (receive identifier), w której zakodowano identyfikator wątku (\lstinline[style=MyCStyle]{TID}) wysyłającego i identyfikator połączenia (\lstinline[style=MyCStyle]{COID}). Identyfikator \lstinline[style=MyCStyle]{rcvid} będzie użyty w funkcji \lstinline[style=MyCStyle]{MsgReply()}, wysyłającej do klienta odpowiedź. W przypadku, gdy funkcja odebrała impuls, identyfikator \lstinline[style=MyCStyle]{rcvid = 0}; struktura \lstinline[style=MyCStyle]{_msg_info} nie jest modyfikowana. Przejścia stanów procesu-klienta zilustrowano na rysunku~\ref{fig:states2}

\begin{figure}[!h]
\centering
\includegraphics[width=0.5\textwidth]{img/states2}
\caption{Przejścia stanów procesu serwera}
\label{fig:states2}
\end{figure}

Przesłanie odpowiedzi na komunikat można zrealizować funkcją:

\begin{lstlisting}[style=MyCStyle]
#include <sys/neutrino.h>

int MsgReply( int rcvid,
              int status,
              const void* msg,
              int size );

int MsgReply_r( int rcvid,
                int status,
                const void* msg,
                int size );
\end{lstlisting}

gdzie 

\begin{myitemize}
\item[] \lstinline[style=MyCStyle]{rcvid} - identyfikator nadawcy, zwrócony przez funkcję \lstinline[style=MyCStyle]{MsgReceive()}.
\item[] \lstinline[style=MyCStyle]{status} - wartość zwracana przez \lstinline[style=MyCStyle]{MsgSend()}, na której jest zablokowany wątek, którego \lstinline[style=MyCStyle]{ID} jest zakodowane w \lstinline[style=MyCStyle]{rcvid}.  
\item[] \lstinline[style=MyCStyle]{msg} - wskaźnik do bufora zawierającego wysyłaną wiadomość.
\item[] \lstinline[style=MyCStyle]{info} - wskaźnik \lstinline[style=MyCStyle]{NULL}, bądź wskaźnik do struktury \lstinline[style=MyCStyle]{_msg_info}, zawierającej dodatkowe informacje o komunikacie.  
\lstinline[style=MyCStyle]{size} - liczba bajtów określająca pojemność bufora do odpowiedzi.
\end{myitemize}

Funkcja zwraca \lstinline[style=MyCStyle]{0}, gdy wykonała się poprawnie i \lstinline[style=MyCStyle]{-1}, gdy wystąpił błąd. Zanim przejdziemy do napisania prostego szkieletu aplikacji typu klient-serwer spójrzmy na rysunek~\ref{fig:dependencies} prezentujący zależności pomiędzy danymi w procesie Send/Receive/Reply. Oczywiście proces-serwer może odpowiedzieć klientowi, nie wysyłając danych. Scenariusz taki może być użyty w celu odblokowania klienta, kiedy przekazujemy tylko status poprawnego zakończenia do funkcji \lstinline[style=MyCStyle]{MsgSend()}. Realizujemy to przez wywołanie \lstinline[style=MyCStyle]{MsgReply(rcvid, EOK, NULL, 0)}. 

\begin{figure}[!h]
\centering
\includegraphics[width=0.75\textwidth]{img/dependencies}
\caption{Zależności pomiędzy danymi w~mechaniźmie komunikatów}
\label{fig:dependencies}
\end{figure}


\begin{example}{[Komunikator typu klient-serwer]} Skompilować oddzielnie program klienta i~program serwera. Najpierw uruchomić serwer, a w~następnej kolejności proces klienta z~odpowiednimi parametrami. 
\lstinputlisting[caption=Szkielet aplikacji typu klient,style=MyCStyle,label=src:client1]{src/lab8/client1.c}
\lstinputlisting[caption=Szkielet aplikacji typu serwer,style=MyCStyle,label=src:server1]{src/lab8/server1.c}
\end{example} 




\subsection{Impulsy}

W aplikacjach często występuje potrzeba powiadomienia procesu o wystąpieniu zdarzenia, ale bez blokady procesu powiadamiającego. System operacyjny QNX Neutrino oprócz synchronicznej komunikacji Send/Receive/Reply, które blokują proces klienta, do czasu, gdy serwer odpowie, wspiera szybką komunikację asynchroniczną. Narzędziem umożliwiającym taką sygnalizację są impulsy (pulses). Impulsy to małe wiadomości, które nie powodują zablokowania procesu wysyłającego. Impuls jest 40-bitowym komunikatem, który zawiera 8-bitowy kod z zakresu od 0 do 127, zdefiniowane w \lstinline[style=MyCStyle]{<sys/neutrino.h>} i~32-bitową wartość, która może być dowolnie wykorzystana przez programistę. Impulsy mogą być odbierane tak, jak inne wiadomości za pomocą funkcji \lstinline[style=MyCStyle]{MsgReceive()}.  Jeśli funkcja odbierze impuls, to zwraca identyfikator nadawcy (receive identifier) równy \lstinline[style=MyCStyle]{0}. We wspomnianym pliku nagłówkowym zdefiniowano impuls, jako następującą strukturę: 

\begin{lstlisting}[style=MyCStyle]
struct _pulse {
_uint16 type;
_uint16 subtype;
_int8 code;
_uint8 zero [3];
union sigval value;
_int32 scoid;
};
union sigval {
int sival_int;
void *sival_ptr;
};
\end{lstlisting}

Najważniejsze pola tej struktury dotyczą elementów code oraz value. Pole code identyfikuje typ impulsu, natomiast element value jest uzupełnieniem pola code i może być dowolnie ustawione, w przeciwieństwie do innych pól tej struktury. 
Impulsy można wysyłać za pomocą funkcji \lstinline[style=MyCStyle]{MsgSendPulse()}: 

\begin{lstlisting}[style=MyCStyle]
#include <sys/neutrino.h>

int MsgSendPulse ( int coid,
                   int priority,
                   int code,
                   int value );
\end{lstlisting}

gdzie

\begin{myitemize}
\item[] \lstinline[style=MyCStyle]{coid} - identyfikator połączenia, ustanowiony przez funkcję \lstinline[style=MyCStyle]{ConnectAttach()}.
\item[] \lstinline[style=MyCStyle]{priority} - priorytet impulsu. 
\item[] \lstinline[style=MyCStyle]{code} - 8-bitowy pole kodu wysyłanego impulsu, określający jego typ. Bezpieczny zakres kodu mieści się między wartością \lstinline[style=MyCStyle]{_PULSE_CODE_MINAVAIL},  a watością  \lstinline[style=MyCStyle]{_PULSE_CODE_MAXAVAIL}.
\item[] \lstinline[style=MyCStyle]{value} - 32-bitowy komunikat.
\end{myitemize}


Funkcja \lstinline[style=MyCStyle]{MsgSendPulse()} umożliwia nieblokujące przekazanie 32-bitowego komunikatu do kanału komunikacyjnego, identyfikowanego przez \lstinline[style=MyCStyle]{coid}. Funkcja zwraca \lstinline[style=MyCStyle]{-1} w przypadku błędu oraz inną wartość, gdy wykona się prawidłowo. Impulsy można odbierać za pomocą funkcji \lstinline[style=MyCStyle]{MsgReceive()}. Jeśli jednak proces ma odbierać tylko impulsy, to można zastosować funkcję \lstinline[style=MyCStyle]{MsgReceivePulse()}.  Opis funkcji można znaleźć w dokumentacji systemu. 

\begin{example}{[Komunikator typu klient-serwer-impulsy]} Skompilować oddzielnie program klienta i~program serwera. Najpierw uruchomić serwer, a w~następnej kolejności proces klienta z~odpowiednimi parametrami. 
\lstinputlisting[caption=Szkielet aplikacji typu klient,style=MyCStyle,label=src:client2]{src/lab8/client2.c}
\lstinputlisting[caption=Szkielet aplikacji typu serwer,style=MyCStyle,label=src:server2]{src/lab8/server2.c}
\end{example} 


\subsection{W jaki sposób klient znajduje serwer?}

Kiedy proces klienta komunikuje się z procesem serwera potrzebuje trzech informacji: identyfikator węzła ND, numer PID procesu serwera oraz identyfikator kanału CHID. W dotychczasowych przykładach, komunikacja występowała na lokalnym węźle, a numery PID i CHID były przekazywane z linii poleceń. W~jaki sposób, w ogólnym przypadku, klient ma uzyskać od serwera informacje o numerach ND/PID/CHID? Istnieje wiele rozwiązań tego problemu. Wybrane sposoby, wg wzrastającego stopnia ogólności podano poniżej.

\begin{myenumerate} 
\item W ustalonym i znanym miejscu serwer tworzy plik tekstowy, w którym będą przechowywane numery ND/PID/CHID. Plik ten może być następnie przeczytany przez procesy klientów. Metoda ta jest często stosowana w systemach UNIX-owych. 
\item Użycie zmiennych globalnych do przechowywania informacji o zmiennych ND/PID/CHID. Sytuacja ta jest typowa dla przypadku procesów macierzystych i potomnych, a także serwerów wielowątkowych.
\item Użycie mechanizmu globalnych nazw GNS (Global Name Service), takich jak \lstinline[style=MyCStyle]{name_attach()}, \lstinline[style=MyCStyle]{name_detach()}, \lstinline[style=MyCStyle]{name_open()} i \lstinline[style=MyCStyle]{name_close()}. 
\item Utworzyć serwer, jako menadżer zasobów. Do identyfikacji procesu serwera można użyć przestrzeni nazw plików. 
\end{myenumerate} 


Pierwsza metoda, pomimo, że dość prosta w implementacji, posiada istotne wady. Utworzone przez serwer pliki z informacjami ND/PID/CHID pozostają w pamięci nawet w przypadku, gdy proces serwera zostanie zakończony, a dane stracą swoją ważność. Może się również zdarzyć sytuacja, że system operacyjny utworzy inny proces o takich samych danych ND/PID/CHID, jak w pliku. W tym przypadku, komunikaty mogą trafić do niewłaściwego adresata i spowodować niepoprawne działanie systemu. 

Drugie podejście wymaga zastosowania zmiennych globalnych do przekazywania informacji o ND/PID/CHID. Rozwiązanie to działa w zakresie pamięci wspólnej, wykluczona jest komunikacja sieciowa. Metoda ta może być stosowana w przypadku aplikacji wieloprocesorowych i wielowątkowych. Proces (wątek) serwera tworzy kanał komunikacyjny i umieszcza w zmiennych globalnych PID/CHID, które następnie mogą być odczytane przez procesy (wątki) klientów. 

Trzecie podejście polega na zastosowaniu mechanizmu globalnych nazw (Global Name Service). Mechanizm ten działa dobrze w przypadku prostych aplikacji typu klient-serwer. 

Ostatnie rozwiązanie jest najbardziej ogólne spośród wymienionych. Proces serwera staje się menadżerem zasobów. Serwer rejestruje unikalną nazwę ścieżki dostępu, natomiast klient może wtedy wykonać prostą operację open() na tej ścieżce. Opis menadżerów zasobów można znaleźć w~dokumentacji QNX. 

\subsection{Ćwiczenia}

\begin{myenumerate}
\item Wywołać w wierszu poleceń funkcję \lstinline[style=MyCStyle]{pidin | more}. Obejrzeć wyniki, zwrócić uwagę na kolumnę STATE i BLOCKED. Pierwsza z nich wskazuje stan procesu. Druga z nich, w przypadku stanu REPLY pokazuje PID procesu serwera, od którego czekamy na odpowiedź, bądź w przypadku stanu RECEIVE pokazuje nr kanału, w którym oczekujemy wiadomości. Otworzyć perspektywę QNX System Information Perspective. Obejrzeć widoki Process Information, Connection Information i~System Blocking Graph. 
\item Na bazie szkieletu aplikacji klient-serwer, napisać program klienta, który będzie w pętli pobierał od użytkownika linie tekstu ze standardowego wejścia (użyć funkcji: \lstinline[style=MyCStyle]{char *fgets(char *str, int size, FILE *stream);} a następnie będzie przekazywał komunikat do serwera. Serwer w~odpowiedzi ma wyświetlać na ekranie komunikat i wysyłać do klienta przerobiony komunikat, opatrzony nagłówkiem (dowolnym napisem, np. ***). Niech informacje o numerach ND/PID/CHID będą przekazywane poprzez plik tekstowy. 
\end{myenumerate} 


\cleardoublepage

\section{Czas w systemie QNX Neutrino. Oprogramowanie timerów i zdarzeń}

\subsection{Wprowadzenie}

W tym rozdziale zostaną przedstawione podstawowe informacje dotyczące pomiaru czasu w systemie operacyjnym czasu rzeczywistego QNX Neutrino. Omówione zostaną timery i zdarzenia. Timery, czyli programowalne liczniki czasu, pozwalają na jednorazowe, bądź cykliczne wywoływanie określonych akcji systemu w ustalonym czasie. Do sygnalizacji takich operacji służą zdarzenia. W końcowej części rozdziału podane zostaną przykłady kodów źródłowych aplikacji typu klient-serwer, w których zaprogramowano cykliczne zdarzenia wyzwalane przez timer.

\subsection{Czas systemowy}

System QNX Neutrino zapewnia wiele usług związanych z pomiarem czasu systemowego. Procedury pomiaru czasu wiążą się z~konstrukcją sprzętową rozpatrywanego urządzenia wbudowanego. W typowym komputerze PC do pomiaru czasu służą podtrzymywane baterią układy zegara czasu rzeczywistego (na ogół rozdzielczość $1s$) lub układ oparty na generatorze kwarcowym, licznikach i systemie przerwań (rozdzielczość w zakresie $1$-$10ms$). W systemie QNX Neutrino mikrojądro dokonuje pomiaru czasu w~jednostkach zwanych taktami (\emph{tick}). Takt jest wyrażony w milisekundach. Początkowa długość taktu jest określana na podstawie częstotliwości pracy procesora. Jeśli ta częstotliwość jest większa niż 40MHz to długość trwania taktu wynosi 1ms, natomiast dla wolniejszych procesorów długość taktu jest równa 10ms. Czas w systemie QNX jest zachowany w postaci 64-bitowej liczby nanosekund, która upłynęła od $1970$ roku.

W standardzie POSIX, jak i systemie QNX Neutrino założono, że czas będzie zapisywany w strukturze \lstinline[style=MyCStyle]{timespec} postaci:

\begin{lstlisting}[style=MyCStyle]
#include <time.h>
struct timespec {
   time_t   tv_sec;
   long     tv_nsec;
}
\end{lstlisting}

\noindent
gdzie \lstinline[style=MyCStyle]{tv_sec} jest liczbą sekund, jaka upłynęła od 1970 roku, natomiast \lstinline[style=MyCStyle]{tv_nsec} to liczba nanosekund, która upłynęła od początku bieżącej sekundy. Bieżący czas urządzenia jest ustalany podczas startu systemu. Data i czas mogą być pobrane np. z zegara podtrzymywanego baterią, bądź dostarczone przez sieć. Do pobrania czasu służą funkcje z rodziny POSIX, jak również funkcje dostarczane przez mikrojądro, np.:

\begin{lstlisting}[style=MyCStyle]
#include <time.h>
int clock_gettime( clockid_t clock_id, struct timespec *tp );
\end{lstlisting}

\noindent
gdzie

\begin{myitemize}
\item[] \lstinline[style=MyCStyle]{clock_id} - identyfikator typu zegara; dostępny typ to: \lstinline[style=MyCStyle]{CLOCK_REALTIME}.
\item[] \lstinline[style=MyCStyle]{tp} - wskaźnik do struktury \lstinline[style=MyCStyle]{timespec}, gdzie przechowywany jest pobrany przez funkcję czas.
\end{myitemize}

Funkcja zwraca wartość \lstinline[style=MyCStyle]{0}, gdy sukes, a \lstinline[style=MyCStyle]{-1} gdy wystąpi błąd.


\begin{example}{[Pomiar czasu]} Przykład ilustruje pomiar czasu wykonania operacji  za pomocą funkcji \lstinline[style=MyCStyle]{clock_gettime()}. Uruchomić poniższy przykład podając jako pierwszy argument wywołania dowolną nazwę procesu, np. \lstinline[style=MyCStyle]{./zegar1 pwd}.
\lstinputlisting[caption=Pomiar czasu,style=MyCStyle,label=src:clockgettime]{src/lab9/clockgettime.c}
\end{example}

Funkcja \lstinline[style=MyCStyle]{clock_getres()} pozwala na uzyskanie rozdzielczości zegara systemowego i~zachowanie jej w~strukturze typu \lstinline[style=MyCStyle]{timespec}. Funkcja posiada następującą sygnaturę:

\begin{lstlisting}[style=MyCStyle]
#include <time.h>
int clock_getres( clockid_t clock_id, struct timespec* res );
\end{lstlisting}

Argumenty wywołania funkcji są takie same, jak w przypadku funkcji \lstinline[style=MyCStyle]{clock_gettime()}.

\begin{example}{[Pomiar rozdzielczości zegara]}
Uzupełnić przykład~\ref{src:clockgettime} o~możliwość pomiaru rozdzielczości zegara systemowego. Jaka jest rozdzielczość zegara systemowego w milisekundach?
\end{example}

Oprócz operacji pobierania czasu systemowego i~wyznaczania rozdzielczości zegara istnieje funkcja, która pozwala ustawiać czas systemowy:

\begin{lstlisting}[style=MyCStyle]
#include <time.h>
int clock_settime( clockid_t id, const struct timespec* tp );
\end{lstlisting}

gdzie parametry \lstinline[style=MyCStyle]{id} oraz \lstinline[style=MyCStyle]{tp} są identyczne z~parametrami funkcji \lstinline[style=MyCStyle]{clock_gettime()}. Funkcja \lstinline[style=MyCStyle]{clock_settime()} ustawia zegar systemowy \lstinline[style=MyCStyle]{id} na wartość przechowywaną w strukturze \lstinline[style=MyCStyle]{tp}. Poniższy przykład ilustruje zastosowanie funkcji \lstinline[style=MyCStyle]{clock_settime()}.

\begin{example}{[Zmiana czasu zegara]} Zmiana czasu zegara systemowego (uwaga: mogą być wymagane uprawnienia administratora).
\lstinputlisting[caption=Zmiana czasu zegara,style=MyCStyle,label=src:clocksettime]{src/lab9/clocksettime.c}
\end{example}

Funkcje analogiczne do wywołań POSIX-owych \lstinline[style=MyCStyle]{clock_gettime()}, \lstinline[style=MyCStyle]{clock_settime()} oraz \lstinline[style=MyCStyle]{clock_getres()} można znaleźć wśród funkcji mikrojądra QNX. Należą do nich odpowiednio funkcje: \lstinline[style=MyCStyle]{ClockTime()} oraz \lstinline[style=MyCStyle]{ClockPeriod()}. Sygnatury tych funkcji można znaleźć w stosownych dokumentacjach.  Do operacji związanych z obsługą czasu należy m.in. funkcja:

\begin{lstlisting}[style=MyCStyle]
#include <sys/neutrino.h>
#include <inttypes.h>
uint64_t ClockCycles( void );
\end{lstlisting}

Funkcja \lstinline[style=MyCStyle]{ClockCycles()} odczytuje licznik cykli procesora. Przy starcie systemu licznik ten jest ustawiany na \lstinline[style=MyCStyle]{0}, a następnie zwiększany o \lstinline[style=MyCStyle]{1}, co każdy cykl procesora. Poniższy przykład ilustruje zastosowanie funkcji \lstinline[style=MyCStyle]{ClockCycles()} do określenia liczby cykli na sekundę. Dane te są pobierane przez makroinstrukcję: \lstinline[style=MyCStyle]{SYSPAGE_ENTRY(qtime)->cycles_per_sec}.

\begin{example}{[Cykle procesora]} Odczyt licznika cykli procesora.
\lstinputlisting[caption=Cykle procesora,style=MyCStyle,label=src:cycles]{src/lab9/cycles.c}
\end{example}

Istnieje grupa funkcji, która pozwala pobierać i ustawiać bieżącą datę w systemie. Do pobierania takich informacji służy funkcja:

\begin{lstlisting}[style=MyCStyle]
#include <time.h>
time_t time( time_t* tloc );
\end{lstlisting}

gdzie \lstinline[style=MyCStyle]{tloc} to wskaźnik do typu \lstinline[style=MyCStyle]{time_t}, gdzie przechowana zostanie data bądź \lstinline[style=MyCStyle]{NULL}. Funkcja \lstinline[style=MyCStyle]{time()} zwraca czas kalendarzowy w sekundach od 1 stycznia 1970. Jeśli \lstinline[style=MyCStyle]{tloc} jest różne od \lstinline[style=MyCStyle]{NULL} to czas kalendarzowy jest również zapisywany w miejsce wskazywane przez \lstinline[style=MyCStyle]{tloc}.

\begin{example}{[Pobranie daty]} Uruchomić program w konsoli.
\lstinputlisting[caption=Pobranie daty,style=MyCStyle,label=src:date]{src/lab9/date.c}
\end{example}

W przykładzie~\ref{src:date} użyto funkcji \lstinline[style=MyCStyle]{char* ctime( const time_t* timer )}, która pozwala na zamianę czasu sekundowego na datę w postaci tekstu. W nagłówku \lstinline[style=MyCStyle]{time.h} zdefiniowano także strukturę \lstinline[style=MyCStyle]{tm}, która opisuje czas kalendarzowy. Szczegóły dotyczące pól struktury można znaleźć pod adresem: \url{http://www.qnx.com/developers/docs/6.4.1/neutrino/lib_ref/t/tm.html}. Istnieją funkcje, które pozwalają dokonywać konwersji czasu kalendarzowego ze struktury \lstinline[style=MyCStyle]{tm} do czasu sekundowego (\lstinline[style=MyCStyle]{mktime()}), bądź na łańcuch tekstowy (\lstinline[style=MyCStyle]{asctime()}).

\subsection{Opóźnienia}

W systemach czasu rzeczywistego RTOS często zachodzi potrzeba odmierzania odcinków czasu i okresowego wykonania pewnych operacji. Najprostszym sposobem osiągnięcia tego celu jest zastosowanie funkcji pozwalających na osiągnięcie opóźnień. Do grupy takich funkcji należą m.in:

\begin{myitemize}
\item[$\bullet$] \lstinline[style=MyCStyle]{unsigned int sleep( unsigned int seconds );}
\item[$\bullet$] \lstinline[style=MyCStyle]{int nanosleep( const struct timespec* rqtp, struct timespec* rmtp );}
\item[$\bullet$] \lstinline[style=MyCStyle]{unsigned int delay( unsigned int duration );}
\end{myitemize}

Wykonanie tych funkcji powoduje zawieszenie procesu (wątku), do momentu, aż minie określony w argumentach funkcji czas lub proces (wątek) otrzyma sygnał o jego zakończeniu. Czas zawieszenia procesu (wątku) jest na ogół dłuższy niż założony ze względu na strategie szeregowania, priorytet procesu, ale również ze względu na błędy pomiaru czasu. Poniższy przykład ilustruje błędny pomiar czasu za pomocą funkcji \lstinline[style=MyCStyle]{delay()}.

\begin{example}{[Błędny pomiar czasu]} Uruchomić poniższy przykład z konsoli wpisując polecenie: \lstinline[style=MyCStyle]{time ./pomiar1}, które zmierzy rzeczywisty czas wykonania programu.
\lstinputlisting[caption=Błędny pomiar czasu za pomocą funkcji delay,style=MyCStyle,label=src:delay]{src/lab9/delay.c}
\end{example}

Czas wykonania powyższego przykładu jest znacznie większy niż założone 1000ms. Skąd ta różnica? Błędy pomiaru czasu w~przykładzie i~we wspomnianych wcześniej funkcjach pochodzą z różnego rodzaju źródeł. Omówione zostaną dwa: błędy akumulacji i błędy odwzorowania czasu w komputerze.

Pierwszym źródłem jest błąd popełniany z każdym wywołaniem funkcji \lstinline[style=MyCStyle]{delay()}. W standardzie POSIX (i rozszerzeniach czasu rzeczywistego) zakłada się, że opóźnienia związane z działaniem funkcji są dopuszczalne. Niepożądane jest przedwczesne zwrócenie sterowania z funkcji \lstinline[style=MyCStyle]{delay()}. Funkcja \lstinline[style=MyCStyle]{delay()} oraz przerwania zegarowe są wywoływane asynchronicznie. Aby zapewnić, że czas opóźnienia zakładany w funkcji \lstinline[style=MyCStyle]{delay()} rzeczywiście upłynął, jądro systemu dodaje do tego czasu jeden takt (\lstinline[style=MyCStyle]{tick}). Gdyby dodatkowy takt nie został dodany, opóźnienie w funkcji \lstinline[style=MyCStyle]{delay()} trwałoby krócej, niż czas który założono.

Drugie źródło błędów popełnianych w pomiarze czasu pochodzi z niedokładności odwzorowania jednostek czasu przez sprzętowe układy pomiaru czasu. Zaprogramowane opóźnienie \lstinline[style=MyCStyle]{1ms} jest w rzeczywistości czasem mniejszym, ale najbliższym założonej wartości. Dla komputera klasy IBM PC jest to wartość równa \lstinline[style=MyCStyle]{0.999847ms}.

Opisanymi błędami mogą być obarczone też inne funkcje, takie jak: \lstinline[style=MyCStyle]{select()}, \lstinline[style=MyCStyle]{alarm()}, \lstinline[style=MyCStyle]{nanospin()}, oraz rodzina funkcji \lstinline[style=MyCStyle]{timer_*()}.

\begin{example}{[Usprawniony pomiar czasu]} Należy napisać program \lstinline[style=MyCStyle]{pomiar2}, w~którym pętla z przykładu~\ref{src:delay} będzie uruchamiana 100 razy, a czas opóźnienia będzie równy 10ms. Uruchomić przykład z~konsoli wpisując polecenie: \lstinline[style=MyCStyle]{time ./pomiar2}.
\end{example}

\subsection{Oprogramowanie timerów}

System QNX Neutrino dostarcza pełnej funkcjonalności timerów, zgodnych ze standardem POSIX. Timery są obiektami, które pozwalają na generowanie zdarzeń, które w ustalonym czasie mają uruchomić określone operacje systemu. Zastosowanie timera wymaga przeprowadzenia kilku operacji, które mogą być wykonane za pomocą funkcji dostarczanych przez QNX Neutrino. Aby użyć timer należy:

\begin{myitemize}
\item[$\bullet$] Określić rodzaj generowanych przez timer zdarzeń (impulsy, sygnały, utworzenie nowego wątku).
\item[$\bullet$] Utworzyć timer.
\item[$\bullet$] Określić sposób pomiaru czasu (absolutny lub względny) i tryb pracy timera (jednorazowy lub cykliczny).
\item[$\bullet$] Uruchomić timer.
\end{myitemize}

\subsubsection{Zdarzenia}

Zdarzenie w systemie QNX Neutrino może być jednym z występujących w Neutrino zawiadomień: impulsem, sygnałem albo zdarzeniem, polegającym na utworzeniu wątku. We wszystkich typach zawiadomień używa się zdefiniowanej w nagłówku \lstinline[style=MyCStyle]{#include <sys/siginfo.h>} struktury:

\begin{lstlisting}[style=MyCStyle]
struct sigevent {
int sigev_notify;
union {
	int sigev_signo;
	int sigev_coid;
	int sigev_id;
	void (*sigev_notify_function) (union sigval);
};
union sigval sigev_value;
union {
	struct {
	short sigev_code;
	short sigev_priority;
};
pthread_attr_t *sigev_notify_attributes;
};
\end{lstlisting}

Pole \lstinline[style=MyCStyle]{sigev_notify} określa znaczenie zawiadomienia, które zostało wysłane.

\begin{table}[h!]
\centering
\caption{Znaczenie pola  \lstinline[style=MyCStyle]{sigev_notify} w strukturze  \lstinline[style=MyCStyle]{sigevent}}
\setlength{\arrayrulewidth}{1pt}
\setlength{\tabcolsep}{6pt}
\renewcommand{\arraystretch}{1.2}
\begin{tabular}{ |p{0.3\textwidth}|p{0.5\textwidth}|}
\hline \rowcolor{gray}
\textbf{Pole} & \textbf{Znaczenie} \\ \hline
\mbox{\lstinline[style=MyCStyle]{SIGEV_PULSE}} & Wysyłanie impulsu \\ \hline
\mbox{\lstinline[style=MyCStyle]{SIGEV_SIGNAL}} & Wysyłanie do procesu sygnału \\ \hline
\mbox{\lstinline[style=MyCStyle]{SIGEV_SIGNAL_CODE}} & Wysyłanie do procesu sygnału z 8-bitowym kodem \\ \hline
\mbox{\lstinline[style=MyCStyle]{SIGEV_SIGNAL_THREAD}} & Wysyłanie do wątku sygnału z 8-bitowym kodem \\ \hline
\mbox{\lstinline[style=MyCStyle]{SIGEV_UNBLOCK}} & Odblokowanie przeterminowanego wątku \\ \hline
\mbox{\lstinline[style=MyCStyle]{SIGEV_INTR}} & Używane w przerwaniach \\ \hline
\mbox{\lstinline[style=MyCStyle]{SIGEV_THREAD}} & Utworzenie wątku \\ \hline
\end{tabular}
\label{tab:sigevent}
\end{table}

\noindent
\textbf{Wysyłanie impulsu}

W przypadku wysyłania impulsów pole \lstinline[style=MyCStyle]{sigev_notify} przybiera wartość \lstinline[style=MyCStyle]{SIGEV_PULSE}. Pozostałe pola struktury sigevent przyjmują wartość z~tabeli~\ref{tab:sigevent2}.

\begin{table}[h!]
\centering
\caption{Znaczenie pola  \lstinline[style=MyCStyle]{sigevent} w przypadku zawiadomień impulsem}
\setlength{\arrayrulewidth}{1pt}
\setlength{\tabcolsep}{6pt}
\renewcommand{\arraystretch}{1.2}
\begin{tabular}{ |p{0.3\textwidth}|p{0.5\textwidth}|}
\hline \rowcolor{gray}
\textbf{Pole} & \textbf{Znaczenie} \\ \hline
\mbox{\lstinline[style=MyCStyle]{sigev_coid}} & Identyfikator COID do którego ma być wysłany impuls \\ \hline
\mbox{\lstinline[style=MyCStyle]{sigev_value}} & 32-bitowa wartość związana z impulsem \\ \hline
\mbox{\lstinline[style=MyCStyle]{sigev_code}} & 8-bitowy kod związany z impulsem \\ \hline
\mbox{\lstinline[style=MyCStyle]{sigev_priority}} & Priorytet impulsu; wartość zero jest niedopuszczalna \\ \hline
\end{tabular}
\label{tab:sigevent2}
\end{table}

Struktura \lstinline[style=MyCStyle]{sigevent} może zostać zainicjalizowana przez następujące makro:

\begin{lstlisting}[style=MyCStyle]
SIGEV_PULSE_INIT( &event, coid, priority, code, value )
\end{lstlisting}


\noindent
\textbf{Wysyłanie sygnałów}

W przypadku wysyłania sygnałów pole \lstinline[style=MyCStyle]{sigev_notify} przybiera jedną z~wartości \lstinline[style=MyCStyle]{SIGEV_SIGNAL}, \lstinline[style=MyCStyle]{SIGEV_SIGNAL_CODE} lub \lstinline[style=MyCStyle]{SIGEV_SIGNAL_THREAD}. Pole struktury dla tego przypadku przedstawiono w tabeli~\ref{tab:sigevent3}.


\begin{table}[h!]
\centering
\caption{Znaczenie pola  \lstinline[style=MyCStyle]{sigevent} w przypadku zawiadomień sygnałów}
\setlength{\arrayrulewidth}{1pt}
\setlength{\tabcolsep}{6pt}
\renewcommand{\arraystretch}{1.2}
\begin{tabular}{ |p{0.3\textwidth}|p{0.5\textwidth}|}
\hline \rowcolor{gray}
\textbf{Pole} & \textbf{Znaczenie} \\ \hline
\mbox{\lstinline[style=MyCStyle]{int sigev_signo}} & Numer wysyłanego sygnału \\ \hline
\mbox{\lstinline[style=MyCStyle]{short sigev_code}} & 8-bitowy kod związany z sygnałem \\ \hline
\end{tabular}
\label{tab:sigevent3}
\end{table}

Struktura \lstinline[style=MyCStyle]{sigevent} może zostać zainicjalizowana przez następujące makro:

\begin{lstlisting}[style=MyCStyle]
SIGEV_SIGNAL_INIT( &event, signal )
\end{lstlisting}

Parametr \lstinline[style=MyCStyle]{signal} to numer wysyłanego sygnału. W przypadku gdy wysyłany ma być sygnał z~kodem używane jest makro:

\begin{lstlisting}[style=MyCStyle]
SIGEV_SIGNAL_CODE_INIT( &event, signal, value, code )
\end{lstlisting}

Parametr \lstinline[style=MyCStyle]{value} jest interpretowany przez procedurę obsługi sygnału. Parametr code jest 8-bitowym kodem związanym z sygnałem. Jeśli sygnał ma być wysłany do określonego wątku, to do inicjacji struktury używa się następującego makra:

\begin{lstlisting}[style=MyCStyle]
SIGEV_SIGNAL_THREAD_INIT( &event, signal, value, code )
\end{lstlisting}

\noindent
\textbf{Utworzenie wątków}

W przypadku gdy pole \lstinline[style=MyCStyle]{sigev_notify} przybiera wartość \lstinline[style=MyCStyle]{SIGEV_THREAD}, to zdarzenie będzie polegało na utworzeniu nowego wątku. W takim przypadku należy określić pola zdefiniowane w~tabeli~\ref{tab:sigevent4}.

\begin{table}[h!]
\centering
\caption{Znaczenie pola  \lstinline[style=MyCStyle]{sigevent} w przypadku utworzenia wątków}
\setlength{\arrayrulewidth}{1pt}
\setlength{\tabcolsep}{6pt}
\renewcommand{\arraystretch}{1.2}
\begin{tabular}{ |p{0.3\textwidth}|p{0.5\textwidth}|}
\hline \rowcolor{gray}
\textbf{Pole} & \textbf{Znaczenie} \\ \hline
\mbox{\lstinline[style=MyCStyle]{sigev_notify_function}} & Adres funkcji (void *)func(void *value), która będzie wywołana w nowo utworzonym wątku \\ \hline
\mbox{\lstinline[style=MyCStyle]{sigev_value}} & Parametr value przekazywany do funkcji func \\ \hline
\mbox{\lstinline[style=MyCStyle]{sigev_notify_attributes}} & Struktura atrybutów wątku, który ma być utworzony \\ \hline
\end{tabular}
\label{tab:sigevent4}
\end{table}

W tym przypadku do inicjalizacji struktury służy makro:

\begin{lstlisting}[style=MyCStyle]
SIGEV_THREAD_INIT( &event, func, value, attributes )
\end{lstlisting}

\subsubsection{Funkcje obsługi timera}

Do podstawowych funkcji operujących na timerach służą funkcje tworzenia, nastawiania, usuwania i pobierania informacji o~ustawieniach obiektu. Aby utworzyć timer należy wywołać funkcję:

\begin{lstlisting}[style=MyCStyle]
#include <signal.h>
#include <time.h>
int timer_create( clockid_t clock_id,
	struct sigevent* evp,
	timer_t* timerid );
\end{lstlisting}

\noindent
gdzie

\begin{myitemize}
\item[] \lstinline[style=MyCStyle]{clock_id } - identyfikator typu zegara; dostępny typ to: \lstinline[style=MyCStyle]{CLOCK_REALTIME}.
\item[] \lstinline[style=MyCStyle]{evp} - struktura typu \lstinline[style=MyCStyle]{sigevent} zawierająca specyfikację generowanego zdarzenia.
\item[] \lstinline[style=MyCStyle]{timerid} - wskaźnik do obiektu \lstinline[style=MyCStyle]{timer_t}, w~którym jest przechowywany nowy timer.
\end{myitemize}

Funkcja \lstinline[style=MyCStyle]{timer_create()} zwraca \lstinline[style=MyCStyle]{0}, gdy sukces, a \lstinline[style=MyCStyle]{-1}, jeśli wystąpi błąd. Utworzony w ten sposób timer jest w stanie nieaktywnym, aż do momentu, gdy wywołana zostanie funkcja \lstinline[style=MyCStyle]{timer_settime()}. Do zainicjowania struktury sigevent używa się makr systemowych: \lstinline[style=MyCStyle]{SIGEV_SIGNAL}, \lstinline[style=MyCStyle]{SIGEV_SIGNAL_CODE}, \lstinline[style=MyCStyle]{SIGEV_SIGNAL_THREAD}, \lstinline[style=MyCStyle]{SIGEV_PULSE}.

Po utworzeniu timera należy zdecydować o podstawowych parametrach obiektu, takich jak rodzaju czasu wyzwolenia zdarzenia (absolutny, względny) i tryb wyzwolenia (jednorazowy, periodyczny). Do ustawianie timera służy funkcja:

\begin{lstlisting}[style=MyCStyle]
#include <time.h>
int timer_settime( timer_t timerid,
	int flags,
	struct itimerspec* value,
	struct itimerspec* ovalue );
\end{lstlisting}

\noindent
gdzie

\begin{myitemize}
\item[] \lstinline[style=MyCStyle]{timerid} - identyfikatora timera zainicjowany przez funkcję \lstinline[style=MyCStyle]{timer_create()}.
\item[] \lstinline[style=MyCStyle]{flags} - sposób odmierzania czasu; \lstinline[style=MyCStyle]{TIMER_ABSTIME} - to czas absolutny; jeśli flaga zostanie ustawiona na \lstinline[style=MyCStyle]{0}, to określony czas jest względny.
\item[] \lstinline[style=MyCStyle]{value} - określenie nowego czasu aktywacji we wskaźniku do struktury \lstinline[style=MyCStyle]{itimerspec}.
\item[] \lstinline[style=MyCStyle]{ovalue} - określenie poprzedniego czasu aktywacji we wskaźniku do struktury \lstinline[style=MyCStyle]{itimerspec}.
\end{myitemize}

Pierwszy parametr \lstinline[style=MyCStyle]{timerid} powinien być zainicjalizowany przez funkcję \lstinline[style=MyCStyle]{timer_create()}. Drugi parametr zawiera ustawienia dotyczące sposobu odmierzania czasu: absolutny, czy względny. Jeśli przekazany zostanie parametr \lstinline[style=MyCStyle]{TIMER_ABSTIME}, to timer zostanie wyzwolony w ściśle określonym momencie. W przypadku gdy \lstinline[style=MyCStyle]{flags} jest równy \lstinline[style=MyCStyle]{0}, to czas aktywacji timera jest względny i określa się go względem bieżącej chwili. Trzeci parametr jest strukturą o następującej postaci:

\begin{lstlisting}[style=MyCStyle]
struct itimerspec {
struct timespec it_value;
struct timespec it_interval;
};
\end{lstlisting}

\noindent
gdzie

\begin{lstlisting}[style=MyCStyle]
struct timespec {
	time_t tv_sec;
	long tv_nsec;
}
\end{lstlisting}

W strukturze \lstinline[style=MyCStyle]{itimerspec} występują dwa pola. Wartość \lstinline[style=MyCStyle]{it_value} określa jak długo timer względny powinien działać lub kiedy timer absolutny powinien się wyłączyć. Ustawienie wartości \lstinline[style=MyCStyle]{it_value} na zero dezaktywizuje timer. Parametr \lstinline[style=MyCStyle]{it_interval} określa wartość cyklicznego wywołania timera. W przypadku gdy wartość \lstinline[style=MyCStyle]{it_interval} jest równa \lstinline[style=MyCStyle]{0} to timer jest jednorazowy. Timer cykliczny otrzymuje się ustawiając wartość \lstinline[style=MyCStyle]{it_value} równą \lstinline[style=MyCStyle]{it_interval} i różną od zera.

Jeśli wartość \lstinline[style=MyCStyle]{ovalue} jest różna od \lstinline[style=MyCStyle]{NULL}, to funkcja \lstinline[style=MyCStyle]{timer_setttime()} zawiera wartości poprzedniego czasu aktywacji lub zero jeśli timer został wyłączony.

\begin{example}{[Timer jednorazowy, względny]}
Przykład timera jednorazowego, względnego, który zostanie aktywowany za $5.5$ sekundy.
\begin{lstlisting}[style=MyCStyle]
it_value.tv_sec = 5;
it_value.tv_nsec = 500000000;
it_interval.tv_sec = 0;
it_interval.tv_nsec = 0;
\end{lstlisting}
\end{example}

\begin{example}{[Timer jednorazowy, absolutny]}
Timer jednorazowy, absolutny, który został aktywowany $987654321$ sekund po $1$ stycznia $1970$, czyli w czwartek, $19$ kwietnia $2001$ roku o godzinie 00:25:21.
\begin{lstlisting}[style=MyCStyle]
it_value.tv_sec = 987654321;
it_value.tv_nsec = 0;
it_interval.tv_sec = 0;
it_interval.tv_nsec = 0;
\end{lstlisting}
\end{example}

\begin{example}{[Timer cykliczny]}
Timer cykliczny, który po upływie $1$ sekundy będzie generował zdarzenia cyklicznie co $0.5$ sekundy.
\begin{lstlisting}[style=MyCStyle]
it_value.tv_sec = 1;
it_value.tv_nsec = 0;
it_interval.tv_sec = 0;
it_interval.tv_nsec = 500000000;
\end{lstlisting}
\end{example}

W celu uzyskania informacji dotyczących czasu wyzwolenia timera można zastosować następującą funkcję:

\begin{lstlisting}[style=MyCStyle]
#include <time.h>
int timer_gettime( timer_t timerid,
                   struct itimerspec *value );
\end{lstlisting}

\noindent
gdzie

\begin{myitemize}
\item[] \lstinline[style=MyCStyle]{timerid} - identyfikator timera zainicjowany przez funkcję \lstinline[style=MyCStyle]{timer_create()}.
\item[] \lstinline[style=MyCStyle]{value} - wskaźnik do struktury \lstinline[style=MyCStyle]{itimerspec}, w~której przechowywany będzie wynik. Element \lstinline[style=MyCStyle]{it_value} zawiera czas pozostały do wyzwolenia timera, a element \lstinline[style=MyCStyle]{it_interval} zawiera wartość czasu cyklicznego wyzwalania timera.
\end{myitemize}

Ostatnią funkcją służącą do obsługi timerów jest operacja kasowania timera:

\begin{lstlisting}[style=MyCStyle]
#include <time.h>
int timer_delete( timer_t timerid );
\end{lstlisting}

gdzie \lstinline[style=MyCStyle]{timerid} jest obiektem zwracanym przez funkcję \lstinline[style=MyCStyle]{timer_create()}.  Skasowany timer wraca do puli wolnych timerów i może być użyty ponownie. Często jednak istotne jest, aby anulować timer bez jego kasowania. Operację taką można przeprowadzić ustawiając wartości \lstinline[style=MyCStyle]{it_value.tv_sec=0} oraz \lstinline[style=MyCStyle]{it_value.tv_nsec=0}. Ponowne uzbrojenie timera osiąga się przez wywołanie funkcji \lstinline[style=MyCStyle]{timer_settime()}, po uprzednim ustawieniu różnych od zera czasów wyzwolenia timera.


\begin{example}{[Serwer pobudzany impulsami z timera i odbierający komunikaty]} Uruchomić program \lstinline[style=MyCStyle]{client} i \lstinline[style=MyCStyle]{serwer1}. W kodzie źródłowym serwera zaprogramowano timer względny, uruchomiony pierwszy raz po \lstinline[style=MyCStyle]{4} sekundach i wyzwalany cyklicznie co \lstinline[style=MyCStyle]{2} sekundy. Oprócz impulsów z~timera, serwer jednocześnie otrzymuje wiadomości od klienta.
\lstinputlisting[caption=Program klienta,style=MyCStyle,label=src:client]{src/lab9/client.c}
\lstinputlisting[caption=Program serwera nr 1,style=MyCStyle,label=src:server1]{src/lab9/server1.c}
\label{ex:serwerimpuls}
\end{example}

\begin{example}{[Timer cyklicznie tworzący wątki]} Uruchomić program \lstinline[style=MyCStyle]{serwer2}. Kod źródłowy klienta pozostaje taki sam, jak kod~\ref{src:client}.
\lstinputlisting[caption=Program serwera nr 2,style=MyCStyle,label=src:server2]{src/lab9/server2.c}
\label{ex:timercykliczny}
\end{example}


\subsection{Ćwiczenia}

\begin{myenumerate}
\item Uzupełnić przykład~\ref{ex:serwerimpuls} o możliwość odpowiedzi do klienta, która będzie potwierdzać otrzymanie impulsu od timera oraz pobrać czas, który pozostał do wyzwolenia timera za pomocą funkcji \lstinline[style=MyCStyle]{timer_gettime (timerID, &timeLeft)}.
\item Uzupełnić przykład~\ref{ex:timercykliczny} o możliwość przeprowadzenia dodatkowych operacji w wątku cyklicznie wyzwalanym przez timer. Umożliwić przekazanie do wątku parametru \lstinline[style=MyCStyle]{chid} i \lstinline[style=MyCStyle]{coid}. Zapisać dane dotyczące licznika w pliku.
\end{myenumerate}

\cleardoublepage

\section{QNX RTOS na platformie BeagleBone Black}





\cleardoublepage


\clearpage

\tableofcontents

\renewcommand{\listtheoremname}{Spis przykładów}
\listoftheorems[ignoreall,show=example]


\cleardoublepage
\phantomsection
%% \addcontentsline{toc}{section}{Spis rysunków}
\listoffigures

\phantomsection
%% \addcontentsline{toc}{section}{Spis tabel}
\listoftables

\cleardoublepage
\bibliographystyle{plplain}
\nocite{*}
\phantomsection \label{pismiennictwo}
%% \addcontentsline{toc}{section}{Piśmiennictwo}
\bibliography{qnxlab}

\end{document}
