\section{Wprowadzenie do QNX Momentics}


W celu opracowania programów pracujących pod kontrolą systemu operacyjnego czasu rzeczywistego (hard real time), będziemy potrzebowali Platformy Programistycznej QNX. W jej skład wchodzi pakiet QNX Momentics Tool Suite, składający się z elementów niezbędnych do rozwoju i uruchomienia oprogramowania pod QNX Neutrino - patrz rysunek~\ref{fig:qnxMomentics}. Do tej grupy należą kompilatory, linker, biblioteki i~inne komponenty systemu operacyjnego, zbudowane dla wszystkich architektur wspieranych przez QNX Neutrino. Posługując się QNX Momentics w systemie operacyjnym Windows i~Linux mamy do dyspozycji zintegrowane środowisko programistyczne na bazie projektu Eclipse.


\begin{figure}[!h]
\centering
\includegraphics[width=0.35\textwidth]{img/qnxMomentics}
\caption{Platforma rozwoju oprogramowania}
\label{fig:qnxMomentics}
\end{figure}

Dzięki platformie programistycznej możemy tworzyć oprogramowanie w konfiguracji cross development (host-target). Na maszynie typu host (Windows) będziemy dysponować platformą programistyczną QNX Momentics, natomiast na maszynie docelowej typu target (QNX Neutrino na maszynie wirtualnej) będziemy uruchamiać nasze programy. Komunikacja pomiędzy komputerem host i target odbywa się przez sieć, a wspomaga go proces \lstinline[style=MyBashStyle]{qconn}.

\begin{figure}[!h]
\centering
\includegraphics[width=0.5\textwidth]{img/konfiguracja}
\caption{Konfiguracja host-target}
\label{fig:konfiguracja}
\end{figure}

Niniejsze laboratorium będzie poswięcone kilka zagadnieniom:
\begin{myenumerate}
\item Podstawy obsługi QNX Momentics
\item Zarządzanie projektami C/C++
\item Edycja kodu źródłowego, kompilacja i~budowanie
\item Dostęp do platformy docelowej oraz uruchamianie aplikacji
\end{myenumerate}

\includepdf[scale=0.75,pages={1-8},pagecommand={\thispagestyle{fancy}{\subsection{Podstawy obsługi QNX Momentics}}},nup=2x4]{img/qnx1.pdf}
\includepdf[scale=0.75,pages={9-},pagecommand={\thispagestyle{fancy}{}},nup=2x4]{img/qnx1.pdf}


\includepdf[scale=0.75,pages={1-8},pagecommand={\thispagestyle{fancy}{\subsection{Zarządzanie projektami C/C++}}},nup=2x4]{img/qnx2.pdf}
\includepdf[scale=0.75,pages={9-},pagecommand={\thispagestyle{fancy}{}},nup=2x4]{img/qnx2.pdf}


\includepdf[scale=0.75,pages={1-8},pagecommand={\thispagestyle{fancy}{\subsection{Edycja kodu, kompilacja i~budowanie}}},nup=2x4]{img/qnx3.pdf}
\includepdf[scale=0.75,pages={9-},pagecommand={\thispagestyle{fancy}{}},nup=2x4]{img/qnx3.pdf}

\includepdf[scale=0.75,pages={1-8},pagecommand={\thispagestyle{fancy}{\subsection{Dostęp do platformy docelowej oraz uruchamianie aplikacji}}},nup=2x4]{img/qnx4.pdf}
\includepdf[scale=0.75,pages={9-},pagecommand={\thispagestyle{fancy}{}},nup=2x4]{img/qnx4.pdf}

%\clearpage
%\phantomsection
%{\label{viRef}
%\includepdf[scale=0.95,pages={1},pagecommand={\thispagestyle{fancy}},pagecommand={\thispagestyle{fancy}{\label{subsec:vi}}},lastpage=1,angle=90]{img/viRef.pdf}
%}

%\subsection{Zarządzanie projektami C}
%
%\subsection{Edycja kodu źródłowego, kompilacja i~budowanie}
%
%\subsection{Dostęp do platformy docelowej oraz uruchamianie aplikacji}
%\subsection{Ćwiczenia}

\cleardoublepage
